\chapter{Fundamentação Teórica}

\section{Sistemas de Recomendação}

Sistemas de Recomendação (SRs) se tornaram uma importante área de pesquisa a partir dos anos 90, quando começaram a
surgir os primeiros trabalhos na área de filtragem colaborativa \cite{adomavicius2005toward}. Os SRs são ferramentas
computacionais que provém sugestões de itens personalizadas aos usuários \cite{ricci2011introduction}. Isso significa
que o usuário recebe um como recomendaçnos um conjunto diferente de itens de acordo com as suas preferências e necessidades.
Nos últimos anos, o interesse na aplicação de SRs têm crescido fortemente \cite{adomavicius2005toward, beel2016towards}.
Exemplos dessas aplicações são: recomendação de Livros, CDs, DVDs, etc., em sites de e-commerce como Amazon e EBAY;
recomendações de filmes em sites como MovieLens e Netflix; recomendação de músicas em sites de streaming como Last.fm ou
Spotify; recomendação de amigos ou de postagens em redes sociais como Facebook ou Twitter.

SRs podem ser presentados formalmente como:

Onde F é a função que busca prever o interesse do usuário pelos itens existentes, U representa o conjunto dos usuários,
I representa o conjunto dos itens e R representa a lista ordenada dos itens pelo interesse previsto para o usuário ativo
(o usuário que irá receber a recomendação). O objetivo do SR então é conseguir prever de maneira mais correta, com as
informações disponíveis, os itens que serão de maior interesse do usuário.

Existem duas formas de capturar os interesses do usuário pelos itens acessados dentro do sistema: (1) Explicita, na
qual o usuário indica explicitamente o seu interesse pelo item que acabou de acessar, geralmente com uma nota 1 a 5 ou
apenas uma indicação de interesse positivo/negativo para o item; (2) Implícita, na qual o usuário não precisa indicar o
seu interesse pelo item, essa informação é capturada implicitamente através do seu comportamento e das suas interações
dentro do sistema.

Os SRs podem ser classificados de acordo com a forma como as recomendações são realizadas (abordagem). As principais
abordagens citadas na literatura são \cite{torres2004personalizaccao, adomavicius2005toward, ricci2011introduction}:
Baseada em Conteúdo, Filtragem Colaborativa, Baseada em Conhecimento e Híbrida. Nas subseções a seguir são descritas
cada uma dessas abordagens.

\subsection{Baseada em Conteúdo}

Segundo \citeonline{ricci2011introduction}, essa é uma abordagem na qual o usuário recebe recomendações de itens
similares aos que se interessou no passado. Consiste em avaliar a semelhança entre um item e os interesses do usuário.
Utilizando a nomenclatura utilizada por \citeonline{adomavicius2005toward}, os métodos dessa abordagem tentam prever o
grau de utilidade de um item para um usuário com base na utilidade que o usuário determinou para os itens similares ao item.

A abordagem Baseada em Conteúdo tem suas raízes na Recuperação da Informação \cite{adomavicius2005toward}. Para a
abordagem Baseada em Conteúdo, teremos um conjunto de atributos descrevendo um item e um conjunto de atributos
descrevendo os gostos e preferências do usuário. A descrição de um item frequentemente é realizada através de
palavras-chave definidas automaticamente por meio de algoritmos usados na área de Recuperação da Informação
\cite{adomavicius2005}. Já a descrição das preferências do usuário pode ser capturada de duas formas: implícita,
através do seu comportamento no ambiente e de itens que acessou; ou explícita, onde o usuário informa suas preferências
ao sistema, por exemplo, respondendo a questionários (ADOMAVICIUS e TUZHILIN, 2005). Dessa forma, os SRs de itens
textuais (e.g., documentos) são os que mais utilizam a abordagem Baseada em Conteúdo, devido à facilidade da aplicação
das técnicas de Recuperação da Informação nesse tipo de item.

Dentro da área de Recuperação da Informação uma forma de medir a similaridade de itens em um SR é o Cosseno. O cálculo da similaridade por Cosseno foi definido por Salton nos anos 60 (SALTON, 1964). Nessa técnica, cada documento é representado por um vetor de termos . Os vetores são dispostos em um espaço vetorial de  dimensões, onde  é o número de termos, e documentos próximos nesse espaço são considerados semelhantes. Para verificar essa proximidade utiliza-se a seguinte fórmula (MANNING et al., 2009):

Onde:  é o resultado da distância dos vetores, variando de [0,1];  é o termo presente na posição  do item 1; é o termo presente na posição  do item 2. Por exemplo, se tivermos três vetores:  representado o usuário,  e representando itens. Ao calcular a similaridade entre esses itens, temos  e , identificando que o item representado por  é mais similar às preferências do usuário c.
Outra técnica de Recuperação da Informação é o tf-idf (term-frequency inverse document frequency). Essa técnica é utilizada para identificar termos importantes em um documento (MANNING et al., 2009) e pode ser utilizada para a descoberta das palavras-chave que descrevem um item. É utilizada a seguinte fórmula para o cálculo dos pesos de cada termo do documento (MANNING et al., 2009):

Onde: representa o peso do termo  no documento ;  é o número de vezes que o termo  aparece no documento ; e o  representa o Inverse document frequency do termo , sendo o responsável por identificar termos que aparecem em muitos documentos diferentes (MANNING et al., 2009). Os termos que aparecem em muitos documentos tendem a perder sua importância. O  é calculado através da seguinte fórmula (MANNING et al., 2009):

Onde:  é o número total de documentos em uma coleção; e  é o número de documentos onde aparece o termo .
A principal vantagem da abordagem Baseada em Conteúdo é não necessitar da opinião de outros usuários para a recomendação (RICCI et al., 2011). As principais desvantagens são: o Cold-Start, em que o sistema não terá informações suficientes sobre os usuários novos para realizar uma boa recomendação; e a Superespecialização, na qual o usuário recebe sempre itens semelhantes aos que já viu (LOPS et al., 2011).

\subsection{Filtragem Colaborativa}

Nessa abordagem o usuário receberá como recomendação itens que usuários com os mesmos interesses que ele se interessaram no passado, ou seja, é a automatização do processo de "boca-a-boca" (JANNACH et al., 2011). A técnica de Filtragem Colaborativa tenta prever a utilidade  do item para o usuário, com base na utilidade do mesmo produto para um conjunto de usuários  possuidores de características semelhantes às suas (FERRO et al., 2011).
Existem duas variações básicas da Filtragem Colaborativa: User-User, onde a similaridade entre os usuários é analisada; Item-Item, onde a similaridade entre itens a serem recomendados é analisada (LOPS et al., 2011).
Para Torres (2004), que considera a variação User-User, a Filtragem Colaborativa ocorre, resumidamente, da seguinte forma:
a.  As opiniões das pessoas sobre itens são armazenadas;
b.  Baseado nessas opiniões, pessoas com perfil semelhantes (vizinhos) são agrupados;
c.  Itens bem avaliados pelos vizinhos são recomendados ao usuário.
Existem duas estratégias para medir a similaridade entre os usuários: Coeficiente de Pearson e Cosseno (TORRES, 2004). Levando em consideração que os usuários são representados pelas notas que deram aos itens, utiliza-se um cálculo matemático para medir a similaridade entre o perfil dos usuários (TORRES, 2004).
O Coeficiente de Pearson é um coeficiente bastante utilizado em modelos econômicos e mede a força do relacionamento de duas variáveis (TORRES, 2004). Esse coeficiente varia no intervalo [-1, 1], sendo "-1" indica ausência de correlação e "+1" indica forte correlação. O cálculo é então feito de acordo com a seguinte fórmula (TORRES, 2004):

Na fórmula,  representa a correlação entre o usuário  e um determinado usuário , onde:  é a avaliação do usuário  para o item ;  é a média de todas as avaliações do usuário ;  é a avaliação do usuário  para o item ;  é a média de todas as avaliações do usuário . A similaridade é calculada apenas com itens que os dois usuários avaliaram.
Considerando que os usuários podem ser representados através de um vetor das notas dadas aos itens, a estratégia do Cosseno (mostrada na seção anterior) pode ser utilizada para calcular a similaridade entre os usuários (TORRES, 2004).
Com o aumento da quantidade de usuários e de itens, se torna um desafio para a filtragem colaborativa User-User realizar uma recomendação, principalmente pela dificuldade de identificar a vizinhança com tantos usuários (JANNACH et al., 2011). A estratégia Item-Item é uma solução para ser utilizada nesse contexto, permitindo a computação das similaridades a acontecer off-line (JANNACH et al., 2011). A ideia principal da estratégia Item-Item é prever a nota que o usuário daria para um item com base nas notas que ele deu para itens semelhantes àquele. Para essa estratégia, o cálculo da similaridade pelo Cosseno, semelhante ao já citado, é uma métrica padrão e a que apresenta os melhores resultados (JANNACH et al., 2011). Esse cálculo da similaridade, ao invés de considerar as notas de cada um dos usuários, considera vetores com as notas de cada item para identificar essa similaridade.
As pessoas que apresentaram preferências similares no passado tendem a concordar no futuro (RICCI et al., 2011). Por isso essa abordagem tende a realizar recomendações que serão bem aceitas pelos usuários.
Como essa abordagem não considera a descrição dos itens e sim as notas desses, uma vantagem dessa abordagem é que as recomendações realizadas podem ser bastante interessantes e inesperadas ao usuário (RICCI et al., 2011).
Por outro lado, a abordagem colaborativa também possui a desvantagem de Cold-Start. Existem dois tipos de Cold-Start nessa abordagem (ADOMAVICIUS e TUZHILIN, 2005): o User Cold-Start e o Item Cold-Start. O User Cold-Start é a dificuldade que o sistema encontra para recomendar um item para um usuário que não avaliou nenhum item ainda. O Item Cold-Start ocorre para um novo item no sistema, que não será recomendado enquanto não for avaliado por algum usuário.
Além disso, outras desvantagens são (ADOMAVICIUS e TUZHILIN, 2005):

\begin{itemize}
\item Esparsidade: quanto maior a quantidade de usuários e de itens disponíveis, mais esparsa ficará a tabela com as notas dos usuários e mais difícil será realizar as comparações. Pode ser difícil prever com precisão usuários com os mesmos gostos, pois cada usuário poderá avaliar conjuntos muito diferentes de itens;
\item Necessidade de uma comunidade de usuários ativa: para essa abordagem é necessário ter uma grande quantidade de usuários ativas no sistema ao mesmo. No caso de um sistema com poucos usuários pode acontecer também a esparsidade pois os usuários acessaram e avaliaram itens diferentes e  não possível calcular a similaridade entre eles;
\item Ovelha Negra: para usuários que possuem gostos distintos demais, se torna um desafio realizar recomendações interessantes para ele. O principal motivo é que não conseguiremos definir outros usuários semelhantes a ele para comparar;
\item Escalabilidade: com o aumento do número de usuários, o custo computacional se torna alto;
\item Confiabilidade: esse problema se refere à confiabilidade das avaliações realizadas pelos usuários, se forem realizadas de forma incorreta irão diminuir a eficiência da abordagem. Outra coisa a ser considerada é a reputação dos usuários: usuários com maior reputação poderiam ter suas avaliações mais consideradas (maior peso) que as outras de outros usuários.
\end{itemize}

\subsection{Baseada em Conhecimento}

A abordagem Baseada em Conhecimento recomenda itens aos usuários com base no conhecimento que o sistema possui sobre como características de um item se encaixam nas necessidades de um usuário e o quão útil esse item será (RICCI et al., 2011). O sistema recebe como entrada a descrição das necessidades e interesses do usuário e o papel do sistema é realizar uma combinação entre essas necessidades e os itens.
Os SRs Baseado em Caso (Case-Based) são um exemplo de SR da abordagem Baseada em Conhecimento. Nesse sistema uma função de similaridade estima o quanto a necessidade de um usuário (descrição de um problema) combina com uma determinada recomendação (solução do problema) (RICCI et al., 2011). Essa similaridade é o grau de utilidade da recomendação.
Outro exemplo da abordagem Baseada em Conhecimento são os SR Baseados em Restrição. Nessa abordagem os itens que não atendam a certas restrições são automaticamente eliminados dos itens a serem recomendados. Segundo Ricci et al. (2011), a principal diferença entre um SR Baseado em Caso e um Baseado em Restrição está no fato de o Baseado em Caso considerar a similaridade entre as necessidades do usuário e o item enquanto a baseada em restrições possui regras específicas para tratar cada uma das necessidades do usuário.
A abordagem Baseada em Conhecimento costuma funcionar melhor que outras (e.g., Filtragem Colaborativa ou Baseada em Conteúdo) no início do desenvolvimento, porém se ela não for equipada com a capacidade de aprender mais sobre o usuário, ela será rapidamente ultrapassada por métodos mais simples que exploram a interação do usuário com o sistema (RICCI et al., 2011).

\subsection{Híbrida}

Essa abordagem utiliza uma combinação das diversas abordagens para recomendar itens ao usuário. O objetivo é reunir as vantagens das abordagens e tentar eliminar suas desvantagens (BURKE, 2002). As principais abordagens que são combinadas são a Baseada em Conteúdo e Filtragem Colaborativa, por serem as mais tradicionais e mais utilizadas (ADOMAVICIUS e TUZHILIN, 2005). Alguns exemplos de algoritmos que utilizam a abordagem híbrida foram dados por Burke (2002):

\begin{itemize}
\item Weighted: a recomendação é o resultado da execução das duas abordagens de recomendação. Essas abordagens podem ser executadas linearmente, uma após a outra, para definir os melhores itens a serem recomendados, ou cada abordagem pode ter pesos diferentes, tornando o resultado de um mais importante que o resultado do outro.
\item Switching: ocorre uma alternância entre as duas abordagens, em certos momentos uma delas é utilizada e em outros momentos a outra é utilizada. O sistema deverá possuir alguns critérios para definir qual abordagem irá utilizar.
\item Mixed: as duas abordagens mencionadas são utilizadas e os resultados aparecem em um mesmo ranking. Esse tipo de abordagem é utilizado quando se deseja realizar um grande número de recomendações diferentes simultaneamente.
\item Feature combination: considera as informações da colaboração como uma característica e utiliza a abordagem Baseada em Conteúdo para realizar a recomendação.
\item Cascade: uma abordagem é utilizada primeiro para gerar um ranking e a outra abordagem refina o resultado dado por esta.
\item Feature augmentation: uma abordagem é utiliza para produzir um ranking ou uma classificação para cada item e o resultado será considerado na execução da outra abordagem.
\end{itemize}

\section{Sistemas de Recomendação Sensíveis ao Contexto}

SRs tradicionais consideram apenas as relações entre os usuários e os itens para recomendar, mas não consideram o contexto em que os usuários estão. De acordo com Dey (2001), o contexto é qualquer informação que pode ser usada para caracterizar a situação de uma entidade. Sendo as principais entidades em SRs o usuário que irá receber uma recomendação e os itens que serão recomendados.
SRs Sensíveis ao Contexto podem ser formalmente definidos como:

Onde F é a função que prediz o interesse em um item ainda não utilizado pelo usuário, U representa o conjunto do usuários, I representa o o conjunto dos itens, C representa o contexto da interação e R representa o conjunto de itens ordenado pelo interesse previsto do usuário para os itens disponíveis.
Vários autores definem conjuntos de dimensões que podem representar o contexto (SCHILIT et al., 1994; CHEN e KOTZ, 2000; ZIMMERMANN et al., 2007) e que diferem pouco entre si. Nesse trabalho, nós seguimos a definição de Schmidt et al. (1999), que é uma das mais completas encontradas:

\begin{itemize}
\item Informações sobre o usuário, e.g., hábitos do usuário, estado emocional, etc.;
\item Ambiente social do usuário, e.g., co-localização com outros usuários, interação em redes sociais, etc.;
\item Tarefas do usuários, e.g., objetivos gerais, se é uma tarefa definida previamente (pelo professor, por exemplo) ou aleatória, etc.;
\item Localização, e.g., posição absoluta, se o usuário está em casa, no trabalho ou na universidade, etc.;
\item Condiçõs do ambiente, e.g., barulho, luminosidade, etc.;
\item Infraestrutura, e.g., velocidade da internet, tipo de dispositivo utilizado, etc.;
\item Tempo, e.g., timestamp de ocorrência de uma interação, dia da semana no qual o usuário pede uma recomendação, etc.
\end{itemize}

Adomavicius and Tuzhilin (2011) definem três paradigmas de uso das dimensões do contexto no processo de recomendação:

\begin{itemize}
\item Pré-Filtragem Contextual, na qual o contexto filtra os dados que representam o usuário e esses dados servem como entrada para um algoritmo tradicional de recomendação;
\item Pós-Filtragem Contextual, na qual uma abordagem tradicional de recomendação é utilizada para gerar uma lista de itens a ser recomendados e depois esses itens são filtrados de acordo com o contexto do usuário;
\item Modelagem Contextual, na qual o contexto é aplicado diretamente no algoritmo de recomendação, gerando um algoritmo diferente dos tradicionais.
\end{itemize}

Verbert et al. (2012) dizem que, em ambientes educacionais, as abordagens tradicionais de SRs não são suficientes para recomendar de forma apropriada para os estudantes, porque esse domínio oferece algumas características específicas que não são cobertas por essas abordagens. Por exemplo, é muito mais perigoso recomendar um item ruim para um estudante, que pode desmotiva-lo nos seus estudos, do que recomendar um produto ruim em um site de e-commerce. De acordo com Verbert et al. (2012) esse domínio requer um nível maior de personalização.
Aplicar algumas dimensões do contexto é uma alternativa para melhorar a personalização em ambientes educacionais, recomendando materiais adequados para a situação atual do usuário. Por exemplo, considerar o histórico de aprendizagem do aluno, as condições do ambiente e a acessibilidade dos recursos (VERBERT et al., 2012).
Na próxima seção é apresentado um tipo específico de SRs Sensíveis ao Contexto que utilizam a dimensão temporal para recomendar. Esse tipo de SR pode também aplicar outras dimensões do contexto em conjunto.

\section{Sistemas de Recomendação Sensível ao Tempo}

Dentre as dimensões do contexto citadas na seção 2.2, o tempo tem uma vantagem de ser fácil de capturar, considerando que praticamente todos os dispositivos tem um relógio que pode capturar o tempo no qual alguma interação ocorreu. Além disso, trabalhos na área demonstraram que o contexto temporal tem potencial para melhorar a qualidade das recomendações (CAMPOS et al., 2014). Esse tipo de SR é chamado de SR Sensível ao Tempo.
SRs Sensíveis ao Tempo são formalmente representados como:

Onde F é a função que prediz o interesse do usuário por item ainda não utilizado, U representa o conjunto de usuários, I representa o conjunto de itens, T representa o contexto temporal e R é a lista dos itens ordenada Where F is the function that predicts the rating for an unknown item, U represents the users, I represents the items, T represents time context and R representa o conjunto de itens ordenado pelo interesse previsto do usuário para os itens disponíveis.
De acordo com o dicionario Michaelis (2017), o tempo é um  "Período de momentos, de horas, de dias, de semanas, de meses, de anos etc. no qual os eventos se sucedem, dando-se a noção de presente, passado e futuro". Com essa informação é possível para um sistema computacional estabelecer uma ordem para os eventos que ocorrem.
O Tempo pode ser representado de uma variável contínua ou categórica. A representação contínua utiliza o exato momento em que os itens foram consumidos/avaliados (CAMPOS et al., 2014), por exemplo: "8 de outubro de 2017, 16:15:03". Enquanto na representação categórica as variáveis são calculadas relação a períodos de interesse (CAMPOS et al., 2014), e.g., Dias da semana: {Domingo, Segunda, Terça, ...} ou Estações do ano: {Primavera, Verão, Outono, Inverno}. Além disso, o tempo pode ser representado por diferentes unidades de tempo, e.g., segundos, minutos, horas, meses, anos, etc., e as unidades tempo são hierárquicas, e.b., um dia tem 24 horas, uma hora tem 60 minutos e 1 minuto tem 60 segundos.
Um mapeamento sistemático foi conduzido sobre os SR Sensíveis ao Tempo (BORBA et al., 2017) utilizando a metolodolia de Peterson et al. (2008). Nesse mapeamento sistemático não foi restringido apenas trabalhos na área educacional. A principal questão de pesquisa desse mapeamento foi: Como o contexto temporal é utilizado em SRs Sensíveis ao Contexto? Para responder a essa questão de pesquisa principal, três questões de pesquisa secundárias foram definas, são elas: (1) Como os algoritmos de recomendação utilizam o tempo? (2) Qual é a diferença entre o uso do tempo em diferentes domínios de aplicação? (3) Que outras dimensões são utilizadas juntamente com o contexto temporal?
Após o processo de seleção dos artigos, 88 trabalhos fizeram parte do estudo e foram considerados para responder as questões de pesquisa. Entre os resultados do mapeamento sistemático desenvolvido em Borba et al. (2017), o principal foi a definição de sete categorias de SRs Sensíveis ao Tempo. Essa categorização foi feita a partir do agrupamento dos artigos que utilizam o tempo de forma semelhante. A partir disso foi possível identificar as sete principais formas de utilizar o tempo nos algoritmos de recomendação que são descritas nas próximas subseções.

\subsection{Restriction}

Na categoria Restriction, o tempo é utilizado para restringir que itens serão utilizados. Isso significa que o SR compara variáveis de tempo relacionadas aos itens e ao usuário para restringir quais itens irão aparecer na lista de recomendações. Existe pelo menos duas formas de restrição para se utilizar: (1) o SR compara o tempo disponível pelo usuário com o tempo necessário para consumir um determinado item, e.g., a duração dos filmes que serão recomendados e o tempo que o usuário tem até o seu próximo compromisso; (2) o SR compara o tempo atual (data e hora) com o horário de funcionamento dos itens que serão recomendados, e.g., na recomendação de restaurantes onde só faz sentido recomendar locais que estejam servindo no momento.

\subsection{Micro-profile}

Na categoria Micro-Profile, o usuário possui perfis distintos para cada período de tempo. Nessa categoria, o tempo deve ser utilizado de forma categórica, onde as categorias que serão utilizadas dependem da aplicação onde for aplicada. É possível, por exemplo, que o usuário possua um perfil para dias da semana e outro perfil para finais de semana, ou então um perfil para a manhã, outro para a tarde e outro para a noite. O objetivo é que as recomendações serão realizadas considerando apenas as interações do usuário que aconteceram no mesmo contexto temporal em que ele está no momento, e.g., recomendar programas de TV para o usuário em um domingo a noite considerando apenas quais programas ele costuma acessar em um domingo a noite.

\subsection{Bias}

Na categoria Bias, o tempo é utilizado para agregar informação na matriz Usuários x Itens normalmente utllizada pela Filtragem Colaborativa. Essa matriz é comumente utilizada com apenas duas dimensões que são os Usuários e o Itens e os valores dessa matriz são as notas dadas pelos usuários para os itens. Ao incorporar o tempo nessa matriz, é possível realizar uma comparação mais precisa entre os usuários do sistema e assim prever o interesse do usuário ativo para os itens ainda não acessados. Dessa forma, usuários que avaliaram os mesmos itens com notas semelhantes e em contextos temporais semelhantes serão considerados vizinhos do usuário ativo e o algoritmo de recomendação tem uma maior chance de acertar nos interesses do usuário.

\subsection{Decay}

Na categoria Decay, o tempo é utilizado como um fator de decaimento na importância das interações do usuário, i.e., interações (itens consumidos, avaliações, etc.) mais antigas tem um peso menor para o algoritmo de recomendação do que as interações mais atuais. Os algoritmos dessa categoria consideram que o interesse do usuário varia com o tempo e é importante considerar que os interesses mais atuais do usuário representam melhor o seu perfil do que interesses mais antigos. É importante notar que as interações antigas não são ignoradas pelo algoritmo de recomendação com Decay, é apenas dado um peso menor para essas interações.

\subsection{Time Rating}

Na categoria Time Rating, o tempo é considerado pelo SR para inferir as preferências do usuário. Nessa categoria o SR utiliza uma estratégia implícita para capturar o interesse do usuário que considera o tempo que o usuário passou em determinado item. A categoria toma como princípio que itens no qual o usuário passou pouco tempo não são do seu interesse, enquanto itens em que ele passou mais tempo indicam os seus interesses. Essa forma de capturar é interessante pois o usuário não precisa explicitamente dar notas ao itens, dessa forma é possível capturar um feedback do usuário para todos os itens acessados por ele.

\subsection{Novelty}

Na categoria Novelty, o SR considera que itens mais novos serão mais relavantes para os usuários do que itens mais antigos. Nessa catoria, existem pelo menos duas estratégias que podem ser utilizadas: (1) o SR possui um threshold definido (por exemplo, duas semanas) e itens que sejam mais velhos que isso serão retirados da lista de recomendação; (2) o SR não ignora itens antigos, porém os itens novos possuem um peso maior e, se dois itens similares estiver para ser recomendados, o mais novo é o escolhido mesmo que o mais antigo esteja mais de acordo com o perfil do usuário. Essa categoria é mais comum em domínios onde novos itens tendem a ser mais relevantes que itens antigos, e.g., redes sociais, notícias, etc.

\subsection{Sequence}

Na categoria Sequence, o SR observa itens que são geralmente consumidos juntos em uma determinada ordem e utiliza essa informação para recomendar. Dessa forma, quando o SR encontra um padrão nos acessos de um usuário que já é conhecido, é possível utilizar os próximos itens da sequência como recomendações para o usuário. Essa categoria que os usuários tendem a seguir algum padrão de acesso (trajetória) enquanto interagem com o sistema.

\section{Avaliação de Sistemas de Recomendação}

A avaliação de SR são divididas em três categorias (SHANI e GUNAWARDANA, 2009):

\begin{itemize}
\item Experimentos offline: avaliação do método de recomendação através de uma base de dados, simulando as ações dos usuários sem necessitar da participação dos mesmos;
\item Estudos com os usuários: em que um pequeno grupo de usuários realiza tarefas específicas relacionadas ao SR e podem ser utilizados em conjunto com medidas qualitativas para mensurar a satisfação dos usuários, por exemplo através de questionários;
\item Uso real do sistema: na qual o SR é avaliado em situações reais de uso e os dados para avaliação quantitativa são capturados de forma automática, por exemplo ferramentas de Web Analytics.
\end{itemize}

Pu e Chen (2010) propõe um framework para a avaliação de SRs utilizando Estudos com os usuários, com o objetivo de realizar uma avaliação centrada no usuário. O framework foi proposto com base em outras ferramentas para avaliação centrada no usuário não exclusivas de SR: Technology Acceptance Model (TAM) e Software Usability Measurement Inventory (SUMI). O TAM consiste em três construtos: Facilidade de Uso Percebida, Utilidade Percebida e Intenções do Usuário em utilizar o sistema. Enquanto o SUMI consiste de cinco construtos (Eficiência, Influência, Ajuda, Controle, Capacidade de Aprendizado) e um questionário de 50 questões.
O framework proposto por Pu e Chen (2010) consiste em quatro construtos: (1) Qualidades Percebidas pelos Usuários, (2) Crenças/Opiniões do Usuário, (3) Atitudes/Propósitos do Usuário, (4) Intenções Comportamentais. Para cada um dos construtos, vários aspectos são avaliados, como pode ser visto na Figura 1. Os autores definem ainda um conjunto de 60 questões para aplicar nessa avaliação como pode ser visto no Anexo A. Nesse questionário as questões são afirmações nas quais o usuário deve ser posicionar em um escala de Likert de 5 pontos, de "Discordo totalmente" até "Concordo totalmente". Os autores ainda afirmam que o conjunto de questões aplicado pode ser reduzido para um subconjunto com 15 questões (questões com asterisco no Anexo A).

\begin{figure}[htb]
  \caption{\label{resque-framework}Construtos do framework de avaliação de SRs centrado no usuário}
  \begin{center}
      \makebox[\textwidth][c]{\includegraphics[width=1.2\textwidth]{./Figuras/resque-framework.png}}
  \end{center}
  \legend{Fonte:}
\end{figure}

\section{Apresentação das Recomendações}

No trabalho de Pu et al. (2012) os autores argumentam que apenas a eficiência do algoritmo não garante que o usuário estará satisfeito com o sistema, será leal e continuará utilizando-o ou que os itens serão "convertidos" (nesse sentido, os autores se referem a conversão como a aceitar a recomendação dada e utilizar/comprar/assistir/etc. o item recomendado). Os autores afirmam que percepção do usuário sobre a qualidade da recomendação é afetada tanto pela qualidade das recomendações, que é responsabilidade do algoritmo de recomendação, quanto pela eficiência na apresentação das recomendações, explicando a razão daquelas recomendações e inspirando a confiança do usuário nas suas decisões. Para isso, os autores defendem uma avalição do SR centrada no usuário, de forma a avaliar não somente o algoritmo de recomendação mas o SR como um todo (PU et al., 2012).
Além disso, Pu et al. (2012) definem um conjunto de vinte diretrizes para o design de um SR bem aceito pelos usuários. Essas diretrizes foram criadas a partir da combinação do resultado de vários trabalhos que executaram experimentos com participação de usuários (i.e., Estudos com usuários) para avaliar a interface de SRs. As principais diretrizes levadas em conta por esse trabalho são (PU et al., 2012):

\begin{itemize}
\item Diretriz 14: Considere aprimorar a acurácia percebida pelo usuário com um layout mais atrativo, rótulos mais efetivos, e explicando como o sistema gerou as recomendações. Fazendo isso pode aumentar a percepção do usuário sobre a eficiência do sistema, sua satisfação com o sistema em geral, sua prontidão para aceitar os itens recomendados e a sua confiança no sistema.
\item Diretriz 18: Considere fornecer como recomendação itens compatíveis ao contexto do usuário. Essa característica pode estar altamente relacionada com a percepção de utilidade do sistema e da satisfação do usuário.
\item Diretriz 19: Considere explicar o porquê do sistema recomendou determinados itens. Esses aspectos podem estar altamente relacionados com a satisfação do usuário, a percepção de controle, as intenções do usuário inspiradas pela confiança do usuário, como a intenção de retornar ao sistema.
\item Diretriz 20: Considere fornecer informação suficiente relacionadas aos itens recomendados, controlar a qualidade das informações e da estrutura de navegação.
\end{itemize}

\section{Considerações sobre o capítulo}
