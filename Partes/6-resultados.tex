\chapter{Resultados}\label{chapter:resultados}

Esse capítulo apresenta os resultados do Teste Piloto e do Experimento descritos no Capítulo \ref{chapter:experimento}.
A Seção \ref{section:resultados-teste-piloto} apresenta os resultados do Teste Piloto enquanto a Seção \ref{section:resultados-experimento}
apresenta os resultados do experimento, através do Questionário de Satisfação aplicado ao final do experimento e da Ánalise
de Uso do Sistema de Recomendação.

\section{Teste Piloto}\label{section:resultados-teste-piloto}

Como dito na Seção \ref{section:planejamento-teste-piloto}, quatro alunos que já realizaram a disciplina de Algoritmos e
Linguagem de Programação participaram do Teste Piloto. Aqui eles serão identificados como Participante 1, Participante 2,
Participante 3 e Participante 4.

Os Partipantes 1 e 2 utilizaram o algoritmo tradicional de recomendação, enquanto os participantes 3 e 4 utilizaram a
proposta desse trabalho, porém eles tinham essa informação durante a realização do Teste. Os participantes foram livres na escolha do Sistema Operacional e Navegador utilizados, bem como
no Modo de Navegação escolhido (Livre ou Tutorial). A Tabela \ref{tab:participantes-teste-piloto} apresenta o
Sistema Operacional, Navegador, Modo de Navegação e o Algoritmo de Recomendação utilizado por
cada participante.

\begin{table}[h]
\footnotesize
\caption[Dados Técnicos do Participantes do Teste Piloto]{Dados Técnicos do Participantes do Teste Piloto}
\label{tab:participantes-teste-piloto}
\centering
\begin{tabular}{|p{2cm}|p{2.5cm}|p{2.5cm}|p{2.5cm}|p{2.5cm}|}
  \hline
  \textbf{Participante} & \textbf{Sistema Operacional} & \textbf{Navegador} & \textbf{Modo de Navegação} & \textbf{Algoritmo de Recomendação} \\
  \hline
  1 & Windows 7 & Firefox & Tutorial & Baseado Em Conteúdo Tradicional \\
  \hline
  2 & Windows 7 & Firefox e Chrome & Livre & Baseado Em Conteúdo Tradicional \\
  \hline
  3 & Ubuntu & Firefox & Tutorial & Baseado Em Conteúdo com Decaimento \\
  \hline
  4 & Windows 7 & Firefox & Livre & Baseado Em Conteúdo com Decaimento \\
  \hline
\end{tabular}
\legend{Fonte: O autor.}
\end{table}

Durante a execução do Teste Piloto, os participantes encontraram erros de digitação no Termo de Consentimento Livre e Esclarecido (TCLE)
e alguns problemas nas questões da avaliação da disciplina, como perguntas de Verdadeiro e Falso com questões duplicadas e uma
questão que não possuía nenhuma resposta certa. Esses problemas encontrados foram resolvidos antes do início do experimento
desse trabalho.

Os alunos também encontraram um problema de código que aconteceu em versões antigas do Firefox, por não possuir suporte a
algumas funções do \textit{JavaScript} como atrelar ações ao evento de \textit{click} de \textit{links}. O problema encontrado foi que não estavam
sendo salvas as recomendações acessadas por esses usuários no momento em que este clicava no \textit{link}, e foi confirmado
que era um problema com o Firefox pois os próprios participantes do Teste Piloto testaram no navegador Chrome e não tiveram
o mesmo problema. Para corrigir isso, foi mudada a forma de implementação do método para salvar os \textit{clicks} nas recomendações
utilizando uma técnica chamada de \textit{Redirect Link}, na qual o usuário ao clicar na recomendação acessa primeiro um
\textit{link} interno ao \adaptweb que salva qual o \textit{link} acessado e depois redireciona o usuário para o \textit{link} externo corretamente.
Dessa forma, a implementação não depende do suporte ao \textit{JavaScript} e pode ser acessado por qualquer versão de navegadores.

Sobre as recomendações, os Participantes 1 e 2 comentaram que aparentemente os Links de Apoio recomendados para eles não
mudaram muito durante toda a interação. As vezes mudavam de ordem apenas, porém continuavam os mesmo itens. Já os Participantes
3 e 4 comentaram que ao acessar um novo conceito pelo menos 3 novos Links era recomendados, enquanto os outros dois continuavam
os mesmos da tela anterior. Esse resultado é um indício da Superespecilização presente na Abordagem Baseada em Conteúdo Tradicional,
e mostra que a proposta desse trabalho utilizando o Decaimento diminui consideravelmente esse problema.

Os Participantes também observaram que desde o primeiro acesso a disciplina eles receberam cinco links com recomendação, i.e.,
o número máximo possível. Isso mostra que, como não foi definido um limiar mínimo para a similaridade entre o perfil do usuário
e os Links de Apoio, mesmo que a similaridade seja muito pequena o algoritmo sempre irá recomendar algo para o aluno. Por
outro lado, não seria interessante adicionar um limiar para o experimento deste trabalho pois estaria adicionando uma
variável interveniente. Como trabalho futuro é possível analisar como o limiar mínimo para a similaridade pode afetar
a qualidade percebida das recomendações.

A Tabela \ref{tab:teste-piloto-dados-de-uso} apresenta os dados de uso de cada Participante do Teste Piloto. Os Itens Acessados
representam os Conceitos, Materiais Complementares e Links de Apoio acessados pelo usuário. Os Links de Apoio acessados
representam as recomendações e as recomendações geradas representam quantas vezes a página acessada pelos usuários apresentou
a área de recomendações. Esse número tende a ser mais alto pois todas as páginas de Conceitos e Materiais Complementares
apresentam a área das recomendações.

\begin{table}[h]
\footnotesize
\caption[Teste Piloto: Dados de Uso]{Teste Piloto: Dados de Uso}
\label{tab:teste-piloto-dados-de-uso}
\centering
\begin{tabular}{|p{2cm}|p{3cm}|p{3cm}|p{3cm}|}
  \hline
  \textbf{Participante} & \textbf{Itens Acessados} & \textbf{Links de Apoio Acessados} & \textbf{Recomendações Geradas} \\
  \hline
  1            & 52              & -                        & 53                    \\
  \hline
  2            & 106             & 5                        & 105                   \\
  \hline
  3            & 29              & 8                        & 23                    \\
  \hline
  4            & 58              & -                        & 64                    \\
  \hline
\end{tabular}
\end{table}

Os Participantes 1 e 4 não tiveram nenhum Link de Apoio Acessado por conta do problema citado anteriormente com o navegador
Firefox numa versão mais antiga. O Participante 2 teve o mesmo problema, mas depois mudou para o Google Chrome e teve os
seus acessos salvos. Além disso, é possível observar que o número de itens foram acessados pelos participantes é bastante
similar ao número de recomendações geradas. Isso porque geralmente um Item acessado está associado ao acesso de uma página
do Minicurso que também gera recomendações, porém também é possível na página de Materiais Complementares de um Tópico o usuário
gerar recomendação e não acessar nenhum material ou acessar mais de material. Por isso, a diferença nos números.

Ao final do Teste Piloto foi realizada uma pequena entrevista com os participantes onde eles afirmaram ter entendido e gostado
da interface do Sistema de Recomendação. Quando revelado que cada participante fazia parte de um grupo diferente e que eles
não estavam todos utilizando o mesmo Sistema de Recomendação os participantes associaram com os comentários que tinham feito
anteriormente sobre a repetição dos itens por alguns e a novidade nas recomendações por outros.

\section{Experimento}\label{section:resultados-experimento}

\lipsum[1]

\subsection{Questionário de Satisfação}

\lipsum[1]

\subsection{Análise do Uso dos Sistemas de Recomendação}

\lipsum[1]

\section{Discussão dos Resultados}

\lipsum[1]