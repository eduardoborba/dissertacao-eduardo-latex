\chapter{Trabalhos relacionados}

Todos os trabalhos descritos neste capítulo foram selecionados dentre os artigos analisados no mapeamento sistemático
da literatura \cite{de2017time} sendo considerados aqueles  estão enquadrados dentro da categoria de Decay no que
se refere ao uso do tempo no algoritmo de recomendação, independente do domínio de aplicação. No total oito trabalhos
foram selecionados, sendo apenas um da área educacional.

\section{Fan et al. 2015}

O trabalho de \citeonline{fan2015modeling} realiza a recomendação de web services, considerando a avaliação do serviço
através da medição do Quality of Service (QoS). QoS considera características do serviço como tempo de resposta,
disponibilidade, taxa de serviço, etc. Os autores consideram que a capacidade prever a qualidade de um serviço diminui
conforme o tempo que passou da última invocação desse serviço, devido a possíveis encerramento do serviço, falhas na
rede, etc. Por isso, a recomendação de serviços dos autores combina técnicas de similaridade com uma função de
decaimento que considera que a QoS diminui com o passar do tempo. O modelo de decaimento proposto considera que as
invocações mais recentes de dois usuários a um serviço devem ter um impacto maior no cálculo da similaridade entre os
usuários. O fator de decaimento é calculado utilizando a seguinte fórmula:

\begin{equation}
  \Delta t = \frac{(\Delta t_i + \Delta t_j)}{2}
  \label{eq:fan-fator-decaimento}
\end{equation}


Onde $\Delta t_i$ é o intervalo de tempo entre a invocação do serviço pelo usuário $u_i$ ao serviço e o tempo atual e
$\Delta t_j$ é o intervalo de tempo entre a invocação do mesmo serviço pelo usuário $u_j$ e o tempo atual. Assim, a
similaridade do serviço????  diminui quanto maior for o $\Delta t$. A função de decaimento de um item k é definida como:

\begin{equation}
  f(t_{i,k}, t_{j,k}) = e^{-\alpha \left|t_{atual} - \Delta t \right|}
  \label{eq:fan-funcao-decaimento}
\end{equation}

Onde $\alpha$ é uma constante de decaimento????? . Uma avaliação do serviço é considerada como a combinação do QoS com
a função de decaimento:

\begin{equation}
  r_{u_i, s_k, t} = r_{u_i, s_k} f(t_{i,k}, t_{j,k})
  \label{eq:fan-avaliacao}
\end{equation}

Os autores utilizam a avaliação combinada com a função de decaimento para calcular a similaridade do itens utilizando
um algoritmo chamado PCC????. Que pode ser definido como:

\begin{equation}
  sim(u_1, u_j, t) = \frac{\sum_{s_k \in w_{u_i, u_j}}{(r_{u_i, s_k, t} - \overline{r_{u_i}})(r_{u_j, s_k, t} - \overline{r_{u_j}})}}{\sum_{s_k \in w_{u_i, u_j}}{(r_{u_i, s_k, t} - \overline{r_{u_i}})}^2 \sum_{s_k \in w_{u_i, u_j}}{(r_{u_j, s_k, t} - \overline{r_{u_j}})}^2}
  \label{eq:fan-avaliacao}
\end{equation}


Onde $w_{u_i, u_j}$ é um conjunto dos itens em comum invocados pelos usuários $u_i$ e $u_j$. Utilizando essa fórmula de
similaridade é possível calcular a similaridade entre os usuários e encontrar os que são mais similares.

A proposta dos autores considera também a localização desses usuários para calcular a similaridade. Quanto mais próximos
eles estão, mais similares eles são considerados.

Foi realizado um experimento com o dataset WS-Dream comparando o algoritmo proposto com outros 6 algoritmos. As
métricas utilizadas para a comparação foram MAE e RMSE significado.... Primeiramente foi realizado um experimento
verificando a influência do threshold q???? nos algoritmos. Esse threshold é considerado para decidir quais são os
itens considerados significantes. O segundo experimento avaliou a influência da proporção da base de treino e de teste
no experimento. E por último, para os algoritmos que utilizavam a filtragem colaborativa (o proposto e mais 3) foi
avaliado a influência da quantidade de k-vizinhos considerada na eficiência do algoritmo. Os resultados mostraram que
o algoritmo proposto foi melhor que outros 6 algoritmos analisados.

\section{Luo et al. 2010}

O trabalho de \citeonline{luo2010context} propõe um modelo de recomendação sensível ao contexto para ambientes de
aprendizagem pervasivos. Esse modelo combina uma abordagem híbrida (baseada em conteúdo com filtragem colaborativa)
com uma personalização pelo contexto. Três tipos de contexto são definidos:

\begin{enumerate}
\item Contexto do aluno, que possui as seguintes dimensões: tipo de dispositivo, ambiente (localização), perfil
(informações pessoais como nome e afiliação), preferências (recursos pelo qual o aluno tem interesse), processo de
aprendizagem (histórico de materiais acessados), pedido de acesso (a recursos educacionais massivos).
\item Contexto do serviço, que possui as seguintes dimensões: ambiente (localização), perfil (nome, parâmetros,
retornos), QoS (parâmetros de qualidade do serviço, como carga de trabalho, reputação, disponibilidade, segurança, etc.).
\item Contexto do recurso, que segue o China ELearning Technology Standard que define as dimensões de um recurso
educacional. Esse padrão define as dimensões como sendo: perfil (informações sobre o recurso como Título, Assunto,
Palavras-chave), criador (nome, organização), audiência (tipo de educação, nível de ensino).
\end{enumerate}

O modelo de recomendação proposto pode ser dividido em dois passos: Logic-Based Resource Relevant Degree e
Situation-Based Recourse Relevant Degree.

Na etapa do Logic-based Resource Relevant Degree é feita uma análise o histórico de recursos acessados pelo aluno e as
suas preferências. Esse passo combina a abordagem baseada em conteúdo, filtragem colaborativa e os padrões sequenciais
de acesso.

A abordagem baseada em conteúdo considera as múltiplas dimensões dos recursos como assunto, assunto secundário, nível
de ensino, etc. Nessa abordagem, é inserido um conceito de Preference Energy (PE) para refletir a variação do interesse
do usuário com o passar do tempo. O PE indica que o interesse de um usuário a um item acessado diminui com o passar do
tempo. Os autores definem a diminuição da PE como sendo:

\begin{equation}
  PE_{attenuation}(x) = e^{- \lambda (x-1)}, com \ x \geqslant 1
  \label{eq:luo-preference-energy}
\end{equation}

Onde $x$ é a ordem do recurso na lista de acessos do usuário e $\lambda$ é o parâmetro de decaimento. Esse valor do PE,
combinado com as avalições feitas pelos usuários para os itens são utilizadas para gerar uma Individual Preference Tree
que auxilia o cálculo da similaridade dos recursos candidatos a serem recomendados.

A Individual Preference Tree utilizada pela abordagem baseada em conteúdo também é considerada pelo algoritmo de
filtragem colaborativa definida pelos autores para encontrar os k-vizinhos mais similares. Dessa forma, não só usuários
que acessaram os mesmos itens podem ser considerados vizinhos (como na filtragem colaborativa tradicional), mas também
usuários que acessaram itens similares entre si (mesmo assunto, palavras-chave, etc.) e os avaliaram de forma similar.

O último método de recomendação utilizado pela etapa chamada Logic-based Resource Relevant Degree utiliza os padrões
sequencias de acesso dos usuários aos recursos. O algoritmo utilizado para a mineração dos padrões sequencias é o
PrefixSpan, que procura sequências (ou subsequências) que apareceram em pelo menos  acessos. Baseado na árvore de
padrões sequenciais resultantes do algoritmo de mineração, é calculado quais os itens mais prováveis de serem acessados
de acordo a sequência atual do usuário. A proposta dos autores define que o algoritmo de mineração de sequências deve
ser executado de forma off-line, para garantir a resposta em um tempo hábil.

A etapa de Logic-based Resource Relevant Degree combina o conjunto de recursos recomendados dos três algoritmos
descritos, removendo da lista os recursos já acessados pelo usuário.

Na etapa de Situation-based Resource Relevant Degree é considerado que mesmo um recurso que combine com as preferências
do usuário pode não ser adequados para a recomendação se o contexto do usuário (dispositivo, ambiente) não for adequado
para utilizar o recurso. Para isso, no contexto dos recursos é descrito quais os dispositivos no qual a utilização do
recurso é adequada e no contexto do usuário é descrito qual o dispositivo do usuário. Também é considerado o grau de
satisfação no tempo para acessar um determinado recurso. Isso pode ser calculado pelo tamanho do recurso e a velocidade
de internet do usuário. Combinando essas duas características é possível ter uma recomendação mais adequada a situação
atual do aluno.

O algoritmo de recomendação proposto pelos autores então calcula uma lista de recursos candidatos a recomendação
utilizando os algoritmos de Logic-Based Resource Relevant Degree e remove dessa lista os recursos que não satisfaçam o
dispositivo do usuário e a satisfação mínima com o tempo de resposta esperado.

Esse algoritmo foi avaliado através de uma simulação utilizando o dataset do Movielens, onde foram adicionados dados de
contexto as interações existentes na base. As métricas utilizadas para avaliar o algoritmo foram Precisão, Utilidade e
Validade?????. Em comparação a algoritmos tradicionais de recomendação, o algoritmo proposto teve melhores resultados no
experimento realizado.

\section{Shroff, Dey e Ghosh 2014}

O trabalho de \citeonline{shroff2014enterprise} propõe um modelo para recomendação sensível ao contexto para o ambiente
corporativo. Nesse tipo de ambiente o usuário sobre uma sobrecarga cognitiva, com informações vindo através de redes
sociais, e-mails, noticias, repositórios, etc. O modelo de recomendação busca encontrar poucos itens que tenham uma alta
relevância imediata na tarefa atual do usuário, como por exemplo trocar e-mails com um fornecedor que está com atraso
na entrega ou resolver um problema técnico que um programador esteja encontrando.

As informações para o perfil do usuário são retiradas das mensagens mandadas por este e pelos documentos acessados.
Enquanto os conteúdos são extraídos do twitter, bases de gerenciamento de conhecimento, outros documentos, etc. A
recomendação dos conteúdos leva em conta uma abordagem baseada em conteúdo, combinando com o contexto do usuário.

A abordagem baseada em conteúdo proposta pelos autores utiliza a técnica do TF-IDF em janelas de tempo para identificar
termos de destaque nas ações do usuário, ou seja, termos que aparecem com uma certa frequência em uma janela de tempo
e que não aparece em outras. Isso porque itens que aparecem todo o tempo para determinado usuário devem ser algo do
domínio dele e é considerado pelos autores que não vale a pena ser recomendado.???????

Já a abordagem sensível ao contexto considera a ação atual do usuário, os conceitos presentes nessa ação (por exemplo,
termos presentes no e-mail que usuário escreveu) e a função do usuário (e.g., executivo, vendedor, programador, etc).
Para essa etapa é utiliza uma ontologia das necessidades do usuário que é human-specified, que possui regras do tipo:
ao criar uma proposta mostrar propostas anteriores de sucesso, quando uma entrega estiver atrasada procurar notícias
de eventos que podem ter afetado a entrega. Com base nessa ontologia e nas informações do contexto do usuário o
algoritmo utiliza uma rede bayesiana para o cálculo da probabilidade de cada item ser recomendado.

O algoritmo de recomendação primeiro seleciona os itens utilizando a abordagem baseada em conteúdo para filtrar apenas
os conceitos que podem ser relevantes ao usuário. Depois, os conceitos são ordenados pela probabilidade de serem
relevantes calculada pela rede bayesiana mencionada anteriormente.

Não é realizada uma avaliação do algoritmo de recomendação proposto pelos autores. Eles descrevem cenários e como o
algoritmo iria se portar no cenário com o intuito de demonstrar a utilidade do método.

\section{Benčič e Bieliková 2012}

O sistema de recomendação proposto em \citeonline{bencic2012action} busca recomendar ações aos usuários no momento que
for propício, de acordo com o contexto do usuário, e não apenas quando uma ação do interesse do usuário é encontrada.
Uma ação se refere a qualquer coisa que seja utilizada pelo usuário final de uma aplicação.

O método proposto para a recomendação representa o contexto do usuário através de símbolos, onde cada símbolo de
composto de duas partes – onde uma representa a dimensão e a outra representa a situação particular. Por exemplo,
$Clima:Limpo$. Para cada símbolo do contexto do usuário é atribuído um valor no intervalo $(0, 1)$ que indica a convicção
de que o usuário está naquele contexto.

A convicção de que o usuário está em determinado contexto é observada de tempos em tempos. Esse intervalo depende da
velocidade de conexão do dispositivo do usuário, nível da bateria, etc. A convicção do usuário estar em determinado
contexto diminui com o passar o tempo (supondo que uma nova observação demore a acontecer). Por isso, os autores
utilizam uma função de decaimento para essa convicção conforme a seguir:

\begin{equation}
  CF_t = \frac{CF_b}{(1+r)^t}
  \label{eq:bencic-conviccao}
\end{equation}

Onde $CF_t$ é a convicção calculada em função do tempo $t$, $CF_b$ é a convicção base, $r$ é o fator de decaimento e $t$
é o tempo em horas passado desde a última observação.

As ações são modeladas através de um conjunto de regras. As regras são definidas automaticamente através do feedback do
usuário e são representadas pelos antecedentes (em que situação a regra se aplica) e a consequência (a ação associada
aquela situação). As regras também possuem um decaimento na convicção com o passar o tempo. Porém, nesse caso o
decaimento não constante como para o contexto do usuário. Para as regras o fator de decaimento é calculado de forma
que não aconteça de a maioria das regras chegarem a uma convicção zero se demorar muito para uma nova observação.

Combinando as convicções nas regras criadas com as convicções no contexto do usuário, são encontradas as ações com
maior probabilidade de ser do adequada. O modelo de recomendação considera não apenas a última observação, mas sim uma
combinação das últimas observações e suas respectivas convicções (com o fator de decaimento aplicado).

A avaliação do sistema foi feita realizando simulações de possíveis interações de um usuário imaginário em um ambiente
de recomendação de notícias durante o período de um mês. Nessa simulação foi capturada a convicção de que uma
recomendação de notícias deveria ser realizada para três tipos de usuários: um que lê notícias todos os dias pela
manhã, um que lê notícias apenas nas segundas pela manhã e sextas a noite e outro que começa lendo as notícias apenas
nas segundas pela manhã e muda o seu comportamento com o passar do tempo para a leitura as sextas a noite. Os resultados
mostraram que o método conseguiu compreender o comportamento dos três tipos de usuário com uma precisão e um recall de
quase 100\%.

\section{Hawalah e Fasli 2014}

O trabalho de \citeonline{hawalah2014utilizing} propõe um método de recomendação utilizando o contexto do usuário
representado através de ontologias. O algoritmo proposto pode se adequar a diversos domínios, de forma que o contexto
seja incorporado aos interesses do usuário independente do que são os itens que serão recomendados. Além disso, o método
considera não só o contexto atual do usuário, mas também os contextos capturados anteriormente. Os autores separam o
método em três fases: Extração da informação, Aprender o perfil do usuário e Personalização.

A etapa de Extração da informação é realizada por um agente de captura dos dados que é genérico o suficiente para ser
adaptado de acordo com o domínio. Em determinados domínios pode ser utilizado uma coleta perguntando explicitamente os
interesses e o contexto ao usuário, enquanto em outros domínios é mais adequado capturar de forma implícita pela
navegação do usuário.

A informação bruta capturada (seja de forma explícita ou implícita) é processada pelo agente extrator, responsável por
extrair informação de mais alto nível. Esse agente está associado a dois tipos de bases de conhecimento: ontologias e
taxonomias. O agente realiza um mapeamento os itens que o usuário demonstrou interesse em conceitos da ontologia de
referência, enquanto também extrai dimensões do contexto de mais alto nível utilizando-se das taxonomias de contexto.

A segunda etapa, responsável por compreender o perfil do usuário, utiliza a abordagem de Pré-filtragem Contextual para
definir qual a parte do perfil do usuário é relevante. É utilizada um método similar aos micro-perfis, onde as
informações do perfil do usuário (itens acessados, notas dadas) que aconteceram em contextos similares ao atual são
consideradas mais relevantes para a recomendação. Para isso, é calculado a importância dos conceitos em cada contexto
possível, de acordo com as informações do perfil do usuário. Nesse cálculo, é considerada a frequência com que o
conceito aparece naquele determinado contexto, bem como a frequência com que esse conceito aparece em outros contextos
e a frequência com que outros conceitos aparecem nesse contexto. Detalhes são descritos em \cite{hawalah2014utilizing}.

Ainda no cálculo da importância do conceito em determinado contexto, é considerado que os interesses do usuário podem
mudar com o tempo. Para isso, é incluído na fórmula o fator de Recência (do inglês Recency), de forma que os interesses
demonstrados pelo usuário mais recentemente são considerados mais importantes. A fórmula a seguir é a responsável pelo
cálculo da recência:

\begin{equation}
  Recency(c_j, ce_l) = \frac{1}{(1+\log(d_t - d_l) \times \alpha)}
  \label{eq:hawalah-recencia}
\end{equation}

Onde $d_t$ é a data de inicialização do cálculo, $d_l$ é a data da última ocorrência do conceito $c_j$ no contexto
$ce_l$ e $\alpha$ é o fator de decaimento.

Dessa forma, como resultado dessa etapa temos os interesses do usuário em cada contexto e o peso de cada um. Essas
informações são utilizadas para construir ontologias de perfil contextual personalizada (CPOP, do inglês, contextual
personalized ontology profile), sendo uma CPOP para cada contexto.

Na terceira etapa de personalização, a ontologia gerada na etapa anterior é utilizada para inferir outros conceitos que
o usuário pode ter interesse, além dos já presentes do seu perfil. Isso é realizado utilizando a técnica de Spreading
Activation, descrita em \citeonline{hawalah2014utilizing}. Utilizando essa técnica, é gerada uma lista de recomendações
para cada CPOP, que são combinadas para gerar a lista final de recomendações para o usuário.

A avaliação do trabalho de \citeonline{hawalah2014utilizing} foi um Estudo com Usuários, visando uma avaliação centrada
no usuário como descrito por \citeonline{kelly2009methods}. O objetivo da avaliação era verificar se a recomendação
contextual proposta no trabalho fornece uma recomendação mais eficiente do que os métodos tradicionais. 24 usuários
participaram da avaliação, onde eles utilizaram um sistema por 30 dias, numa estratégia Between-subjects, i.e., cada
grupo testa apenas uma versão do sistema. No total, 4 versões do sistema foram testadas: o algoritmo proposto (CAPS), o
algoritmo proposto sem o uso do contexto (CAPS-C), um método de recomendação personalizado chamado aqui Simple-P e um
método de recomendação não personalizado chamado de Non-P.

O resultado da avaliação analisou a nota dada pelos usuários para os itens em uma escala de likert de 1 a 4. Sendo itens
com grau 1 e 2 considerados uma recomendação ruim e os itens com grau 3 e 4 considerados uma boa recomendação. Com isso
foi possível calcular a Precision at N (P@N). O algoritmo proposto possível o melhor resultado de P@N entre os
algoritmos testados.

\section{Qiao e Zhang 2012}

O trabalho de \citeonline{qiao2015personalized} propõe um algoritmo de recomendação que considera as informações
contextuais disponíveis em dispositivos móveis, como tempo, localização, tipo do dispositivo, etc. O algoritmo de
recomendação é genérico, ou seja, sem um domínio de aplicação definido.

O objetivo dos autores é combinar a filtragem colaborativa com o contexto do usuário, considerando a variação temporal
nos interesses do usuário. Para tal, são encontrados os k usuários com os interesses similares ao usuário atual,
considerando o contexto do usuário, através de técnicas de clusterização. Após encontrados os k vizinhos, utiliza-se
uma fórmula para a predição das notas para itens ainda não acessados pelo usuário com a incorporação de uma função
temporal. A fórmula é a seguinte:

\begin{equation}
  p_{u,i} = \overline{r_u} + \frac{\sum_{v \in U}{sim(u, v)(r_{u,i} - \overline{r_v})f(t_{ni})}}{\sum_{v \in U}{sim(u, v)f(t_{ni})}}
  \label{eq:qiao-predicao}
\end{equation}

Sendo $p_{u,i}$ a nota prevista do usuário $u$ para o item $i$, $\overline{r_u}$ a média das notas dadas pelos usuário
$u$, $sim(u, v)$ a similaridade entre o usuário $u$ e o seu vizinho $v$, $r_{u,i}$ o grau de interesse do usuário $u$
pelo item $i$, $t$ é o tempo no qual o usuário $u$ quer utilizar o item $i$. ?????? Não entendi A função $f$ é a função
exponencial de decaimento que representa a diminuição do interesse do usuário por um determinado item. A função de
decaimento é definida como:

\begin{equation}
  f(t_{ni}) = e^{-t_{ni}}
  \label{eq:qiao-funcao-decaimento}
\end{equation}

Os itens com as notas previstas mais altas são recomendados ao usuário.

A avaliação desse algoritmo foi realizada utilizando a base do MovieLens – 100 K, através da métrica MAE (Erro Absoluto
Médio). Quando comparado a filtragem colaborativa tradicional o algoritmo proposto alcançou melhores resultados, ou
seja, teve uma taxa de erro menor.

\section{Kushwaha et al. 2016}

O trabalho de \citeonline{kushwaha2016inclusion} propõe uma versão modifica da técnica de Joint Matrix Factorization
para uso em um sistema de recomendação de músicas incorporado ao Last.fm. A proposta busca reduzir a esparsidade dos
dados e melhorar a qualidade das recomendações. Para isso, é incorporado informações como descrição dos itens, perfil
do usuário e o seu contexto na matriz das notas dadas pelos Usuários para os Itens comumente utilizada na filtragem
colaborativa. Como a matriz dessa forma possui uma maior dimensionalidade e complexidade, a técnica de fatoração de
matrizes é essencial para reduzir a dimensão desta e permitir a extração de informações latentes importantes para a
recomendação.

Além disso, o algoritmo proposto por \citeonline{kushwaha2016inclusion} considera a variação temporal da informação.
Os autores consideram o decaimento da importância das tags colocadas pelo usuário nos itens com o passar do tempo. A
fórmula utilizada pelos autores para representar o decaimento foi baseada em \citeonline{iofciu2009time}, que pode ser
vista a seguir:

\begin{equation}
  postScore_i = \lambda^{\Delta Time_i}
  \label{eq:kushwaha-funcao-decaimento}
\end{equation}

Onde $postScore_i$ é a importância temporal da tag $i$, $\lambda$ é o fator de decaimento, que deve ser menor do que 1
e foi utilizado por Kushwaha et al. (2016) como 0.9 e $\Delta Time_i$ é o tempo passado desde a interação. Além disso,
é considerada a especificidade da tag para a nota final da tag. A fórmula da especificidade é definida como:

\begin{equation}
  tagSpecificity_i = \log(50+tagCont_i)
  \label{eq:kushwaha-especificidade}
\end{equation}

Onde $tagSpecificity_i$ é o fator de especificidade da tag e $tagCount_i$ representa quanta vezes a tag foi adicionada
ao item $i$. A nota final da tag é:

\begin{equation}
  tagScore_i = \frac{\sum_1^n{postScore}}{tagSpecifity_i}
  \label{eq:kushwaha-nota-final}
\end{equation}

A fórmula acima combina a importância temporal da tag com a sua especificidade. A nota final da tag é incorporada na matriz latente resultado da fatoração de matrizes para servir como fator de decaimento para a importância das tags, representando a variação nos interesses do usuário.
O algoritmo proposto pelos autores foi avaliado utilizando uma base de dados do próprio Last.fm, combinado com a base de dados do DBpedia para a captura de informações sobre os artistas, compositores e músicas. A avaliação considerou a métrica RMSE (root mean square error) para as previsões para as notas que o usuário daria para um determinado item comparada com a nota real. Quando comparado com outros dois algoritmos recentes de recomendação que incorporam informação social para recomendação, o algoritmo proposto por \citeonline{kushwaha2016inclusion} se saiu melhor em 3 das 6 condições de experimento realizadas.

\section{Wei, Khoury e Fong 2013}

\citeonline{wei2013web} descrevem uma proposta de recomendação que utiliza a filtragem colaborativa e que aplica um
decaimento temporal na importância das interações dos usuários. No trabalho, é proposto um serviço para a recomendação
de propagandas em redes sociais, que leva em consideração a confiança entre usuários, a reputação dos usuários e as
relações entre usuários. Os autores afirmam que usuários com uma recomendação gerada com base em usuários conhecidos
pelo usuário atual serão mais bem aceitas, assim como usuários no qual ele confia e nos usuários com alta reputação
(especialistas).

O algoritmo de recomendação considera que a importância da relação entre os usuários diminui gradativamente com o tempo.
Então, um comentário realizado hoje deve ter um peso maior que um comentário realizado ha um mês atrás quando for
avaliada a relação entre dois usuários. O decaimento é incluído diretamente na formula utilizada para realizar a
comparação de similaridade entre dois itens, como pode ser visto a seguir:

\begin{equation}
  s_{i,j}(t) = \frac{\sum_{u \in U_i^t \cap U_j^t}{(f_{ui}^\alpha(t) \cdot r_{ui})(f_{uj}^\alpha(t) \cdot r_{uj})}}{\sqrt{\sum_{u \in U_i^t}{(f_{ui}^\alpha(t) \cdot r_{ui})}^2 \sum_{u \in U_j^t}{(f_{uj}^\alpha(t) \cdot r_{uj})}^2}}
  \label{eq:wei-similaridade}
\end{equation}

Onde a função $f$ é definida pelos autores como relevância temporal do item e pode ser vista como:

\begin{equation}
  f_{uj}^\alpha(t) = e^{- \alpha (t - t_{ui})}
  \label{eq:wei-relevancia-temporal}
\end{equation}

Onde fator $\alpha$ é responsável por controlar a taxa de decaimento, $t$ representa o tempo atual e $t_{ui}$ representa
o tempo no qual o usuário $u$ utilizou o item $i$.

Além do decaimento na relevância das relações entre os usuários, os autores também incluem a confiança entre os
usuários e a reputação de especialistas da área na formula de similaridade visando melhorar a qualidade das
recomendações.

O algoritmo proposto foi avaliado utilizando três bases de dados: MovieLens, Facebook e Delicious. O algoritmo foi
comparado a outros dois algoritmos: Filtragem Colaborativa usando correlação de Pearson e usando correlação de Pearson
com efeito temporal. As métricas utilizadas foram Mean Absolute Error e Root Mean Square Deviation. Os resultados
mostraram que o algoritmo proposto melhorou significativamente a qualidade das recomendações.

\section{Considerações sobre o capítulo}
