\chapter{Considerações Finais}\label{chapter:conclusoes}

Sistemas de Recomendação (SR) são ferramentas de software que sugerem itens para os usuários de forma automatizada e personalizada,
sem a necessidade do usuário formular uma consulta para encontrar os itens do seu interesse. Esses sistemas são
explorados em Ambientes Virtuais de Aprendizagem (AVA) com o objetivo de reduzir alguns problemas existentes nesses ambientes
quando a quantidade de materiais disponíveis é grande, tais como: sobrecarga cognitiva, dificuldade de encontrar os materiais
do seu interesse e muitos materiais nunca serem utilizados.

Pesquisadores da área argumentam que os algoritmos de SRs tradicionais não são suficientes para os AVAs \cite{verbert2012context, drachsler2015panorama},
sendo necessário um nível maior de personalização a situação do usuário, como considerar dimensões do contexto. Para isso,
em \citeonline{de2017time} foi realizado um mapeamento sistemático da literatura com o objetivo de identificar como os SRs Sensíveis
ao Contexto Temporal (também chamados de SR Sensíveis ao Tempo) são utilizados. Nesse estudo, foram considerados todos os domínios
de aplicação e não apenas o domínio educacional.

Foi observado que, dos 88 artigos que utilizam esse tipo de SR, apenas quatro são aplicados no domínio educacional e esses
trabalhos carecem em avaliações em ambientes reais de uso ou que utilizem bases de dados educacionais. Analisando esses
88 artigos, também foi possível categorizar os SRs propostos nesses trabalhos pela forma que eles utilizam o tempo para
a recomendação. As sete categorias criadas como resultado do mapeamento são apresentadas na Seção \ref{section:sr-sensivel-tempo}.

O objetivo desse trabalho é a criação de perfis de usuário que levem em conta a mudança dos interesses destes usuários
ao longo do tempo. Esses perfis considerando o contexto temporal serão aplicados no algoritmo de recomendação proposto nesse
trabalho. Dentre as categorias de SR Sensíveis ao Tempo presentes na Seção \ref{section:sr-sensivel-tempo}, a proposta
desse trabalho se encaixa no \textit{Decay}.

O algoritmo proposto no Capítulo \ref{chapter:proposta} combina a (1) similaridade do perfil do usuário (representado
pelos materiais acessados pelo usuário) com os itens disponíveis para a recomendação com a (2) recência dos materiais
acessados pelo usuários, além da (3) informação se aquele item disponível para a recomendação já foi acessado ou não. A
proposta leva em conta que o ritmo de estudo dos alunos pode ser diferente, portanto a recência é considerada em relação
a sequência de itens acessados e não ao tempo absoluto (em segundos) desde o acesso. Dessa forma, para cada aluno o
decaimento acaba sendo personalizado ao seu ritmo de estudo. Também é considerado que itens já acessados podem ser
recomendados novamente, porém esses itens tem um probabilidade menor de ser recomendados do que itens ainda não acessados.

Como continuação desse trabalho está a etapa de implementação da proposta e a experimentação utilizando um ambiente
real de uso. A proposta desse trabalho será incorporada ao ambiente \adaptweb e será avaliado através de um minicurso de
algoritmos que é ministrado no ambiente todo semestre, no qual participam alunos da primeira fase dos cursos
do Centro de Ciências Tecnológicas (CCT) da Universidade do Estado de Santa Catarina (UDESC). O algoritmo proposto será
comparado a abordagem Baseada em Conteúdo tradicional através de um experimento utilizando um estratégia \textit{Between Subjects}.

O objetivo do experimento é verificar se existe diferença na percepção do usuário sobre a qualidade das recomendações do
algoritmo proposto em relação a abordagem Baseada em Conteúdo tradicional. A percepção do usuário será capturada utilizando
o questionário proposto por \citeonline{pu2011user} para identificar a percepção do usuário sobre a qualidade das recomendações,
presente no Anexo \ref{ane:questoes-framework}.

\section{Cronograma}

O cronograma proposto para a execução do restante desse trabalho pode ser visto no Apêndice \ref{ape:cronograma}. A etapa
de implementação da interface e dos algoritmos de recomendação está previsto para começar no mês de Dezembro e ir até
a metade do mês de Fevereiro. O período de preparação para o experimento irá acontecer da metade de Fevereiro até o
final do mês de Março, incluindo a mobilização de participantes para o experimento. O experimento propriamente dito está
previsto para acontecer durante o mês de Abril e até a metade do mês de Maio. Da metade do mês de Maio até o final do mês de Junho
está previsto para ser realizadas as análises dos resultados e a finalização da dissertação.

\section{Resultados Parciais}

Como resultados parciais desse trabalho temos as seguintes publicações:

\begin{itemize}
\item BORBA, E. J.; GASPARINI, I.; LICHTNOW, D. Sistema De Recomendação Para Um AVA Que Leva Em Conta Necessidades De Curto E Longo Prazo Dos Usuários. Congresso Internacional de Informática Educativa (TISE), Santiago (Chile). v. 12, p. 444-448., 2016.
\item BORBA, E. J.; Gasparini, I.; LICHTNOW, D. Time-Aware Recommender Systems: A Systematic Mapping. International Conference on Human-Computer Interaction (HCI), Vancouver, Part II, LNCS 10272, v. II, p. 464-479, 2017.
\item BORBA, E. J.; GASPARINI, I.; LICHTNOW, D. The Use of Time Dimension in Recommender Systems for Learning. Proceedings of the 19th International Conference on Enterprise Information Systems (ICEIS), Porto (Portugal) 2017. v. 2. p. 600-609.
\item BORBA, E. J.; GASPARINI, I.; LICHTNOW, D. Sistema de Recomendação Sensível ao Tempo em Ambientes Educacionais. IV Workshop de Teses e Dissertações em IHC (WTD-IHC), Joinville, 2017.
\end{itemize}