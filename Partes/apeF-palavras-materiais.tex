\chapter{Palavras-chave dos materiais no Minicurso de Algotimos e Linguagem de Programação}\label{ape:dicionario-palavras-chave}

\section{Links de Apoio}

\begin{longtable}{| p{.10\textwidth} | p{.45\textwidth} | p{.45\textwidth} |}
\hline
\#  & Link de Apoio                                                                                                                                                                                                                & palavras\_chave                                                                                                                                   \\ \hline
1   & \href{http://academicotech.blogspot.com.br/2014/02/v-behaviorurldefaultvmlo.html}{Lógica de Programação - Fluxograma e Portugol                                       } & Logica de programacao, Fluxograma, Portugol, Pseudocodigo, Operadores, Blocos, Tipos de dados, Variaveis,                                         \\ \hline
2   & \href{http://blog.academiadocodigo.com.br/2014/12/macas-e-laranjas-diferencas-entre-compilador-e-interpretator/}{Maçãs e laranjas: diferenças entre compilador e interpretator                       } & Compilador, Interpretador                                                                                                                         \\ \hline
3   & \href{http://blog.triadworks.com.br/por-que-aprender-a-programar}{Por que aprender a programar?                                                       } & Aplicacao de algoritmo, Programa de computador                                                                                                    \\ \hline
4   & \href{http://br.ccm.net/faq/9709-algoritmo-definicao-e-introducao}{Algoritmo: Definição e introdução                                                   } & Algoritmo, Linguagem de maquina, Compilador, Linguagem de programacao                                                                             \\ \hline
5   & \href{http://coral.ufsm.br/pet-si/index.php/os-beneficios-e-o-porque-de-aprender-a-programar/}{Os Benefícios e o Porquê de Aprender a Programar.                                   } & Aplicacao de algoritmo, Algoritmo, Instrucao, Programa de computador                                                                              \\ \hline
6   & \href{http://dcm.ffclrp.usp.br/~augusto/teaching/ici/Vetores-Matrizes.pdf}{Vetores e Matrizes                                                                  } & Vetor, Matriz, Contantes, Entrada, Saida                                                                                                          \\ \hline
7   & \href{http://download2.nust.na/pub4/sourceforge/v/vi/visualg30/IP\_03\_VisuALG\_Repeticao.pdf}{VisuALG - estrutura de repetição                                                    } & Estrutura de repetição, Enquanto, Para, Repita                                                                                                    \\ \hline
8   & \href{http://download2.nust.na/pub4/sourceforge/v/vi/visualg30/IP\_02\_VisuALG\_Basico.pdf}{Introdução ao VisuALG                                                               } & Portugol, Entrada, Saida, Operadores aritmeticos, Precedencia de operadores, Operadores logicos, Estrutura condicional                            \\ \hline
9   & \href{http://download2.nust.na/pub4/sourceforge/v/vi/visualg30/IP\_04\_VisuALG\_Arrays.pdf}{VisuALG – Arrays e Strings                                                          } & Portugol, Vetor, Tipo texto                                                                                                                       \\ \hline
10  & \href{http://eletrica.ufpr.br/~rogerio/visualg/Help/linguagem.htm}{A Linguagem de Programação do VisuAlg                                               } & Portugol, Pseudocodigo, Tipos de dados, Variaveis, Constantes                                                                                     \\ \hline
11  & \href{http://fabrica.ms.senac.br/2013/06/algoritmo-estrutura-de-vetores-e-matrizes/}{Algoritmo: Estrutura de vetores e matrizes.                                         } & Portugol, Entrada, Saida, Vetor, Matriz                                                                                                           \\ \hline
12  & \href{http://fig.if.usp.br/~esdobay/c/c.pdf}{Programação em C                                                                    } & Linguagem C, Variaveis, Constantes, Operadores, Vetor, Matriz                                                                                     \\ \hline
13  & \href{http://knoow.net/ciencinformtelec/informatica/linguagem-maquina/}{Linguagem máquina                                                                   } & Linguagem de maquina                                                                                                                              \\ \hline
14  & \href{http://linguagemc.com.br/estruturas-de-decisao-encadeadas-if-else-if-else/}{Estruturas de decisão encadeadas – if – else – if – else                            } & Estrutura condicional, Decisao Encadeada, Se                                                                                                      \\ \hline
15  & \href{http://linguagemc.com.br/loop-infinito-em-c/}{Loop infinito em C                                                                  } & Linguagem C, Estrutura de repeticao, Para, Repita                                                                                                 \\ \hline
16  & \href{http://marmsx.msxall.com/cursos/c3.html}{Começando a programar                                                               } & Linguagem C, Sintaxe, Semantica                                                                                                                   \\ \hline
17  & \href{http://nerdsti.com.br/?p=259}{Lógica de Programação - Vetores e Matrizes                                          } & Vetor, Matriz, Sintaxe                                                                                                                            \\ \hline
18  & \href{http://wiki.icmc.usp.br/images/5/57/Estruturas\_Controle\_I\_SCC0120\_v2.pdf}{Introdução à Ciência da Computação - Estruturas de Controle – Parte I               } & Portugol, Estrutura condicional, Decisao simples, Decisao composta, Decisao encadeada                                                             \\ \hline
19  & \href{http://www.berriel.com.br/ltpi/aula01/aula01.htm}{Conceito e formas de representação de algoritmos                                    } & Algoritmo, Tipos de algoritmo, Narrativa, Fluxograma, Pseudocodigo                                                                                \\ \hline
20  & \href{http://www.bosontreinamentos.com.br/logica-de-programacao/12-logica-de-programacao-desvio-condicional-aninhado-se-entao-senao-se/}{Lógica de Programação – Desvio Condicional Aninhado (SE…ENTÃO…SENÃO…SE)             } & Portugol, Estrutura condicional, Decisao composta, Se                                                                                             \\ \hline
21  & \href{http://www.cprogressivo.net/2013/03/O-que-e-alocacao-dinamica-de-memoria-em-Linguagem-C.html}{O que é e para que serve alocação dinâmica                                          } & Linguagem C, Vetor                                                                                                                                \\ \hline
22  & \href{http://www.cprogressivo.net/p/arquivos-em-c.html}{Arquivos em c - tutorial completo                                                   } & Linguagem C                                                                                                                                       \\ \hline
23  & \href{http://www.cristiancechinel.pro.br/my\_files/algorithms/bookhtml/node18.html}{Linguagem Natural                                                                   } & Linguagem natural                                                                                                                                 \\ \hline
24  & \href{http://www.cristiancechinel.pro.br/my\_files/algorithms/bookhtml/node19.html}{Linguagem de Máquina e Assembler                                                    } & Linguagem de maquina                                                                                                                              \\ \hline
25  & \href{http://www.cristiancechinel.pro.br/my\_files/algorithms/bookhtml/node43.html}{Expressões Lógicas                                                                  } & Operadores logicos                                                                                                                                \\ \hline
26  & \href{http://www.cristiancechinel.pro.br/my\_files/algorithms/bookhtml/node44.html}{Operadores Relacionais                                                              } & Operadores relacionais                                                                                                                            \\ \hline
27  & \href{http://www.dainf.cefetpr.br/~robson/prof/common/c/aspec.htm}{Aspectos básicos de linguagem C                                                     } & Linguagem C, Tipos de dados, Variaveis, Constantes, Entrada, Saida, Operadores                                                                    \\ \hline
28  & \href{http://www.dei.estt.ipt.pt/portugol/node/10}{Tipos de dados \textgreater{}\textgreater Constantes                                } & Constantes, Portugol                                                                                                                              \\ \hline
29  & \href{http://www.dei.estt.ipt.pt/portugol/node/14}{Entrada/Saída \textgreater{}\textgreater Ler                                        } & Entrada, Portugol                                                                                                                                 \\ \hline
30  & \href{http://www.dei.estt.ipt.pt/portugol/node/15}{Entrada/Saída \textgreater{}\textgreater Escrever                                   } & Saida, Portugol                                                                                                                                   \\ \hline
31  & \href{http://www.dei.estt.ipt.pt/portugol/node/17}{Operadores Aritméticos                                                              } & Operadores aritmeticos, Portugol                                                                                                                  \\ \hline
32  & \href{http://www.dei.estt.ipt.pt/portugol/node/21}{Operadores Lógicos                                                                  } & Operadores logicos, Portugol                                                                                                                      \\ \hline
33  & \href{http://www.dei.estt.ipt.pt/portugol/node/22}{Operadores Relacionais                                                              } & Operadores relacionais, Portugol                                                                                                                  \\ \hline
34  & \href{http://www.dei.estt.ipt.pt/portugol/node/24}{Decisão \textgreater{}\textgreater Se                                               } & Se, Estrutura condicional, Portugol                                                                                                               \\ \hline
35  & \href{http://www.dei.estt.ipt.pt/portugol/node/25}{Decisão \textgreater{}\textgreater Escolhe                                          } & Escolha, Estrutura condicional, Portugol                                                                                                          \\ \hline
36  & \href{http://www.dei.estt.ipt.pt/portugol/node/27}{Repetição \textgreater{}\textgreater Enquanto                                       } & Enquanto, Estrutura de repeticao, Portugol                                                                                                        \\ \hline
37  & \href{http://www.dei.estt.ipt.pt/portugol/node/28}{Repetição \textgreater{}\textgreater Para                                           } & Para, Estrutura de repeticao, Portugol                                                                                                            \\ \hline
38  & \href{http://www.dei.estt.ipt.pt/portugol/node/29}{Repetição \textgreater{}\textgreater Repete                                         } & Repita, Estrutura de repeticao, Portugol                                                                                                          \\ \hline
39  & \href{http://www.dei.estt.ipt.pt/portugol/node/30}{Repetição \textgreater{}\textgreater Faz                                            } & Repita, Estrutura de repeticao, Portugol                                                                                                          \\ \hline
40  & \href{http://www.dei.estt.ipt.pt/portugol/node/6}{Linguagem Algorítmica                                                               } & Sintaxe, Semantica, Portugol                                                                                                                      \\ \hline
41  & \href{http://www.dei.estt.ipt.pt/portugol/node/8}{Tipos de dados \textgreater{}\textgreater Básicos                                   } & Tipos de dados, Tipo inteiro, Tipo real, Tipo logico, Tipo caracter, Tipo texto                                                                   \\ \hline
42  & \href{http://www.dei.estt.ipt.pt/portugol/node/9}{Tipos de dados \textgreater{}\textgreater Variáveis                                 } & Variaveis, Portugol                                                                                                                               \\ \hline
43  & \href{http://www.devmedia.com.br/fluxogramas-diagrama-de-blocos-e-de-chapin-no-desenvolvimento-de-algoritmos/28550}{Fluxogramas, diagrama de blocos e de Chapin no desenvolvimento de algoritmos        } & Tipos de algoritmo, Fluxograma                                                                                                                    \\ \hline
44  & \href{http://www.dicasdeprogramacao.com.br/estrutura-de-selecao-multipla-escolha-caso/}{Estrutura de seleção multipla ESCOLHA-CASO                                          } & Estrutura condicional, Decisao encadeada, Escolha                                                                                                 \\ \hline
45  & \href{http://www.dicasdeprogramacao.com.br/o-que-e-algoritmo/}{O que é Algoritmo?                                                                  } & Algoritmo, Problema                                                                                                                               \\ \hline
46  & \href{http://www.din.uem.br/~teclopes/FCaula5.pdf}{Algoritmos – Estruturas de Controle                                                 } & Estrutura condicional, Decisao simples, Decisao composta, Decisao encadeada, Se, Escolha                                                          \\ \hline
47  & \href{http://www.gazetadopovo.com.br/vida-e-cidadania/o-uso-cotidiano-do-algoritmo-4x3n9sw4bkhoam6fzqcp27mfi}{O uso cotidiano do algoritmo                                                        } & Aplicacao de algoritmo                                                                                                                            \\ \hline
48  & \href{http://www.ic.unicamp.br/~sheila/mc102/02\_Entrada\%20Saida\%20e\%20Operadores.pdf}{Comandos de entrada e saída                                                         } & Linguagem C, Entrada, Saida, Atribuicao, Operadores                                                                                               \\ \hline
49  & \href{http://www.ic.unicamp.br/~wainer/cursos/2s2011/Cap05-EstruturasCondicionais-texto.pdf}{Estruturas Condicionais                                                             } & Linguagem C, Estrutura condicional, Operadores relacionais                                                                                        \\ \hline
50  & \href{http://www.inf.pucrs.br/~benso/progi/guia.htm}{Programação para Engenharia I                                                       } & Linguagem C, Tipos de dados, Tipo texto                                                                                                           \\ \hline
51  & \href{http://www.inf.pucrs.br/~pinho/LaproI/ComandosDeRepeticao/Repeticao.html}{Comandos de decisão Comandos de seleção                                             } & Linguagem C, Estrutura condicional, Estrutura de repeticao                                                                                        \\ \hline
52  & \href{http://www.inf.pucrs.br/~pinho/LaproI/IntroC/IntroC.htm}{Introdução à Linguagem C                                                            } & Variaveis, Tipos de dados, Tipo texto, Entrada, Saida, Tipo real, Operadores aritmeticos                                                          \\ \hline
53  & \href{http://www.inf.pucrs.br/~pinho/LaproI/Vetores/Vetores.htm}{Matrizes e Vetores                                                                  } & Linguagem C, Vetor, Matriz, Entrada, Saida, Atribuicao                                                                                            \\ \hline
54  & \href{http://www.inf.ufpr.br/cursos/ci067/Docs/NotasAula/notas-19\_Arrays\_Multidimensionais.html}{Arrays multidimensionais                                                            } & Vetor, Matriz                                                                                                                                     \\ \hline
55  & \href{http://www.inf.ufpr.br/cursos/ci067/Docs/NotasAula/notas-6\_Operadores\_Logicos.html}{Operadores Lógicos                                                                  } & Linguagem C, Operadores logicos                                                                                                                   \\ \hline
56  & \href{http://www.inf.ufsc.br/~bosco/ensino/ine5201/ApostilaVisuAlg20.pdf}{Manual do Visualg                                                                   } & Portugol, Variaveis, Tipos de dados, Sintaxe, Constantes, Atribuicao, Entrada, Saida, Operadores, Estrutura condicional, Estrutura de repeticao   \\ \hline
57  & \href{http://www.ipb.pt/~cmca/algor1.pdf}{Noção e Representação de Algoritmos                                                 } & Algoritmo, Tipos de algoritmo, Problema, Narrativa, Fluxograma                                                                                    \\ \hline
58  & \href{http://www.omundodaprogramacao.com/representacao-de-algoritmos/}{Representação de Algoritmos                                                         } & Algoritmo, Tipos de algoritmo, Narrativa, Fluxograma, Diagrama de Chapin, Pseudocodigo                                                            \\ \hline
59  & \href{http://www.rafaeltoledo.net/estruturas-de-selecao/}{Estrutura de seleção                                                                } & Pseudocodigo, Estrutura condicional                                                                                                               \\ \hline
60  & \href{http://www.rafaeltoledo.net/introducao-a-linguagem-c-parte-i/}{Introdução à linguagem C - Parte 1                                                  } & Linguagem C, Tipos de dados, Operadores relacionais, Tipo texto                                                                                   \\ \hline
61  & \href{http://www.rafaeltoledo.net/introducao-a-logica-de-programacao/}{Introdução à lógica de programação                                                  } & Algoritmo, Logica de programacao, Portugol, Pseudolinguagem                                                                                       \\ \hline
62  & \href{http://www.rafaeltoledo.net/lacos-de-repeticao/}{Laços de repetição                                                                  } & Pseudocodigo, Estrutura de repeticao                                                                                                              \\ \hline
63  & \href{http://www.rafaeltoledo.net/vetores-e-matrizes/}{Vetores e Matrizes                                                                  } & Pseudocodigo, Vetor, Matriz                                                                                                                       \\ \hline
64  & \href{http://www.tecmundo.com.br/programacao/2082-o-que-e-algoritmo-.htm}{O que é algoritmo?                                                                  } & Algoritmo, Tipos de algoritmo                                                                                                                     \\ \hline
65  & \href{http://www.tiexpert.net/programacao/algoritmo/o-que-e-um-algoritmo.php}{O que é um algoritmo?                                                               } & Algoritmo                                                                                                                                         \\ \hline
66  & \href{http://www.univasf.edu.br/~andreza.leite/aulas/AP/VetoresMatrizes.pdf}{Vetores e Matrizes                                                                  } & Vetor, Matriz                                                                                                                                     \\ \hline
67  & \href{http://www.univasf.edu.br/~jose.valentim/aula1.pdf}{Algoritmo e Programação                                                             } & Algoritmo, Tipos de algoritmo, Fluxograma, Narrativa, Portugol, Variaveis, Tipos de dados                                                         \\ \hline
68  & \href{http://www.univasf.edu.br/~ricardo.aramos/disciplinas/AlgProgAgr\_2011\_1/cap03AnaEmilia.pdf}{Estruturas de Controle                                                              } & Estrutura sequencial, Estrutura condicional, Decisao simples, Decisao composta, Decisao encadeada, Estrutura de repeticao, Para, Repita, Enquanto \\ \hline
69  & \href{https://becode.com.br/principais-linguagens-de-programacao/}{As 15 principais linguagens de programação do mundo!                                } & Linguagem de programacao                                                                                                                          \\ \hline
70  & \href{https://dicasdeprogramacao.com.br/o-que-sao-vetores-e-matrizes-arrays/}{O que são Vetores e Matrizes (arrays)                                               } & Vetor, Matriz, Portugol, Estrutura de repeticao                                                                                                   \\ \hline
71  & \href{https://dicasdeprogramacao.com.br/operadores-relacionais/}{Conheça os Operadores Relacionais!                                                  } & Portugol, Operadores relacionais                                                                                                                  \\ \hline
72  & \href{https://lucianopascal.wordpress.com/2010/04/02/aprendendo-a-interpretar-exercicios-de-algoritmos/}{Aprendendo a interpretar exercícios de algoritmos                                   } & Algoritmo, Problema, Entrada, Processamento, Saida                                                                                                \\ \hline
73  & \href{https://msdn.microsoft.com/pt-br/library/474dd6e2.aspx}{Operadores de atribuição C                                                          } & Linguagem C, Atribuicao                                                                                                                           \\ \hline
74  & \href{https://msdn.microsoft.com/pt-br/library/6swh93dx.aspx}{Operadores relacionais e de igualdade C                                             } & Linguagem C, Operadores relacionais                                                                                                               \\ \hline
75  & \href{https://msdn.microsoft.com/pt-br/library/exefbdtf.aspx}{Operador de expressão condicional                                                   } & Linguagem C, Operadores logicos, Operadores relacionais, Estrutura condicional                                                                    \\ \hline
76  & \href{https://msdn.microsoft.com/pt-br/library/z68fx2f1.aspx}{Operadores lógicos C                                                                } & Linguagem C, Operadores logicos                                                                                                                   \\ \hline
77  & \href{https://pt.wikipedia.org/wiki/Algoritmo}{Algoritmo                                                                           } & Algoritmo, Tipos de algoritmo, Programa de computador, Interpretador, Compilador                                                                  \\ \hline
78  & \href{https://pt.wikipedia.org/wiki/Estrutura\_de\_controle}{Estrutura de controle                                                               } & Estrutura sequencial, Estrutura condicional, Estrutura de repeticao                                                                               \\ \hline
79  & \href{https://pt.wikipedia.org/wiki/Estrutura\_de\_repeti\%C3\%A7\%C3\%A3o}{Estrutura de repetição                                                              } & Estrutura de repeticao, Enquanto, Para, Repita                                                                                                    \\ \hline
80  & \href{https://pt.wikipedia.org/wiki/Operadores\_em\_C\_e\_C\%2B\%2B}{Operadores em C e C++                                                               } & Linguagem C, Operadores                                                                                                                           \\ \hline
81  & \href{https://sites.google.com/site/itabits/treinamento/introducao-a-programacao-em-c/comandos-de-repeticao}{Comandos de Repetição (Laços ou Loops)                                              } & Linguagem C, Estrutura de repeticao, Enquanto, Para, Repita                                                                                       \\ \hline
82  & \href{https://www.dca.ufrn.br/~ivan/DCA0800/tiposDados.pdf}{Outros conceitos sobre logica de programação: Tipos de dados, variaveis e expressões} & Tipos de dados, Variaveis, Operacoes, Operacoes aritmeticas, Operacoes logicas                                                                    \\ \hline
83  & \href{https://www.dcc.fc.up.pt/~nam/aulas/9900/ic/slides/sliic9918/}{Linguagens de Programação                                                           } & Linguagem de programacao, Linguagem de maquina, Compilador                                                                                        \\ \hline
84  & \href{https://www.devmedia.com.br/estrutura-de-decisao-em-c-c/24031}{Estrutura de Decisão em C/C++                                                       } & Linguagem C, Estrutura condicional, Se, Escolha                                                                                                   \\ \hline
85  & \href{https://www.ime.usp.br/~hitoshi/introducao/03-Fundamentos.pdf}{Fundamentos                                                                         } & Linguagem C, Entrada, Saida, Atribuicao, Operadores, Variaveis, Precedencia de operadores                                                         \\ \hline
86  & \href{https://www.ime.usp.br/~pf/algoritmos/aulas/aloca.html}{Alocação dinâmica de memória                                                        } & Vetor, Matriz, Linguagem C                                                                                                                        \\ \hline
87  & \href{https://www.ime.usp.br/~pf/algoritmos/aulas/string.html}{Strings                                                                             } & Tipo texto, Entrada, Saida, Constantes                                                                                                            \\ \hline
88  & \href{https://www.inf.pucrs.br/~pinho/LaproI/Vetores/Vetores.htm}{Programação C/C++ - Matrizes e Vetores                                              } & Linguagem C, Vetor, Matriz                                                                                                                        \\ \hline
89  & \href{https://www.learnconline.com/2010/03/if-else-statement-c-programming-language.html}{The if-else Statement in c programming language                                     } & Se, Linguagem C                                                                                                                                   \\ \hline
90  & \href{https://www.learnconline.com/2010/03/while-loop-statement-in-c.html}{While Statement in c programming language                                           } & Enquanto, Linguagem C                                                                                                                             \\ \hline
91  & \href{https://www.programiz.com/c-programming/c-multi-dimensional-arrays}{C Programming Multidimensional Arrays                                               } & Linguagem C, Vetor, Matriz                                                                                                                        \\ \hline
92  & \href{https://www.tutorialspoint.com/cprogramming/c\_arrays.htm}{C- arrays                                                                           } & Linguagem C, Vetor                                                                                                                                \\ \hline
93  & \href{https://www.tutorialspoint.com/cprogramming/c\_multi\_dimensional\_arrays.htm}{Multi-dimensional Arrays in C                                                       } & Linguagem C, Matriz                                                                                                                               \\ \hline
94  & \href{https://www.tutorialspoint.com/cprogramming/c\_strings.htm}{C - Strings                                                                         } & Linguagem C, Tipo texto                                                                                                                           \\ \hline
95  & \href{https://www.youtube.com/watch?v=7oA8SBAOOAo}{Programar em C - Revisão Vetores/Matrizes                                           } & Vetor, Matriz, Problema                                                                                                                           \\ \hline
96  & \href{https://www.youtube.com/watch?v=7ph98Ih\_ckc}{Lógica de programação - Aula 03 - Legibilidade do código                            } & Algoritmo, Comentario, Blocos                                                                                                                     \\ \hline
97  & \href{https://www.youtube.com/watch?v=Ds1n6aHchRU}{Lógica de programação - Aula 01 - Introdução                                        } & Logica de programacao, Aplicacao de algoritmo                                                                                                     \\ \hline
98  & \href{https://www.youtube.com/watch?v=dZq7l9Oj-\_c}{Portugol - VisuALG - Aula 01 (Princípios Básicos)                                   } & Tipos de dados, Tipo inteiro, Tipo real, Tipo logico, Tipo caracter, Entrada, Saida, Atribuicao, Operadores aritmeticos                           \\ \hline
99  & \href{https://www.youtube.com/watch?v=ED7QtgXDShY}{Libraries                                                                           } & Linguagem C                                                                                                                                       \\ \hline
100 & \href{https://www.youtube.com/watch?v=g0iIVeeQo1M}{Lógica de programação - Aula 05 - Expressões, operadores e comandos                 } & Entrada, Saida, Operadores, Operadores relacionais, Operadores lógicos, Operadores aritmeticos, Atribuicao                                        \\ \hline
101 & \href{https://www.youtube.com/watch?v=JLlTo3SwxJE}{Lógica de programação - Aula 02 - Tipos de algoritmo                                } & Tipos de algoritmo, Fluxograma, Pseudocodigo, Narrativa                                                                                           \\ \hline
102 & \href{https://www.youtube.com/watch?v=l26oaHV7D40}{Programming Basics: Statements \& Functions: Crash Course Computer Science          } & Semantica, Sintaxe, Blocos                                                                                                                        \\ \hline
103 & \href{https://www.youtube.com/watch?v=Lelg\_sOYSm0}{Portugol Studio - Teste de Mesa                                                     } & Algoritmo, Processamento, Logica de programacao                                                                                                   \\ \hline
104 & \href{https://www.youtube.com/watch?v=mHW1Hsqlp6A}{Por que todos deveriam aprender a programar?                                        } & Aplicacao de algoritmo                                                                                                                            \\ \hline
105 & \href{https://www.youtube.com/watch?v=UuTmEcy5rV0}{Vetores e Matrizes                                                                  } & Vetor, Matriz, Fluxograma, Pseudocodigo                                                                                                           \\ \hline
106 & \href{https://www.youtube.com/watch?v=vgu8x\_Ivjd0}{Lógica de Programação - Estruturas de Repetição (Enquanto, Para, FacaEnquanto)      } & Logica de programacao, Estrutura de repeticao, Para, Enquanto, Repita                                                                             \\ \hline
107 & \href{https://www.youtube.com/watch?v=vp4jgXA\_BB0}{Lógica de programação - Aula 04 - Variáveis e constantes                            } & Variaveis, Constantes, Tipos de dados, Tipo inteiro, Tipo real, Tipo texto, Tipo logico, Vetor, Matriz                                            \\ \hline
108 & \href{https://www.zemoleza.com.br/trabalho-academico/exatas/informatica/as-principais-bibliotecas-em-linguagem-c/}{As principais bibliotecas em linguagem C                                            } & Linguagem C \\ \hline
\caption{Palavras-chave associadas com os Links de Apoio}
\label{tab:palavras-chave-links-de-apoio}
\end{longtable}
