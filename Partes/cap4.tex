\chapter{SR Sensível ao Tempo}\label{chapter:proposta}

Neste capítulo é apresentado o SR Sensível ao Tempo para ambientes educacionais proposto nesse trabalho. O algoritmo proposto
considera o decaimento nos interesses do usuário com o passar do tempo (categoria do \textit{Decay} - Seção \ref{section:decay}). Nesse capítulo é apresentado
o algoritmo proposto, uma análise das suas vantagens e cenários que ilustram o uso desse algoritmo.

\section{Algoritmo Proposto}\label{section:algoritmo-proposto}

O algoritmo de recomendação proposto nesse trabalho combina a Abordagem Baseada em Conteúdo tradicional com o uso do
contexto temporal através da categoria \textit{Decay} (ver seção \ref{section:decay}).

O cálculo da relevância de um determinado item $i$ para um usuário $u$ no algoritmo proposto está formalizada na Equação
\ref{eq:relevancia-proposta}.

\begin{equation}
  F(u,i) = S(u,i) \cdot R(I_{u,i}) + A(i)
  \label{eq:relevancia-proposta}
\end{equation}

Onde: $F$ é a função que calcula a relevância de um item $i$ para um usuário $u$; $S$ é a função de similaridade entre
o perfil do usuário $u$ (representado através de conjunto de palavras-chave dos itens acessados) e o item $i$
(representado pelo conjunto de palavras-chave que o caracterizam); $R$ é o maior valor de recência dos itens do conjunto
$I_{u,i}$ (itens acessados pelo usuário $u$ e com alguma similaridade com o item $i$); $A$ é uma função que retorna $1$
se o item $i$ nunca foi acessado pelo usuário u e $0$ se o item já foi acessado.

A função $S$ de similaridade entre o usuário $u$ e o item $i$ é calculada utilizando a fórmula do cosseno (ver Seção
\ref{subsection:baseada-em-conteudo}). A função de recência $R$, responsável pelo \textit{Decay}, estéa definida na Equação
\ref{eq:recencia-proposta}.

\begin{equation}
  R(I_{u,i}) = \max_{\{j \in I_{u,i}\}}{\frac{x_j}{\left| I_u \right|}}
  \label{eq:recencia-proposta}
\end{equation}

Onde: $x_j$ é a posição do item j na lista de itens acessados pelo usuário $u$ ordenada de forma crescente pelo
\textit{timestamp} do acesso; e $\left| I_u \right|$ é a quantidade de itens acessados pelo usuário $u$. Essa fórmula considera
que mais de um item acessado pelo usuário pode ser similar ao item $i$, por isso a fórmula retorna a maior recência de
todos os itens presentes no perfil do usuário que são similares a $i$. A seguir temos um exemplo do uso da fórmula da
recência de um item, que pode ser estendida para o uso com vários itens e aplicada a função $max$ como descrito na
fórmula acima.

Considerando um usuário $u$ que acessou três itens diferentes nos seguintes \textit{timestamps} (em \textit{epoch}):
Item A acessado $1503670382$; Item B acessado em $1500027182$; Item C acessado em $1508051582$. Ao ordenar esses itens
pelo \textit{timestamp}, em ordem crescente, temos: (1) Item B, (2) Item A, (3) Item C. As equações \ref{eq:recencia-item-a},
\ref{eq:recencia-item-b} e \ref{eq:recencia-item-c} apresentam o resultado do cálculo de recência para cada um dos itens:

\begin{equation}
  R(A) = \frac{2}{3} = 0.\overline{6}
  \label{eq:recencia-item-a}
\end{equation}

\begin{equation}
  R(B) = \frac{1}{3} = 0.\overline{3}
  \label{eq:recencia-item-b}
\end{equation}

\begin{equation}
  R(C) = \frac{3}{3} = 1.0
  \label{eq:recencia-item-c}
\end{equation}

No Algoritmo \ref{alg:proposta-pseudocodigo} é possível observar o pseudo-código do algoritmo proposto. Nesse algoritmo
está representado em mais alto nível as etapas da recomendação. Primeiramente, são buscados no banco de dados a lista de itens
acessados pelo usuário e a lista de itens que podem ser recomendados. O cálculo da recência para cada item do perfil do usuário
é feito também na fase de inicialização, pois esse cálculo só precisa ser realizado uma vez. Depois disso, para cada item
candidato a ser recomendado para o usuário, é calculada a similaridade do perfil do usuário (composto pelas palavras dos itens acessados
por ele) com as palavras-chave do item candidato. Nessa etapa também é encontrado qual o item do perfil do usuário que é semelhante
ao item candidato e é mais recente, para utilizar o seu valor de recência no cálculo do score.

\begin{algorithm}\label{alg:proposta-pseudocodigo}
  \caption{Pseudo-código do Algoritmo de Recomendação Proposto}

  \SetKwInOut{Input}{input}
  \SetKwInOut{Output}{output}
  \SetKwProg{algoritmoRecomendacaoDecay}{algoritmoRecomendacaoDecay}{}{}

  \algoritmoRecomendacaoDecay{$(Usuário)$}{
    \Input{Usuário que irá receber a recomendação.}
    \Output{Lista com cinco itens recomendados para o usuário.}
    itens = buscarItensCandidatos()\;
    itensPerfilUsuario = buscarItensPerfil(Usuário)\;


    recênciaItensPerfil = calcularRecência(itensPerfilUsuario)\;
    scores = \{\}\;

    \ForEach{$item \in itens$}{
      similaridade, itensSimilares = calcularCosseno(itensPerfilUsuario, item)\;
      recência = calcularMaxRecência(itensSimilares, recênciaItensPerfil)\;
      scores[item] = similaridade * recência + item.acessado\;
    }

    \Return scores.max(5)\;
  }
\end{algorithm}

O cálculo do score para o item candidato é finalmente realizado utilizando a Equação \ref{eq:relevancia-proposta}. Os cinco
itens com o maior score são retornados pelo algoritmo para serem recomendados ao usuário.

No Algoritmo \ref{alg:calcular-recencia-pseudocodigo} é demonstrado o cálculo da recência para cada um dos itens do perfil do
usuário, que é utilizado pelo Algoritmo \ref{alg:proposta-pseudocodigo} descrito anteriormente. Nesse cálculo, os itens são
ordenados pela data de acesso inicialmente e após isso uma versão simplificada da Equação \ref{eq:recencia-proposta} é aplicada.
A função $max$ presente na Equação \ref{eq:recencia-proposta} não é aplicada nesse momento, sendo cálculada apenas a recência
para cada item individualmente. No método descrito no Algoritmo \ref{{alg:calcular-recencia-pseudocodigo}} é que será aplicado
o $max$ nas recências, para encontrar a maior recência entre os itens similares ao item candidato à recomendação.

\begin{algorithm}
  \caption{Pseudo-código do cálculo da recência para cada item presente no perfil do usuário \label{alg:calcular-recencia-pseudocodigo}}

  \SetKwInOut{Input}{input}
  \SetKwInOut{Output}{output}
  \SetKwProg{calcularRecencia}{calcularRecência}{}{}

  \calcularRecencia{$(itensPerfilUsuario)$}{
    \Input{Lista de itens acessados pelo usuário.}
    \Output{Hash com a recência calculada para cada item no perfil do usuário.}
    recênciaItensPerfil = \{\}\;
    itensOrdenados = itensPerfilUsuario.ordenarPor(dataAcesso)\;
    tamanhoPerfilUsuario = itensPerfilUsuario.tamanho()\;

    \ForEach{$item \in itensPerfilUsuario$}{
      recênciaItensPerfil[item] = itensOrdernados.find(item).posição() / tamanhoPerfilUsuario\;
    }

    \Return recênciaItensPerfil\;
  }
\end{algorithm}

\begin{algorithm}\label{alg:max-recencia-pseudocodigo}
  \caption{Pseudo-código do cálculo da recência máxima entre os itens do perfil do usuário similares ao item candidato}

  \SetKwInOut{Input}{input}
  \SetKwInOut{Output}{output}
  \SetKwProg{calcularMaxRecencia}{calcularMaxRecência}{}{}

  \calcularMaxRecencia{$(itensSimilares, recênciaItensPerfil)$}{
    \Input{Lista de itens similares e hash com a recência calculada para cada item do perfil do usuário.}
    \Output{Valor máximo de recência entre itens similares.}

    recênciaItensSimilares = recênciaItensPerfil.find(itensSimilares)\;

    \Return recênciaItensSimilares.max(1)\;
  }
\end{algorithm}


\section{Discussão Sobre o Algoritmo Proposto}

Um ponto negativo da Filtragem Colaborativa é a necessidade de uma comunidade de usuários ativa, que nem sempre é
possível em um AVA onde as turmas muitas vezes são menores (entre 10 e 100 alunos). A Abordagem Baseada
em Conteúdo é considerada para o SR porque essa abordagem permite a recomendação em um sistema que não possui uma
comunidade ativa e pode suprir as necessidades desse domínio.

Os SRs Sensíveis ao Tempo tem uma vantagem em relação à outros SRs Sensíveis ao Contexto por a informação temporal ser
mais simples de capturar e manipular que outras informações contextuais, e.g., localização. Além disso, esses tipos de
algoritmos estão sendo explorados em outros domínios de aplicação, como pode ser visto nos 88 artigos analisados no
Mapeamento Sistemático realizado \cite{de2017time}, e demonstraram bons resultados. Por isso, esse trabalho busca
aplicar o contexto temporal no algoritmo de recomendação na área educacional (na qual foram encontrados apenas quatro
trabalhos) e avaliar os resultados.

A escolha do \textit{Decay} se justifica como forma de procurar minimizar uma das grandes desvantagens da abordagem Baseada em Conteúdo: a
Superespecialização. Na abordagem Baseada em Conteúdo as recomendações seriam geradas levando em conta todos os itens
acessados pelo usuário igualitariamente. Enquanto ao aplicar o \textit{Decay}, os itens acessados mais recentemente possuem um
grau de importância maior para o algoritmo do que itens acessados anteriormente. Dessa forma, mesmo que o usuário
tenha acessado muitos materiais sobre determinado assunto, ao começar a acessar materiais sobre outro assunto o algoritmo de
recomendação consegue rapidamente se adaptar e gerar recomendações sobre esse novo conteúdo.

Um cenário de uso do \textit{Decay} é apresentado a seguir. \textit{O aluno Pedro está matriculado em uma disciplina de Estrutura de Dados
que possui quatro tópicos, sendo eles Pilhas, Filas, Listas e Árvores. Nessa disciplina, para cada um dos conteúdos é
aplicada uma prova para avaliar os conhecimentos dos alunos. Até o momento da primeira prova sobre o conteúdo de Pilhas,
Pedro acessou apenas materiais relacionados a Pilhas. O algoritmo de recomendação utilizando a abordagem Baseada em
Conteúdo recomenda para Pedro apenas materiais relacionados a Pilhas. Após a primeira prova, Pedro começa a acessar
materiais relacionados a Filas, o segundo tópico da disciplina. Em uma abordagem Baseada em Conteúdo tradicional as
recomendações continuariam sendo sobre o conteúdo de Pilhas por um bom tempo, pois o perfil de Pedro seria em grande
parte composto por materiais acessados sobre esse assunto. Porém, com o uso do \textbf{Decay}, no momento em que Pedro começar a
acessar materiais sobre Filas o algoritmo de recomendação dará um peso maior para esses materiais (sem ignorar os itens
acessados anteriormente) e em pouco tempo Pedro já estará recebendo recomendações sobre o novo tópico estudado. Da mesma
forma, se Pedro voltar a acessar conteúdos anteriores para relembrar algum conceito, o SR também perceberá isso e
recomendará itens relacionados ao primeiro conteúdo novamente.}

O algoritmo proposto nesse trabalho considera o decaimento no peso dos itens do perfil do usuário em função da posição
do item na sequência de materiais acessados, como mostrado na Seção \ref{section:algoritmo-proposto}, e não na quantidade de tempo passada em
segundos como feito na maioria dos trabalhos relacionados mostrados no Capítulo \ref{chapter:trabalhos-relacionados}. Isso é uma vantagem, pois no domínio
educacional o passar do tempo não é tão relevante quanto em outros domínios, como na recomendação de \textit{Web Services} por
exemplo.

Os alunos em um AVA podem ter ritmos de estudo diferentes e, por isso, é assumido nesse trabalho que faz
mais sentido analisar quantos itens foram acessados desde do acesso de determinado material do que o tempo passado desde
a interação. Dessa forma, esse algoritmo não considera se o usuário acessou todos os itens do seu perfil em um
único dia, ou se ele acessou metade do materiais na primeira semana do curso e a outra metade na última semana ou se ele
acessou alguns materiais todos os dias durante o curso.

Além disso, no algoritmo proposto nesse trabalho não é definido um fator de decaimento único para os todos os usuários como nos trabalhos
relacionados apresentados no Capítulo \ref{chapter:trabalhos-relacionados}. Isto porque é necessário considerar
que cada aluno tem um estilo de aprendizagem diferente e o decaimento para um aluno poderia ser diferente do decaimento
para outros alunos. E não foram encontrados nos trabalhos relacionados que mostrem um modelo para calcular o fator de decaimento de forma
personalizada para cada usuário. Em geral, os autores utilizam o fator de decaimento escolhido de forma empírica e
aplicam o mesmo fator para todos os usuários.

\section{Considerações sobre o capítulo}

Nesse capítulo é apresentada a proposta desse trabalho. O algoritmo proposto utiliza a abordagem Baseada em Conteúdo em
conjunto com o uso do tempo através da categoria \textit{Decay}, ou seja, é dado um peso menor na recomendação para os
itens acessados a mais tempo pelo usuário e um peso maior para os itens mais recentes. No Sistema de Recomendação (SR)
proposto é combinada a similaridade entre o perfil do usuário e os itens, a recência dos itens acessados pelo usuário e
se os links recomendados já foram ou não acessados. O algoritmo proposto será avaliado utilizando o experimento descrito
no Capítulo \ref{chapter:experimento}


