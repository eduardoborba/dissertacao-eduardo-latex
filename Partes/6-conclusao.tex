\chapter{Considerações Finais}\label{chapter:conclusoes}

Sistemas de Recomendação (SR) são ferramentas de software que sugerem itens para os usuários de forma automatizada e personalizada,
sem a necessidade do usuário formular uma consulta para encontrar os itens do seu interesse. Esses sistemas são
explorados em Ambientes Virtuais de Aprendizagem (AVA) com o objetivo de reduzir alguns problemas existentes nesses ambientes
quando a quantidade de materiais disponíveis é grande, tais como: sobrecarga cognitiva, dificuldade de encontrar os materiais
do seu interesse e muitos materiais nunca serem utilizados.

Pesquisadores da área argumentam que os algoritmos de SRs tradicionais não são suficientes para os AVAs \cite{verbert2012context, drachsler2015panorama},
sendo necessário um nível maior de personalização a situação do usuário, como considerar as informações contextuais do aluno.
Com o propósito de investigar sobre esse tema, em \citeonline{de2017time} foi realizado um mapeamento sistemático da
literatura com o objetivo de identificar como os SRs Sensíveis ao Contexto Temporal (também chamados de SR Sensíveis ao
Tempo) são utilizados. Nesse estudo foram considerados todos os domínios de aplicação e não apenas o domínio educacional,
devido a quantidade reduzida de trabalhos nesse tema aplicados à essa área.

Foi observado que, dos 88 artigos que utilizam esse tipo de SR, apenas quatro são aplicados no domínio educacional e esses
trabalhos carecem em avaliações em ambientes reais de uso ou que utilizem bases de dados educacionais. Analisando esses
88 artigos, também foi possível categorizar os SRs propostos nesses trabalhos pela forma que eles utilizam o tempo para
a recomendação. As sete categorias criadas como resultado do mapeamento são apresentadas na Seção \ref{section:sr-sensivel-tempo}.

O objetivo desse trabalho é analisar se a aplicação do contexto temporal, através do decaimento, em um sistema de recomendação
voltado a AVAs influencia o desempenho do algoritmo de recomendação e a percepção do aluno sobre a qualidade das recomendações
recebidas. Dentre as categorias de SR Sensíveis ao Tempo presentes na Seção \ref{section:sr-sensivel-tempo}, a proposta
desse trabalho se encaixa no \textit{Decay}.

O algoritmo proposto no Capítulo \ref{chapter:proposta} combina a (1) similaridade do perfil do usuário (representado
pelos materiais acessados pelo usuário) com os itens disponíveis para a recomendação com a (2) recência dos materiais
acessados pelo usuários, além da (3) informação se aquele item disponível para a recomendação já foi acessado ou não. A
proposta leva em conta que o ritmo de estudo dos alunos pode ser diferente, portanto a recência é considerada em relação
a sequência de itens acessados e não ao tempo absoluto (em segundos) desde o acesso. Dessa forma, para cada aluno o
decaimento acaba sendo personalizado ao seu ritmo de estudo. Também é considerado que itens já acessados podem ser
recomendados novamente, porém esses itens tem um probabilidade menor de ser recomendados do que itens ainda não acessados.

Para alcançar o objetivo do trabalho e responder a questão de pesquisa, um experimento controlado foi realizado em um ambiente
real de uso comparando a proposta do trabalho com a abordagem Baseada em Conteúdo Tradicional. O experimento aconteceu
dentro do ambiente \adaptwebspace durante um Minicurso de Algoritmos e Linguagem de Programação no período de 16 de Abril de
2018 à 10 de Maio de 2018. Durante a execução do experimento, dados de acesso e avaliação das recomendações pelos alunos
foram capturadas para analisar o desempenho do algoritmo proposto. Ao final do experimento os alunos responderam ao questionário presente no Apêndice
\ref{ape:questionario-de-satisfacao}, com questões selecionadas do questionário proposto por \citeonline{pu2011user} para
avaliar a percepção do usuário sobre a qualidade das recomendações recebidas.

Os resultados do experimento presentes na Seção \ref{:section:analise-experimento} e mostraram que a proposta desse
trabalho Com Decaimento teve resultados estatísticamente melhores que a Abordagem Baseada em Conteúdo Tradicional em
relação à Cobertura e ao F-measure do algoritmo (métricas do desempenho do algoritmo) e em relação à uma das perguntas
do questionário aplicado (que visava medir a percepção do usuário). Em relação às outras métricas de desempenho e às
outras questões do questionário, não foi encontrado diferença significativa entre o resultado para os dois algoritmos.
Mesmo assim, foi possível aceitar as duas hipóteses alternativas definidas que afirmar que existe diferença significativa
no desempenho do algoritmo e na percepção do usuário entre os dois algoritmos comparados, com resultado favorável
a proposta desse trabalho. Com esses resultados, pode-se responder à questão de pesquisa desse trabalho afirmando que considerar o Decaimento do
interesse do aluno em relação aos itens acessados anteriormente influencia positivamente tanto o desempenho do algoritmo
quanto a percepção do usuário sobre isso.

Como outros resultados desse trabalho tem-se as seguintes publicações:

\begin{itemize}
\item BORBA, E. J.; Gasparini, I.; LICHTNOW, D. Time-Aware Recommender Systems: A Systematic Mapping. International Conference on Human-Computer Interaction (HCI), Vancouver, Part II, LNCS 10272, v. II, p. 464-479, 2017.
\item BORBA, E. J.; GASPARINI, I.; LICHTNOW, D. The Use of Time Dimension in Recommender Systems for Learning. Proceedings of the 19th International Conference on Enterprise Information Systems (ICEIS), Porto (Portugal) 2017. v. 2. p. 600-609.
\item BORBA, E. J.; GASPARINI, I.; LICHTNOW, D. Sistema de Recomendação Sensível ao Tempo em Ambientes Educacionais. IV Workshop de Teses e Dissertações em IHC (WTD-IHC), Joinville, 2017.
\item BORBA, E. J.; GASPARINI, I.; LICHTNOW, D. Describing Scenarios and Architectures for Time-Aware Recommender Systems for Learning. In: Enterprise Information Systems, Springer International Publishing AG, part of Springer Nature, no. 17, 2018. no prelo.
\end{itemize}

Como trabalhos futuros, propõe-se o estudo mais aprofundado com relação à um limiar mínimo de similaridade entre os itens
acessados pelo alunos e os itens candidatos a recomendação para que a recomendação efetivamente aconteça. Ao conseguir
definir esse limiar, através de experimentos em um AVA, será possível garantir que apenas recomendações realmente relevantes
para o aluno sejam recomendadas para ele e que itens que possam desmotivar ou tirar a confiança do aluno no SR não
sejam recomendados.

Outro aspecto a se considerar é estudar a influência de um fator de decaimento no algoritmo de recomendação. O fator de
decaimento é uma constante, geralmente no intervalo de 0 a 1, que foi evitado nesse trabalho através do algoritmo proposto
porém muito utilizado nos trabalhos relacionados presentes no Capítulo \ref{chapter:trabalhos-relacionados}. Esse
fator de decaimento pode ser personalizado para cada aluno ou igual para todos e vai influenciar a velocidade com que o
algoritmo considera que o interesse do aluno por um determinado item acessado dimunui.

Aplicar do algoritmo proposto nesse trabalho em outros domínios de aplicação e analise se o resultado alcançado é o mesmo,
utilizando-se de hipóteses e análises similares às feitas aqui, é também possível como trabalhos futuros para comparar
com o resultado alcançado na área educacional. Além disso, é possível propor outros algoritmos utilizando outra categoria
de SRs Sensíveis ao Tempo e comparar com o algoritmo Com Decaimento para ver qual deles é mais adequado para a área educacional.
