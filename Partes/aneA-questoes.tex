\chapter{60 Questões do Framework de Avaliação de Sistemas de Recomendação ResQue}\label{ane:questoes-framework}

\section{Quality of Recommended Items}

\subsection{Accuracy}

\begin{itemize}
\item The items recommended to me matched my interests.*
\item The recommender gave me good suggestions.
\item I am not interested in the items recommended to me (reverse scale).
\end{itemize}

\subsection{Relative Accuracy}

\begin{itemize}
\item The recommendation I received better fits my interests than what I may receive from a friend.
\item A recommendation from my friends better suits my interests than the recommendation from this system (reverse scale).
\end{itemize}

\subsection{Familiarity}

\begin{itemize}
\item Some of the recommended items are familiar to me.
\item I am not familiar with the items that were recommended to me (reverse scale).
\end{itemize}

\subsection{Attractiveness}

\begin{itemize}
\item The items recommended to me are attractive.
\end{itemize}

\subsection{Enjoyability}

\begin{itemize}
\item I enjoyed the items recommended to me.
\end{itemize}

\subsection{Novelty}

\begin{itemize}
\item The items recommended to me are novel and interesting.*
\item The recommender system is educational.
\item The recommender system helps me discover new products.
\item I could not find new items through the recommender (reverse scale).
\end{itemize}

\subsection{Diversity}

\begin{itemize}
\item The items recommended to me are diverse.*
\item The items recommended to me are similar to each other (reverse scale).*
\end{itemize}

\subsection{Context Compatibility}

\begin{itemize}
\item I was only provided with general recommendations.
\item The items recommended to me took my personal context requirements into consideration.
\item The recommendations are timely.
\end{itemize}

\section{Interaction Adequacy}

\begin{itemize}
\item The recommender provides an adequate way for me to express my preferences.
\item The recommender provides an adequate way for me to revise my preferences.
\item The recommender explains why the products are recommended to me.*
\end{itemize}

\section{Interface Adequacy}

\begin{itemize}
\item The recommender’s interface provides sufficient information.
\item The information provided for the recommended items is sufficient for me.
\item The labels of the recommender interface are clear and adequate.
\item The layout of the recommender interface is attractive and adequate.*
\end{itemize}

\section{Perceived Ease of Use}

\subsection{Ease of Initial Learning}

\begin{itemize}
\item I became familiar with the recommender system very quickly.
\item I easily found the recommended items.
\item Looking for a recommended item required too much effort (reverse scale).
\end{itemize}

\subsection{Ease of Preference Elicitation}

\begin{itemize}
\item I found it easy to tell the system about my preferences.
\item It is easy to learn to tell the system what I like.
\item It required too much effort to tell the system what I like (reversed scale).
\end{itemize}

\subsection{Ease of Preference Revision}

\begin{itemize}
\item I found it easy to make the system recommend different things to me.
\item It is easy to train the system to update my preferences.
\item I found it easy to alter the outcome of the recommended items due to my preference changes.
\item It is easy for me to inform the system if I dislike/like the recommended item.
\item It is easy for me to get a new set of recommendations.
\end{itemize}

\subsection{Ease of Decision Making}

\begin{itemize}
\item Using the recommender to find what I like is easy.
\item I was able to take advantage of the recommender very quickly.
\item I quickly became productive with the recommender.
\item Finding an item to buy with the help of the recommender is easy.*
\item Finding an item to buy, even with the help of the recommender, consumes too much time.
\end{itemize}

\section{Perceived Usefulness}

\begin{itemize}
\item The recommended items effectively helped me find the ideal product.*
\item The recommended items influence my selection of products.
\item I feel supported to find what I like with the help of the recommender.*
\item I feel supported in selecting the items to buy with the help of the recommender.
\end{itemize}

\section{Control/Transparency}

\begin{itemize}
\item I feel in control of telling the recommender what I want.
\item I don’t feel in control of telling the system what I want.
\item I don’t feel in control of specifying and changing my preferences (reverse scale).
\item I understood why the items were recommended to me.
\item The system helps me understand why the items were recommended to me.
\item The system seems to control my decision process rather than me (reverse scale).
\end{itemize}

\section{Attitudes}

\begin{itemize}
\item Overall, I am satisfied with the recommender.*
\item I am convinced of the products recommended to me.*
\item I am confident I will like the items recommended to me. *
\item The recommender made me more confident about my selection/decision.
\item The recommended items made me confused about my choice (reverse scale).
\item The recommender can be trusted.
\end{itemize}

\section{Behavioral Intentions}

\subsection{Intention to Use the System}

\begin{itemize}
\item If a recommender such as this exists, I will use it to find products to buy.
\end{itemize}

\subsection{Continuance and Frequency}

\begin{itemize}
\item I will use this recommender again.*
\item I will use this type of recommender frequently.
\item I prefer to use this type of recommender in the future.
\end{itemize}

\subsection{Recommendation to Friends}

\begin{itemize}
\item I will tell my friends about this recommender.*
\end{itemize}

\subsection{Purchase Intention}

\begin{itemize}
\item I would buy the items recommended, given the opportunity.*
\end{itemize}
