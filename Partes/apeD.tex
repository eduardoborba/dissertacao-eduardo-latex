\chapter{Questionário de Satisfação - Questões selecionadas traduzidas}\label{ape:questionario-de-satisfacao}

\section{QUALITY OF RECOMMENDED ITEMS}
\subsection{Accuracy}
\begin{itemize}
\item Os itens recomendados corresponderam com os meus interesses.
\end{itemize}
\subsection{Diversity}
\begin{itemize}
\item Os itens recomendados para mim são diversificados (o sistema se preocupa em trazer itens diferentes a cada recomendação).
\end{itemize}
\subsection{Context Compatibility}
\begin{itemize}
\item Os itens recomendados corresponderam aos  interesses e necessidades que eu tinha no momento.
\item As recomendações são feitas no momento adequado.
\end{itemize}
\section{INTERACTION ADEQUACY}
\begin{itemize}
\item O sistema de recomendação explica porque os links são recomendados para mim.
\end{itemize}
\section{INTERFACE ADEQUACY}
\begin{itemize}
\item A informação apresentada na interface para os itens recomendados é suficiente para mim.
\item O layout do sistema de recomendação é atrativo e adequado.
\end{itemize}
\section{PERCEIVED EASE OF USE}
\subsection{Ease of Initial Learning}
\begin{itemize}
\item Eu encontrei facilmente o local onde os itens são recomendados.
\end{itemize}
\subsection{Ease of Preference Revision}
\begin{itemize}
\item Eu percebi que o sistema de recomendação aprendia sobre minhas necessidades/preferências conforme eu avançava na disciplina.
\end{itemize}
\subsection{Ease of Decision Making}
\begin{itemize}
\item É facil encontrar um item para estudar com a ajuda do sistema de recomendação.
\end{itemize}
\section{PERCEIVED USEFULNESS}
\begin{itemize}
\item Eu me senti apoiado para encontrar itens do meu interesse com a ajuda do sistema de recomendação.
\end{itemize}
\section{CONTROL/TRANSPARENCY}
\begin{itemize}
\item Eu entendi porque os itens foram recomendados para mim.
\end{itemize}
\section{ATTITUDES}
\begin{itemize}
\item No geral, estou satisfeito com o sistema de recomendação.
\end{itemize}
