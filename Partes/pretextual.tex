%!TEX root = ../Principal.tex
%Capa do Trabalho
% \imprimircapa

%Folha de Rosto
%* indica que tem ficha catalográfica
\imprimirfolhaderosto*

% ---
% Caso a Biblioteca da UDESC forneça, utilize o comando
% ---
\begin{fichacatalografica}
    \includepdf{./Partes/ficha-catalografica.pdf}
\end{fichacatalografica}

% % ---
% % Folha de Aprovação
% % ---
% % Exemplo de folha de aprovação antes da Banca. Após isso, incluia o pdf digitalizado com as assinaturas%
\includepdf{./Partes/folhadeaprovacao_final.pdf}
% \begin{folhadeaprovacao}
% 	\begin{center}
% 		{\ABNTEXchapterfont\bfseries\imprimirautor}
% 		\vspace{2em}

% 			\ABNTEXchapterfont\bfseries\imprimirtitulo

% 	\end{center}
% 		\vspace{1em}
% 		{\justify
%     Dissertação apresentada ao Programa de Pós-Graduação em Computação Aplicada, da Universidade do Estado de Santa
%     Catarina, como requisito parcial para a obtenção do grau de Mestre em Computação Aplicada.}
% 	 \vspace{2em}
% 	\noindent

% 	{\justify \bfseries Banca Examinadora}

%   \vspace{2em}

%   \noindent{Orientadora:\hfill \assinatura*{\textbf{\imprimirorientadorRotulo \imprimirorientador} \\ Universidade do Estado de Santa Catarina (UDESC)}}

%   \noindent{Coorientador:\hfill \assinatura*{\textbf{\imprimircoorientadorRotulo \imprimircoorientador} \\ Universidade Federal de Santa Marina (UFSM)}}

%   \noindent{Membros:}

% 	\noindent{\assinatura*{\textbf{Dr. Rui Jorge Tramontin Junior} \\ Universidade do Estado de Santa Catarina (UDESC)}}
%   \noindent{\assinatura*{\textbf{Dra. Rebeca Schroeder Freitas} \\ Universidade do Estado de Santa Catarina (UDESC)}}
%   \noindent{\assinatura*{\textbf{Dr. Tiago T. Primo} \\ Universidade Federal de Pelotas (UFPel)}}
%   \noindent{\assinatura*{\textbf{Dr. José Palazzo M. de Oliveira} \\ Universidade Federal do Rio Grande do Sul (UFRGS)}}

%     \vspace*{\fill}
%     \begin{center}
%     	\imprimirlocal,\,\imprimirfulldata
%     \end{center}
% \end{folhadeaprovacao}

% ---
% Dedicatória
% ---
% \begin{dedicatoria}
% Dedico este trabalho aos meus familiares, amigos, colegas e professores que me acompanharam e me deram forças nessa magnífica trajetória.
% \end{dedicatoria}

% ---
% Agradecimentos
% ---
\begin{agradecimentos}
Gostaria de agradecer primeiramente aos meus pais, José Carlos e Sandra, por todo o apoio e incetivo em minha educação
e por terem feito de mim a pessoa que sou hoje. Por participarem das decisões difíceis e por darem suporte sempre que
necessário para que eu pudesse alcançar os meus sonhos. Eles fizeram (e ainda fazem) muito mais por mim do que muitos
pais fazem pelos seus filhos e sou muito grato por isso.

Gostaria também de agradecer a minha namorada Amanda Mikos, pelo apoio e compreensão durante toda minha caminhada, desde
a graduação até a finalização desse ciclo importante da minha vida que foi o mestrado. Por todos os momentos de dúvida
e cansaço onde o seu apoio e paciência foram fundamentais. Sei que você sempre acreditou em mim, mesmo quando eu não acreditava,
e também acredito e torço muito por você.

Gostaria de agredecer imensamente aos meus orientadores Dra. Isabela Gasparini e Dr. Daniel Lichtnow por acreditarem no
meu trabalho e me guiarem durante todo o processo. Foram caminhos difíceis, mas com às suas luzes me guiando foi possível
chegar até aqui. Espero que vocês tenham apreciado esse trajeto como eu também apreciei.

Gostaria de agradecer aos então alunos do Bacharelado em Ciência da Computação Matheus Bombassaro, Caroline Sala, Luiz Engler
e Cláudia Pimental com a ajuda na execução do experimento. Bem como do então aluno do Mestrado em Ensino de Ciências,
Matemática e Tecnologias na elaboração do dicionário de palavras-chave.

Agradeço aos professores da banca Dr. Rui Jorge Tramontin Junior, Dra. Rebeca Schroeder Freitas, Dr. Tiago T. Primo e
Dr. José Palazzo M. de Oliveira que tanto na qualificação quanto na defesa do trabalho contribuiram com comentários muito
relevantes que ajudou também a guiar o desenvolvimento dessa dissertação.

Agradeço aos demais professores da UDESC - CCT que participaram de alguma forma da minha trajetória, tanto na graduação
quanto no mestrado, e também aos demais amigos e familiares que me apoiaram e torceram por mim.
\end{agradecimentos}

% ---
% Epígrafe
% ---
% \begin{epigrafe}
% ``Independentemente das circunstâncias, devemos ser sempre humildes, recatados e despidos de orgulho.''
% \\
% \par
% Dalai Lama
% \end{epigrafe}

% ---
% RESUMOS
% ---

% Português
\begin{resumo}
  Sistemas de Recomendação (SR) são ferramentas de \textit{software} que sugerem itens para os usuários de forma automatizada e
  personalizada, sem a necessidade do usuário formular uma consulta para encontrar os itens do seu interesse. Esses
  sistemas são explorados em Ambientes Virtuais de Aprendizagem (AVA) com o objetivo de reduzir alguns problemas
  existentes nesses ambientes quando a quantidade de materiais disponíveis é grande, tais como: sobrecarga cognitiva,
  dificuldade de encontrar os materiais do seu interesse e muitos materiais nunca serem utilizados. Pesquisadores da
  área argumentam que os algoritmos de SRs tradicionais não são suficientes para os AVAs, sendo necessário um nível
  maior de personalização da situação do usuário, como considerar informações do seu contexto. O objetivo desse trabalho é
  avaliar se considerar a variação temporal dos interesses do usuário em itens acessados anteriormente em SRs voltados a AVAs
  influencia o desempenho da abordagem de recomendação e a percepção dos alunos sobre as recomendações recebidas. O
  algoritmo com decaimento proposto combina a (1) similaridade do perfil do usuário
  (representado pelos materiais acessados pelo usuário) com os itens disponíveis para a recomendação com a (2) recência
  do acesso ou uso desse materiais, além da (3) informação se aquele item disponível para a recomendação já foi acessado
  ou não. A proposta leva em conta que o ritmo de estudo dos alunos pode ser diferente, portanto a recência é
  considerada em relação à sequência de itens acessados e não ao tempo absoluto (em segundos) desde o acesso. A proposta
  desse trabalho foi incorporada ao ambiente \adaptwebspace e avaliada através de um experimento utilizando um Minicurso
  de Algoritmos e Linguagem de Programação ministrado no ambiente. O algoritmo proposto foi comparado à abordagem
  Baseada em Conteúdo Tradicional utilizando uma estratégia \textit{Between Subjects}. Os resultados mostraram
  que existe diferença significativa em relação a abordagem Tradicional na Cobertura, F-measure e em uma das questões
  sobre a percepção do usuário e que a abordagem com Decaimento teve resultados melhores nessas métricas. Portanto, as
  duas hipóteses alternativas definidas para o experimento foram aceitas e indicam que considerar o Decaimento em um SR
  para AVAs influencia positivamente o desempenho do algoritmo e a percepção dos alunos sobre as recomendações.

  \vspace{\onelineskip}

  \noindent
  \textbf{Palavras-chaves}: Sistema de Recomendação; Sensível ao Tempo; Sensível ao Contexto; Decaimento; Ambiente Virtual de Aprendizagem; \adaptweb.
\end{resumo}

% Inglês
\begin{resumo}[Abstract]
 \begin{otherlanguage*}{english}
  	Recommender Systems (RS) are software tools that provide items as suggestions to users automatically and personalized to his
    interests, without the need to formulate a search argument to achieve them. This systems are applied to Virtual Learning
    Environments (VLE) aiming to reduce some drawbacks existing in these enviroments when the number of available items is
    huge, e.g., cognitive overload, difficulty finding items of user's interest or some materials never get accessed. Researchers
    in this area argues that traditional RS approaches are not enough for VLE, being required a major level of personalization
    to user's current context. This work goals is to evaluate whether the use of Decay on user's interest for past items
    on RS for VLEs influence the algorithm performance and the user's perception of the recommendations. The proposed
    algorithm combines (1) similarity between user profile (represented by the materials accessed by the user) with the items
    available to recommendation with (2) the recency of materials accessed by the user and (3) the information about if
    the item available to be recommended was accessed or not. The proposal takes into account that each learner can have
    a different study rhythm, therefore the recency considers the sequence of items accessed and not the absolute time (in
    seconds) from the access. The proposal of this work was incorporated to the \adaptwebspace environment and evaluated
    through an experiment using an Algorithms and Programming Language's course. The proposal was compared to the Content-Based traditional approach
    through using a Between Subjects strategy. The results show that there is significant difference on the Coverage,
    F-measure and in one of the questions about the user's perception and that the approach with Decay had better results
    in these metrics. Therefore, both alternative hypothesis defined for the experiment were accepted and it indicates
    that consider the Decay on a RS for VLEs influences the algorithm performance and the user's perception of the recommendations.
    \vspace{\onelineskip}

    \noindent
    \textbf{Keywords}: Recommender System; Time-Aware; Context-Aware; Decay; Virtual Learning Environment; \adaptweb.
 \end{otherlanguage*}
\end{resumo}

% ---
% Lista de Figuras
% ---
\pdfbookmark[0]{\listfigurename}{lof}
\listoffigures*
\cleardoublepage
% ---

% ---
% Lista de Tabelas
% ---
\pdfbookmark[0]{\listtablename}{lot}
\listoftables*
\cleardoublepage
