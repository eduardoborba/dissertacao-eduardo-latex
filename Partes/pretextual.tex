%!TEX root = ../Principal.tex
%Capa do Trabalho
\imprimircapa

%Folha de Rosto
%* indica que tem ficha catalográfica
\imprimirfolhaderosto*

% ---
% Caso a Biblioteca da UDESC forneça, utilize o comando
% ---
% \begin{fichacatalografica}
%     \includepdf{fig_ficha_catalografica.pdf}
% \end{fichacatalografica}

% ---
% Geração da Ficha Catalográfica Via LaTeX
% ---
% \begin{fichacatalografica}
% 	\vspace*{\fill}					% Posição vertical
% 	\begin{center}					% Minipage Centralizado
% 	\begin{minipage}[c]{12.5cm}		% Largura

% 	\imprimirautor

% 	\hspace{0.5cm} \imprimirtitulo  / \imprimirautor. --
% 	\imprimirlocal, \imprimirdata-

% 	\hspace{0.5cm} \pageref{LastPage} p. : il. (algumas color.) ; 30 cm.\\

% 	\hspace{0.5cm} \imprimirorientadorRotulo~\imprimirorientador\\

% 	\hspace{0.5cm}
% 	\parbox[t]{\textwidth}{\imprimirtipotrabalho~--~\imprimirinstituicao,
% 	\imprimirdata.}\\

% 	\hspace{0.5cm}
% 		1. Tópico 01.
% 		2. Tópico 02.
% 		I. Prof. Dr. xxxxx.
% 		II. Universidade do Estado de Santa Catarina.
% 		III. Centro de Ciências Tecnológicas.
% 		IV. identificação xxxx\\

% 	\hspace{8.75cm} CDU 02:121:005.7\\

% 	\end{minipage}
% 	\end{center}
% \end{fichacatalografica}

% % ---
% % Folha de Aprovação
% % ---
% % Exemplo de folha de aprovação antes da Banca. Após isso, incluia o pdf digitalizado com as assinaturas%
% % \includepdf{folhadeaprovacao_final.pdf}
\begin{folhadeaprovacao}
	\begin{center}
		{\ABNTEXchapterfont\bfseries\imprimirautor}
		\vspace{2em}

			\ABNTEXchapterfont\bfseries\imprimirtitulo

	\end{center}
		\vspace{1em}
		{\justify
    Qualificação apresentada ao Programa de Pós-Graduação em Computação Aplicada, da Universidade do Estado de Santa
    Catarina, como requisito parcial para a obtenção do grau de Mestre em Computação Aplicada.}
	 \vspace{2em}
	\noindent

	{\justify \bfseries Banca Examinadora}

  \vspace{2em}

  \noindent{Orientadora:\hfill \assinatura*{\textbf{\imprimirorientador} \\ Universidade do Estado de Santa Catarina (UDESC)}}

  \noindent{Coorientador:\hfill \assinatura*{\textbf{\imprimircoorientador} \\ Universidade Federal de Santa Marina (UFSM)}}

	\noindent{Membros:\hfill \assinatura*{\textbf{Dr. Rui Jorge Tramontin Junior} \\ Universidade do Estado de Santa Catarina (UDESC)}}

  \hfill \assinatura*{\textbf{Dr. Thiago T. Primo (UFPel)} \\ Universidade Federal de Pelotas (UFPel)}

  \hfill \assinatura*{\textbf{Dr. José Palazzo M. de Oliveira (UFRGS)} \\ Universidade Federal do Rio Grande do Sul (UFRGS)}

    \vspace*{\fill}
    \begin{center}
    	\imprimirlocal,\,\imprimirfulldata
    \end{center}
\end{folhadeaprovacao}

% ---
% Dedicatória
% ---
% \begin{dedicatoria}
% Dedico este trabalho aos meus familiares, amigos, colegas e professores que me acompanharam e me deram forças nessa magnífica trajetória.
% \end{dedicatoria}

% % ---
% % Agradecimentos
% % ---
% \begin{agradecimentos}
% Gostaria de agradecer...

% Aqui devem ser colocadas os agradecimentos às pessoas que de alguma forma contribuíram para a realização do trabalho.
% \end{agradecimentos}

% ---
% Epígrafe
% ---
% \begin{epigrafe}
% ``Independentemente das circunstâncias, devemos ser sempre humildes, recatados e despidos de orgulho.''
% \\
% \par
% Dalai Lama
% \end{epigrafe}

% ---
% RESUMOS
% ---

% Português
\begin{resumo}
  Sistemas de Recomendação (SR) são ferramentas de software que sugerem itens para os usuários de forma automatizada e personalizada,
  sem a necessidade do usuário formular uma consulta para encontrar os itens do seu interesse. Esses sistemas são
  explorados em Ambientes Virtuais de Aprendizagem (AVA) com o objetivo de reduzir alguns problemas existentes nesses ambientes
  quando a quantidade de materiais disponíveis é grande, tais como: sobrecarga cognitiva, dificuldade de encontrar os materiais
  do seu interesse e muitos materiais nunca serem utilizados. Pesquisadores da área argumentam que os algoritmos de SRs tradicionais não são suficientes para os AVAs,
  sendo necessário um nível maior de personalização a situação do usuário, como considerar dimensões do contexto. O objetivo
  desse trabalho é a criação de perfis de usuário que levem em conta a mudança dos interesses destes usuários
  ao longo do tempo. O algoritmo proposto combina a (1) similaridade do perfil do usuário (representado
  pelos materiais acessados pelo usuário) com os itens disponíveis para a recomendação com a (2) recência do acesso ou uso
  desse materiais, além da (3) informação se aquele item disponível para a recomendação já foi acessado ou não. A
  proposta leva em conta que o ritmo de estudo dos alunos pode ser diferente, portanto a recência é considerada em relação
  a sequência de itens acessados e não ao tempo absoluto (em segundos) desde o acesso. A proposta desse trabalho será
  incorporada ao ambiente \adaptweb e será avaliado através de um minicurso de algoritmos ministrado no ambiente.
  O algoritmo proposto será comparado a abordagem Baseada em Conteúdo tradicional através de um experimento utilizando um
  estratégia \textit{Between Subjects}. O objetivo do experimento é verificar se existe diferença na percepção do usuário
  sobre a qualidade das recomendações do algoritmo proposto em relação a abordagem Baseada em Conteúdo tradicional.

  \vspace{\onelineskip}

  \noindent
  \textbf{Palavras-chaves}: Sistema de Recomendação; Sensível ao Tempo; Sensível ao Contexto; Ambiente Virtual de Aprendizagem; \adaptweb.
\end{resumo}

% Inglês
\begin{resumo}[Abstract]
 \begin{otherlanguage*}{english}
  	Recommender Systems (RS) are software tools that provide items as suggestions to users automatically and personalized to his
    interests, without the need to formulate a search argument to achieve this. This systems are applied to Virtual Learning
    Environments (VLE) aiming to reduce some drawbacks existing in this enviroments when the number of available items is
    huge, e.g., cognitive overload, difficulty finding items of user's interest or some materials never get used. Researchers
    in this area argues that traditional RS approaches are not enough for VLE, being required a major level of personalization
    to user's current context. This work goals is to create user profiles that take into account the changing of user's interests
    with the time. This profiles are going to be applied in the recommendation algorithm proposed in this work. The proposed
    algorithm combines (1) similarity between user profile (represented by materials accessed by the user) with the items
    available to recommendation with (2) the recency of materials accessed by the user and (3) the information about if
    the item available to be recommended were accessed or not. The proposal takes into account that each learner can have
    a different study rhythm, therefore the recency considers the sequence of items accessed and not the absolute time (in
    seconds) from the access. The proposal of this work will be incorporate to the \adaptweb environment and will be evaluated
    through an algorithms course in the environment. The proposal will be compared to the Content-Based traditional approach
    through an experiment using a Between Subjects strategy. The objective of this experiment is to verify if exists differences
    in user's perception of quality recommendations between the proposal when compared with Content-Based approach.
    \vspace{\onelineskip}

    \noindent
    \textbf{Keywords}: Recommender System; Time-Aware; Context-Aware; Virtual Learning Environment; \adaptweb.
 \end{otherlanguage*}
\end{resumo}

% ---
% Lista de Figuras
% ---
\pdfbookmark[0]{\listfigurename}{lof}
\listoffigures*
\cleardoublepage
% ---

% ---
% Lista de Tabelas
% ---
\pdfbookmark[0]{\listtablename}{lot}
\listoftables*
\cleardoublepage