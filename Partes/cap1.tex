\chapter{Introdução}\label{introducao}

\section{Objetivos}

Foram definidos objetivos geral e específicos para orientar o processo de pesquisa desse trabalho.

\subsection{Objetivo Geral}

Considerando sistemas de recomendação voltados para ambientes de aprendizagem, criar modelos de usuário que levem em
conta a mudança dos interesses destes usuários ao longo do tempo.

\subsection{Objetivos Específicos}

\begin{itemize}
\item Definir algumas formas de levar em conta aspectos temporais em um algoritmo de recomendação
\item Definir como usar o aspecto temporal em sistemas de recomendação para ambiente educacionais
\item Implementar o algoritmo de recomendação proposto do ambiente AdaptWeb\textsuperscript{\textregistered}
\item Considerar diretrizes para a apresentação de recomendação para desenvolver a interface do sistema de recomendação
no ambiente AdaptWeb\textsuperscript{\textregistered}
\item Avaliar a qualidade de uso do sistema de recomendação proposto pela perspectiva do usuário
\end{itemize}

\section{Escopo}

Esse trabalho não irá considerar outras dimensões do contexto além do tempo na recomendação, e dentro do uso do contexto
temporal apenas a categoria de \textit{Decay} será aplicada nesse trabalho. Além disso, a única abordagem de recomendação
utilizada será a Baseada em Conteúdo, apesar de as categorias de Sistemas de Recomendação Sensíveis ao Tempo poderem ser
aplicadas em quaisquer abordagens de recomendação. A avaliação do Sistema de Recomendação proposto será feita apenas em ambientes educacionais, mesmo sendo
possível aplicá-lo em outros domínios de aplicação. Não será avaliado, nesse trabalho, o impacto da proposta na aprendizagem
dos alunos.

\section{Metodologia}

\section{Estrutura}



