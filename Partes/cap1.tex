\chapter{Introdução}\label{introducao}

Um Ambiente Virtual de Aprendizagem (AVA) é um ambiente computacional com a finalidade de integrar diversas mídias
(e.g., vídeos, apresentações, textos) e dar suporte à educação online \cite{drachsler2015panorama}. Esses ambientes, além de simularem uma sala
de aula permitindo o relacionamento professor-aluno e aluno-aluno, disponibilizam conteúdos e materiais para os usuários
poderem acessar.

Quando a quantidade de materiais disponíveis nos AVAs é muito grande, existem alguns problemas que podem acontecer. São eles:

\begin{itemize}
\item O aluno sofrer uma sobrecarga cognitiva, aumentando o esforço necessário para compreender o ambiente
e encontrar os itens de seu interesse, atrapalhando o processo de aprendizagem;
\item O aluno não encontrar um material que seja de seu interesse, devido a enorme quantidade de materiais disponíveis;
\item Parte do material disponibilizado, que poderia auxiliar os alunos no processo de aprendizagem, nunca ser
utilizado.
\end{itemize}

Com o objetivo de reduzir esses problemas, pesquisadores tem aplicado técnicas de personalização para selecionar os melhores itens para cada estudante,
considerando o seu conhecimento, objetivos, preferências e necessidades \cite{brusilovsky1998methods}. Os Sistemas de Recomendação são um alternativa
para reduzir esses problemas, sugerindo itens para o usuários utilizando informações sobre seus interesses e sobre os itens disponíveis \cite{adomavicius2005toward}.

Porém, pesquisadores da área argumentam que no domínio educacional os SRs tradicionais (aqueles que consideram apenas
informações sobre os usuários e os itens para recomendar) não são suficientes \cite{verbert2012context, drachsler2015panorama}.
\citeonline{verbert2012context} afirmam que nessa área é necessário um nível maior de personalização, como utilizar informações
do contexto do usuário na recomendação.

Apesar de existir uma grande quantidade de trabalhos utilizando o contexto em SRs no domínio educacional, como pode ser visto
em \citeonline{verbert2012context} e \citeonline{drachsler2015panorama}, pouco foi encontrado da aplicação do contexto
temporal nesse domínio \cite{de2017time}. O contexto temporal é relevante pois leva em consideração a variação dos
interesses do usuário com o passar do tempo. Além disso, os SR Sensíveis ao Tempo demonstraram bons resultados em outros
domínios de aplicação \cite{campos2014time}.

\section{Problema}

Como dito anteriormente, foram encontrados poucos trabalhos sobre o uso do contexto temporal em SRs para AVAs. Além disso,
nos trabalhos encontrados as propostas não foram avaliadas em ambientes reais de uso, não sendo possível avaliar o
impacto dos SRs Sensíveis ao Tempo nesse domínio. Assim, a pergunta a ser respondida por este trabalho
é: "\textbf{Como considerar a variação dos interesses do aluno com o passar do tempo em um AVA para recomendar itens para ele?}".

\section{Objetivos}

Foram definidos objetivos geral e específicos para orientar o processo de pesquisa desse trabalho buscando responder a questão
de pesquisa definida acima.

\subsection{Objetivo Geral}

Criar modelos de usuário e algoritmos para sistemas de recomendação voltados a AVAs que levem em
conta a mudança dos interesses destes usuários ao longo do tempo.

\subsection{Objetivos Específicos}

\begin{itemize}
\item Definir formas de levar em conta aspectos temporais em um algoritmo de recomendação;
\item Definir como usar o aspecto temporal em sistemas de recomendação para ambiente educacionais;
\item Implementar o algoritmo de recomendação proposto em um AVA;
\item Considerar diretrizes de apresentação de recomendação ao desenvolver a interface do sistema de recomendação
em um AVA;
\item Avaliar a percepção do usuário sobre o algoritmo proposto em um ambiente real de uso.
\end{itemize}

\section{Escopo}

Esse trabalho não irá considerar outras dimensões do contexto além do tempo na recomendação, e dentro do uso do contexto
temporal apenas a categoria de \textit{Decay} será aplicada nesse trabalho, que está relacionada a perda de interesse por
itens acessados anteriormente. Além disso, a única abordagem de recomendação utilizada será a Baseada em Conteúdo,
apesar de as categorias de Sistemas de Recomendação Sensíveis ao Tempo poderem ser aplicadas em quaisquer abordagens de
recomendação. A avaliação do Sistema de Recomendação proposto será feita apenas em ambientes educacionais, mesmo sendo
possível aplicá-lo em outros domínios de aplicação. Não será avaliado, nesse trabalho, o impacto da proposta na
aprendizagem dos alunos, pois apesar de o objetivo final ser a melhoria da aprendizagem, esse tipo de avaliação exigiria uma
comparação de grupos com e sem o uso do SR utilizando-se de algum método confiável de medir a
aprendizagem, algo que é ainda muito discutido por pedagogos \cite{luckesi2014avaliaccao}.

\section{Metodologia}

A pesquisa desse trabalho é de natureza Aplicada, pois busca gerar conhecimentos através da implementação e experimentação
de SRs em um ambiente real de uso. A abordagem do problema deste trabalho é quantitativa, através de questionários utilizando
escala de Likert para quantificar a opinião dos alunos. Os objetivos dessa pesquisa tem caráter Explicativo, visando
identificar fatores que influenciam a percepção dos alunos sobre a qualidade das recomendações. O procedimento utilizado
para o desenvolvimento dessa pesquisa é Experimental, onde o objeto de estudo é a percepção dos alunos sobre a qualidade
das recomendações e a variável é o algoritmo de recomendação utilizado.

\section{Estrutura}

Este trabalho está estruturado da seguinte forma: o Capítulo \ref{chapter:fundamentacao-teorica} conceitua os Sistemas de
Recomendação (SR), as suas abordagens, as formas de avaliação e a apresentação das recomendações; o Capítulo \ref{chapter:trabalhos-relacionados}
descreve os trabalhos relacionados que utilizam a categoria \textit{Decay} nas recomendações e compara com a proposta desse
trabalho; o Capítulo \ref{chapter:proposta} apresenta em detalhe a proposta desse trabalho; o Capítulo \ref{chapter:experimento}
apresenta o experimento utilizado para avaliação dessa proposta. Por último, o Capítulo \ref{chapter:conclusoes} apresenta
as considerações parciais deste trabalho, recapitulando o que foi feito até o momento.


