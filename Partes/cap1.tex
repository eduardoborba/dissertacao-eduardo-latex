%!TEX root = ../Principal.tex
\chapter{Introdução}\label{introducao}

% ---
\section{Objetivos}

Foram definidos objetivos geral e específicos para orientar o processo de pesquisa desse trabalho.

\subsection{Objetivo Geral}

IDEIAS:
Aplicar o contexto temporal em um algoritmo de recomendação para ambientes educacionais
Criar um algoritmo de recomendação que considere o decaimento do interesse dos alunos nos materiais em função do
tempo categórico
Criar um algoritmo de recomendação que considere o decaimento do interesse dos alunos nos materiais em função da
sequência de acessos realizados
Criar um sistema de recomendação sensível ao tempo para garantir uma recomendação apropriada a situação atual dos alunos

\subsection{Objetivos Específicos}

\begin{itemize}
\item Definir como usar o aspecto temporal em sistemas de recomendação para ambiente educacionais
\item Implementar o algoritmo de recomendação proposto do ambiente AdaptWeb\textsuperscript{\textregistered}
\item Considerar diretrizes para a apresentação de recomendação para desenvolver a interface do sistema de recomendação
no ambiente AdaptWeb\textsuperscript{\textregistered}
\item Avaliar a qualidade de uso do sistema de recomendação proposto pela perspectiva do usuário
\end{itemize}

\section{Escopo}

Esse trabalho não irá considerar outras dimensões do contexto além do tempo na recomendação, e dentro do uso do contexto
temporal apenas a categoria de \textit{Decay} será aplicada nesse trabalho. Além disso, a única abordagem de recomendação
utilizada será a Baseada em Conteúdo, apesar de as categorias de Sistemas de Recomendação Sensíveis ao Tempo poderem ser
aplicadas em quaisquer abordagens de recomendação. Tanto o algoritmo de recomendação quanto a apresentação dessas
recomendações são propostas para se adequar aos objetivos e a estrutura do ambiente AdaptWeb\textsuperscript{\textregistered}.
Nesse trabalho as palavras-chave que descrevem os materiais de estudo são cadastradas pelo professor, e dessa forma a
qualidade da abordagem Baseada em Conteúdo considerada por este trabalho tem como limitação a categorização correta dos
itens no sistema.

\section{Metodologica}

\section{Estrutura}



