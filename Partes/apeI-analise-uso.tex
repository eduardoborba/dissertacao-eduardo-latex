\chapter{Análise Estatística do Desempenho do Sistema de Recomendação}\label{ape:analise-estatistica-do-uso}

\section{Links Acessados por Aluno}

\begin{figure}[htb]
  \caption{\label{fig:uso-sr-boxplot}Boxplot dos links acessados por aluno}
  \begin{center}
      \includegraphics[scale=0.4]{./Figuras/uso-sr-boxplot.png}
  \end{center}
  \legend{Fonte: O autor.}
\end{figure}

\noindent
Total de alunos comparados: 85

\begin{multicols}{2}

\noindent\textbf{Tradicional}\\
Min =  1.000\\
1\textsuperscript{o} Quad =  2.000\\
Mediana =  4.000\\
Média =  4.935\\
3\textsuperscript{o} Quad = 6.500\\
Max = 18.000\\

\columnbreak

\noindent\textbf{Proposta}\\
 Min =   1.00\\
 1\textsuperscript{o} Quad =   2.00\\
 Mediana =   5.00\\
 Média =  10.15\\
 3\textsuperscript{o} Quad =  9.00\\
 Max = 107.00
\end{multicols}

Shapiro-Wilk normality test

\noindent
data:  data[["quantidade"]]\\
W = 0.42461, p-value < 2.2e-16

\noindent
\textbf{Resultado: Aceita a hipótese alternativa - Distribuição não normal}

Wilcoxon rank sum test with continuity correction

\noindent
data:  data[["quantidade"]] by data[["algoritmo\_recomendacao"]]\\
W = 820, p-value = 0.4957\\
alternative hypothesis: true location shift is not equal to 0

\noindent
\textbf{Resultado: Aceita a hipótese nula - Sem diferença significativa}

\newpage
\section{Links Avaliados Positivamente por Aluno}

\begin{figure}[htb]
  \caption{\label{fig:avaliados-positivamente-boxplot}Boxplot dos links avaliados positivamente por aluno}
  \begin{center}
      \includegraphics[scale=0.4]{./Figuras/avaliados-positivamente-boxplot.png}
  \end{center}
  \legend{Fonte: O autor.}
\end{figure}

\noindent
Total de alunos comparados: 52

\begin{multicols}{2}

\noindent\textbf{Tradicional}\\
Min =  1.0\\
1\textsuperscript{o} Quad =  1.0\\
Mediana =  4.0\\
Média =  4.7\\
3\textsuperscript{o} Quad = 6.0\\
Max = 16.0\\
\columnbreak

\noindent\textbf{Proposta}\\
Min =   1.00\\
1\textsuperscript{o} Quad =   1.00\\
Mediana =   2.50\\
Média =  10.64\\
3\textsuperscript{o} Quad =  9.75\\
Max = 107.00
\end{multicols}

  Shapiro-Wilk normality test

\noindent
data:  data[["quantidade"]]\\
W = 0.37836, p-value = 1.654e-13

\noindent
\textbf{Resultado: Aceita a hipótese alternativa - Distribuição não normal}

Wilcoxon rank sum test with continuity correction

\noindent
data:  data[["quantidade"]] by data[["algoritmo\_recomendacao"]]\\
W = 318, p-value = 0.8275\\
alternative hypothesis: true location shift is not equal to 0

\noindent
\textbf{Resultado: Aceita a hipótese nula - Sem diferença significativa}

\newpage
\section{Links avaliados negativamente por aluno que acessou pelo menos uma recomendação}

\begin{figure}[htb]
  \caption{\label{fig:uso-sr-boxplot}Boxplot dos links acessados por aluno}
  \begin{center}
      \includegraphics[scale=0.4]{./Figuras/uso-sr-boxplot.png}
  \end{center}
  \legend{Fonte: O autor.}
\end{figure}

\noindent
Total de alunos comparados: 7

\begin{multicols}{2}

\noindent\textbf{Tradicional}\\
Min = 1\\
1\textsuperscript{o} Quad = 1\\
Mediana = 1\\
Média = 1\\
3\textsuperscript{o} Quad =1\\
Max = 1\\

\columnbreak

\noindent\textbf{Proposta}\\
Min = 1.0\\
1\textsuperscript{o} Quad = 1.5\\
Mediana = 2.0\\
Média = 2.0\\
3\textsuperscript{o} Quad =2.5\\
Max = 3.0
\end{multicols}

  Shapiro-Wilk normality test

\noindent
data:  data[["quantidade"]]\\
W = 0.45297, p-value = 4.136e-06

\noindent
\textbf{Resultado: Aceita a hipótese alternativa - Distribuição não normal}

Wilcoxon rank sum test with continuity correction

\noindent
data:  data[["quantidade"]] by data[["algoritmo\_recomendacao"]]\\
W = 2.5, p-value = 0.2059\\
alternative hypothesis: true location shift is not equal to 0

\noindent
\textbf{Resultado: Aceita a hipótese nula - Sem diferença significativa}

\newpage
\section{Precisão dos algoritmos de recomendação por aluno}

\begin{figure}[htb]
  \caption{\label{fig:precisao-boxplot}Boxplot da precisão dos algoritmos de recomendação por aluno}
  \begin{center}
      \includegraphics[scale=0.4]{./Figuras/precisao-boxplot.png}
  \end{center}
  \legend{Fonte: O autor.}
\end{figure}

\noindent
Total de alunos comparados: 85

\begin{multicols}{2}

\noindent\textbf{Tradicional}\\
Min = 0.04348\\
1\textsuperscript{o} Quad = 0.20000\\
Mediana = 0.42857\\
Média = 0.49679\\
3\textsuperscript{o} Quad =0.81364\\
Max = 1.00000\\

\columnbreak

\noindent\textbf{Proposta}\\
Min = 0.05556\\
1\textsuperscript{o} Quad = 0.18333\\
Mediana = 0.31818\\
Média = 0.39859\\
3\textsuperscript{o} Quad =0.52778\\
Max = 1.00000
\end{multicols}

  Shapiro-Wilk normality test

\noindent
data:  data[["precisao"]]\\
W = 0.89495, p-value = 4.234e-06

\noindent
\textbf{Resultado: Aceita a hipótese alternativa - Distribuição não normal}

Wilcoxon rank sum test with continuity correction

\noindent
data:  data[["precisao"]] by data[["algoritmo\_recomendacao"]]\\
W = 1025, p-value = 0.2603\\
alternative hypothesis: true location shift is not equal to 0

\noindent
\textbf{Resultado: Aceita a hipótese nula - Sem diferença significativa}

\newpage
\section{Cobertura dos algoritmos de recomendação por aluno}

\begin{figure}[htb]
  \caption{\label{fig:coverage-boxplot}Boxplot da cobertura dos algoritmos de recomendação por aluno}
  \begin{center}
      \includegraphics[scale=0.4]{./Figuras/coverage-boxplot.png}
  \end{center}
  \legend{Fonte: O autor.}
\end{figure}

\noindent
Total de alunos comparados: 85

\begin{multicols}{2}

\noindent\textbf{Tradicional}\\
Min = 0.04630\\
1\textsuperscript{o} Quad = 0.06481\\
Mediana = 0.09259\\
Média = 0.10064\\
3\textsuperscript{o} Quad =0.12963\\
Max = 0.21296\\

\columnbreak

\noindent\textbf{Proposta}\\
Min = 0.04630\\
1\textsuperscript{o} Quad = 0.08796\\
Mediana = 0.14815\\
Média = 0.17450\\
3\textsuperscript{o} Quad =0.16667\\
Max = 0.99074
\end{multicols}

    Shapiro-Wilk normality test

\noindent
data:  data[["coverage"]]\\
W = 0.55934, p-value = 1.421e-14

\noindent
\textbf{Resultado: Aceita a hipótese alternativa - Distribuição não normal}

  Wilcoxon rank sum test with continuity correction

\noindent
data:  data[["coverage"]] by data[["algoritmo\_recomendacao"]]\\
W = 525.5, p-value = 0.001031\\
alternative hypothesis: true location shift is not equal to 0

\noindent
\textbf{Resultado: Aceita a hipótese alternativa - Com diferença significativa}

\newpage
\section{F-measure do algoritmo de recomendação por aluno}

\begin{figure}[htb]
  \caption{\label{fig:media-harmonica-boxplot}Boxplot do F-measure dos dois algoritmos}
  \begin{center}
      \includegraphics[scale=0.4]{./Figuras/media-harmonica-boxplot.png}
  \end{center}
  \legend{Fonte: O autor.}
\end{figure}

\noindent
Total de alunos comparados: 85

\begin{multicols}{2}

\noindent\textbf{Tradicional}\\
Min = 0.07221\\
1\textsuperscript{o} Quad = 0.09545\\
Mediana = 0.12952\\
Média = 0.13852\\
3\textsuperscript{o} Quad =0.16618\\
Max = 0.30717\\

\columnbreak

\noindent\textbf{Proposta}\\
Min = 0.07519\\
1\textsuperscript{o} Quad = 0.10913\\
Mediana = 0.16393\\
Média = 0.21719\\
3\textsuperscript{o} Quad =0.24919\\
Max = 0.99535
\end{multicols}

Shapiro-Wilk normality test

\noindent
data:  data[["media\_harmonica"]]\\
W = 0.63292, p-value = 2.931e-13

\noindent
\textbf{Resultado: Aceita a hipótese alternativa - Distribuição não normal}

Wilcoxon rank sum test with continuity correction

\noindent
data:  data[["media\_harmonica"]] by data[["algoritmo\_recomendacao"]]\\
W = 634.5, p-value = 0.02082\\
alternative hypothesis: true location shift is not equal to 0

\noindent
\textbf{Resultado: Aceita a hipótese alternativa - Com diferença significativa}
