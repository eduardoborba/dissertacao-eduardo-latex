\chapter{Planejamento do Experimento}

Neste capítulo é apresentado o experimento que será realizado para a avaliação da proposta apresentada no Capítulo \ref{cap:proposta}.
O experimento que será realizado para avaliar a proposta deste trabalho irá utilizar uma avaliação centrada no usuário,
que segundo a definição de \citeonline{shani2011evaluating} se encaixa em um Estudo com usuários. O experimento realizado
no ambiente AdaptWeb\textsuperscript{\textregistered}, o ambiente no qual foi incorporado o Sistema de Recomendação (SR)
proposto, em um minicurso de algoritmos do qual serão convidados a participar alunos de todos os cursos do Centro de
Ciências Tecnológicas (CCT) da Universidade do Estado de Santa Catarina (UDESC). Isso porque em todos os cursos do CCT
a disciplina de algoritmos está presente.

Nesse minicurso costumam se matricular em média 100 alunos por semestre. Sendo que desses, pelo menos 60 alunos chegam até
o final do curso e realizam a prova final. O curso terá a duração de aproximadamente 45 dias e irá ocorrer no primeiro semestre
de 2018, entre os meses de Abril e Maio.

\section{Definição do experimento}

Para a execução do experimento, o SR proposto será comparado a abordagem Baseada em Conteúdo tradicional utilizando uma
estratégia \textit{Between Subjects}, i.e., os usuários serão divididos em dois grupos e cada grupo irá testar apenas
um dos sistemas. Porém, para garantir que a única variável será qual o SR utilizado, ambos os grupos irão utilizar a mesma
interface proposta para as recomendações.

Os usuários ao se matricularem no minicurso serão aleatóriamente divididos nos dois grupos de usuários, de forma que ambos
os grupos tenham uma quantidade de usuários similar. Além disso, os alunos irão receber um Termo de Consentimento Livre
e Esclarecido (TCLE) que explica o objetivo do experimento e no qual eles irão consentir em participar dos experimentos e
em permitir o uso dos resultados do experimento para essa pesquisa, sempre garantindo a anonimidade dos participants.

Ao final dos 45 dias do minicurso, os alunos irão receber um questionário para responder sobre a sua experiência com o SR.
Esse questionário será produzindo utilizando um subconjunto das perguntas definidas por \citeonline{pu2011user}, que deverão ser
traduzidas para o português para garantir o entendimento de todos os alunos. Os resultados dos questionários serão
analisados através de métodos estatísticos para definir se existe diferença significativa entre a Experiência do Usuário
nos dois SRs (o proposto e o Baseado em Conteúdo tradicional) e, se existir, qual deles teve o melhor desempenho.

\section{Minicurso de Algoritmos}

\section{Desenvolvimento dos instrumentos}

\section{Teste piloto}

Antes do experimento ser realizado com os alunos das disciplinas de algoritmos do CCT, será realizado um teste piloto com
outros alunos da universidade que já realizaram essa disciplina. O objetivo desse teste piloto será avaliar os
instrumentos do experimento, além de permitir encontrar problemas na experiência do  usuário para serem corrigidos
antes da execução do minicurso.

Para o teste piloto, os alunos receberão o TCLE ao qual eles também serão pedidos a assinar e serão pedidos a realizar algumas
tarefas dentro do minicurso de algoritmos enquanto são observados por um pesquisador. As tarefas envolverão o acesso a
alguns conceitos, exemplos, exercícios e materiais complementares, como também a ferramenta de recomendação. Ao final, os
participantes do teste piloto irão responder ao questionário desenvolvido (o mesmo que será aplicado no experimento) e
serão entrevistados pelo pesquisador sobre problemas encontrados nos instrumentos.

Com o resultado da observação do pesquisador e do questionário respondido pelos participantes do teste piloto será possível
identificar problemas na experiência do usuário para serem corrigidos antes da execução do experimento. Já o objetivo da
entrevista será exclusivamente para aprimorar os instrumentos (i.e., TCLE e o questionário desenvolvido).

\section{Considerações sobre o capítulo}
