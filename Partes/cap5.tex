\chapter{Planejamento do Experimento}\label{chapter:experimento}

Neste capítulo é apresentado o experimento que será realizado para a avaliação da proposta apresentada no Capítulo \ref{chapter:proposta}.
Para avaliar a proposta deste trabalho irá utilizar uma avaliação centrada no usuário, que segundo a definição de
\citeonline{shani2011evaluating} se encaixa em um Estudo com usuários. A próxima seção explica em mais detalhes o experimento.

\section{Definição do experimento}

O experimento visa avaliar a experiência dos alunos ao interagir com o SR proposto neste trabalho, se comparado a um SR
utilizando a abordagem Baseada em Conteúdo tradicional. Este experimento foi
baseado nas seguintes hipóteses:

\begin{itemize}
\item \textbf{H\textsubscript{0}:} Não há diferenças na experiência do usuário ao interagir com o SR utilizando a abordagem
Baseada em Conteúdo tradicional e a proposta desse trabalho.
\item \textbf{H\textsubscript{1}:} Há diferenças na experiência do usuário ao interagir com o SR utilizando a abordagem
Baseada em Conteúdo tradicional e a proposta desse trabalho.
\end{itemize}

Para a execução do experimento, o SR proposto será comparado a abordagem Baseada em Conteúdo tradicional utilizando uma
estratégia \textit{Between Subjects}, i.e., os alunos serão divididos em dois grupos e cada grupo irá testar apenas
um dos sistemas. Porém, para garantir que a única variável será qual o SR utilizado, ambos os grupos irão utilizar a mesma
interface proposta para as recomendações.

O experimento será realizado no ambiente AdaptWeb\textsuperscript{\textregistered}, o ambiente no qual foi incorporado o
Sistema de Recomendação (SR) proposto, em um minicurso de algoritmos do qual serão convidados a participar alunos de todos os cursos do Centro de
Ciências Tecnológicas (CCT) da Universidade do Estado de Santa Catarina (UDESC). Isso porque em todos os cursos do CCT
a disciplina de algoritmos está presente. O processo de design instrucional desse minicurso foi realizado pelo aluno
Lúcio Vasconcelos dos Santos, bolsista do Programa de Pós-Graduação em Ensino de Ciências, Matemática e Tecnologias
oferecido pela UDESC, em conjunto com os professores que lecionam a disciplina.

Nesse minicurso costumam se matricular em média 100 alunos por semestre. Sendo que desses, espera-se que pele menos 60 alunos chegam até
o final do curso e realizam a prova final. Estima-se que o minicurso terá a duração de aproximadamente 45 dias e irá ocorrer no primeiro semestre
de 2018, entre os meses de Abril e Maio.

Os usuários ao se matricularem no minicurso serão aleatóriamente divididos nos dois grupos de usuários, de forma que ambos
os grupos tenham uma quantidade de usuários similar. Além disso, os alunos irão receber um Termo de Consentimento Livre
e Esclarecido (TCLE) que explica o objetivo do experimento e no qual eles irão consentir em participar dos experimentos e
em permitir o uso dos resultados do experimento para essa pesquisa, sempre garantindo a anonimidade dos participants.

Ao final dos 45 dias do minicurso, os alunos irão receber um questionário para responder sobre a sua experiência com o SR.
Esse questionário será produzindo utilizando um subconjunto das perguntas definidas por \citeonline{pu2011user} (presente no
Anexo \ref{ane:questoes-framework}), que deverão ser traduzidas para o português para garantir o entendimento de todos os alunos.
Os resultados dos questionários serão  analisados através de métodos estatísticos para definir se existe diferença
significativa entre a Experiência do Usuário nos dois SRs (o proposto e o Baseado em Conteúdo tradicional) e, se existir,
qual deles teve o melhor desempenho.

\section{Teste piloto}

Antes do experimento ser realizado com os alunos das disciplinas de algoritmos do CCT, será realizado um teste piloto com
outros alunos da universidade que já realizaram essa disciplina. O objetivo desse teste piloto será avaliar os
instrumentos do experimento, além de permitir encontrar problemas na experiência do  usuário para serem corrigidos
antes da execução do minicurso.

Para o teste piloto, os alunos receberão o TCLE ao qual eles também serão pedidos a assinar e serão pedidos a realizar algumas
tarefas dentro do minicurso de algoritmos enquanto são observados por um pesquisador. As tarefas envolverão o acesso a
alguns conceitos, exemplos, exercícios e materiais complementares, como também a ferramenta de recomendação. Ao final, os
participantes do teste piloto irão responder ao questionário desenvolvido (o mesmo que será aplicado no experimento) e
serão entrevistados pelo pesquisador sobre problemas encontrados nos instrumentos.

Com o resultado da observação do pesquisador e do questionário respondido pelos participantes do teste piloto será possível
identificar problemas na experiência do usuário para serem corrigidos antes da execução do experimento. Já o objetivo da
entrevista será exclusivamente para aprimorar os instrumentos (i.e., TCLE e o questionário desenvolvido).

\section{Considerações sobre o capítulo}

Neste capítulo foi definido o experimento para a avaliação da proposta desse trabalho. O experimento visa avaliar a experiência
do usuário com o Sistema de Recomendação (SR) proposto no Capítulo \ref{chapter:proposta} em comparação a um SR utilizando
a abordagem Baseada em Conteúdo tradicional.

O experimento acontecerá no ambiente AdaptWeb\textsuperscript{\textregistered} através de um minicurso de algoritmos
desenvolvido no ambiente nos meses de Abril e Maio de 2018. Ao final do minicurso os alunos receberão um questionário
para responder sobre a sua experiência ao interagir com o SR, que será adaptado do conjunto de questões definidas por \citeonline{pu2011user}.

Antes da realização do experimento em si, será realizado um Teste Piloto com alunos que já realizaram a disciplina de algoritmos
com o objetivo de avaliar os instrumentos (TCLE e Questionário), além de encontrar possíveis problemas preliminares na
ferramenta de recomendação.

Ao final do experimento, espera-se encontrar a uma conclusão de se existe uma diferença na experiência do usuário ao
utilizar um SR Sensível ao Tempo, que se adapta as variações de interesse do aluno, em relação a uma abordagem tradicional
de recomendação.
