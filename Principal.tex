% Arquivo Principal para Dissertações do PPGCA - Udesc Joinville

% abnTeX2: Modelo de Trabalho Academico em conformidade com
% ABNT NBR 14724:2011: Informacao e documentacao - Trabalhos academicos -
% Apresentacao

% Adaptado com base no abnTeX2
% Por: Luís Felipe Bilecki
% E-mail: luis.bilecki@gmail.com
% ------------------------------------------------------------------------
% ------------------------------------------------------------------------
\documentclass[
	12pt,				% tamanho da fonte
	openright,			% capítulos começam em pág ímpar (insere página vazia caso preciso)
	oneside,
	a4paper,			% tamanho do papel.
	chapter=TITLE,		% títulos de capítulos convertidos em letras maiúsculas
	section=TITLE,		% títulos de seções convertidos em letras maiúsculas
	%subsection=TITLE,	% títulos de subseções convertidos em letras maiúsculas
	%subsubsection=TITLE,% títulos de subsubseções convertidos em letras maiúsculas
	% -- opções do pacote babel --
	english,			% idioma adicional para hifenização
	brazil,				% o último idioma é o principal do documento
  hidelinks
	]{abntex2}

%Pacotes prinicipais e customização
\usepackage{./Estilo/udesc}

% ---
% Dados da Capa
% ---

\titulo{SISTEMA DE RECOMENDAÇÃO QUE CONSIDERA A RECÊNCIA DOS ACESSOS DO ALUNO EM AMBIENTES VIRTUAIS DE APRENDIZAGEM}
\autor{EDUARDO JOSÉ DE BORBA}
\local{Joinville}
\instituicao{Universidade do Estado de Santa Catarina - UDESC}
\campus{Centro de Ciências Tecnológicas - CCT}
\curso{Mestrado em Computação Aplicada}
\data{2018}
\fulldata{28 de Junho de 2018}

% ---
% Folha de Rosto
% ---
\inforosto{Dissertação apresentada ao Programa de Pós-Graduação em Computação Aplicada, da Universidade do Estado de Santa
    Catarina, como requisito parcial para a obtenção do grau de Mestre em Computação Aplicada.}
\orientador{Isabela Gasparini}
\orientadorRotulo{Dra. }
\coorientador{Daniel Lichtnow}
\coorientadorRotulo{Dr. }

% ----
% Início do documento
% ----
\begin{document}
% ----
% Elementos Pré-Textuais
% ----
%!TEX root = ../Principal.tex
%Capa do Trabalho
% \imprimircapa

%Folha de Rosto
%* indica que tem ficha catalográfica
\imprimirfolhaderosto*

% ---
% Caso a Biblioteca da UDESC forneça, utilize o comando
% ---
\begin{fichacatalografica}
    \includepdf{./Partes/ficha-catalografica.pdf}
\end{fichacatalografica}

% % ---
% % Folha de Aprovação
% % ---
% % Exemplo de folha de aprovação antes da Banca. Após isso, incluia o pdf digitalizado com as assinaturas%
\includepdf{./Partes/folhadeaprovacao_final.pdf}
% \begin{folhadeaprovacao}
% 	\begin{center}
% 		{\ABNTEXchapterfont\bfseries\imprimirautor}
% 		\vspace{2em}

% 			\ABNTEXchapterfont\bfseries\imprimirtitulo

% 	\end{center}
% 		\vspace{1em}
% 		{\justify
%     Dissertação apresentada ao Programa de Pós-Graduação em Computação Aplicada, da Universidade do Estado de Santa
%     Catarina, como requisito parcial para a obtenção do grau de Mestre em Computação Aplicada.}
% 	 \vspace{2em}
% 	\noindent

% 	{\justify \bfseries Banca Examinadora}

%   \vspace{2em}

%   \noindent{Orientadora:\hfill \assinatura*{\textbf{\imprimirorientadorRotulo \imprimirorientador} \\ Universidade do Estado de Santa Catarina (UDESC)}}

%   \noindent{Coorientador:\hfill \assinatura*{\textbf{\imprimircoorientadorRotulo \imprimircoorientador} \\ Universidade Federal de Santa Marina (UFSM)}}

%   \noindent{Membros:}

% 	\noindent{\assinatura*{\textbf{Dr. Rui Jorge Tramontin Junior} \\ Universidade do Estado de Santa Catarina (UDESC)}}
%   \noindent{\assinatura*{\textbf{Dra. Rebeca Schroeder Freitas} \\ Universidade do Estado de Santa Catarina (UDESC)}}
%   \noindent{\assinatura*{\textbf{Dr. Tiago T. Primo} \\ Universidade Federal de Pelotas (UFPel)}}
%   \noindent{\assinatura*{\textbf{Dr. José Palazzo M. de Oliveira} \\ Universidade Federal do Rio Grande do Sul (UFRGS)}}

%     \vspace*{\fill}
%     \begin{center}
%     	\imprimirlocal,\,\imprimirfulldata
%     \end{center}
% \end{folhadeaprovacao}

% ---
% Dedicatória
% ---
% \begin{dedicatoria}
% Dedico este trabalho aos meus familiares, amigos, colegas e professores que me acompanharam e me deram forças nessa magnífica trajetória.
% \end{dedicatoria}

% ---
% Agradecimentos
% ---
\begin{agradecimentos}
Gostaria de agradecer primeiramente aos meus pais, José Carlos e Sandra, por todo o apoio e incetivo em minha educação
e por terem feito de mim a pessoa que sou hoje. Por participarem das decisões difíceis e por darem suporte sempre que
necessário para que eu pudesse alcançar os meus sonhos. Eles fizeram (e ainda fazem) muito mais por mim do que muitos
pais fazem pelos seus filhos e sou muito grato por isso.

Gostaria também de agradecer a minha namorada Amanda Mikos, pelo apoio e compreensão durante toda minha caminhada, desde
a graduação até a finalização desse ciclo importante da minha vida que foi o mestrado. Por todos os momentos de dúvida
e cansaço onde o seu apoio e paciência foram fundamentais. Sei que você sempre acreditou em mim, mesmo quando eu não acreditava,
e também acredito e torço muito por você.

Gostaria de agredecer imensamente aos meus orientadores Dra. Isabela Gasparini e Dr. Daniel Lichtnow por acreditarem no
meu trabalho e me guiarem durante todo o processo. Foram caminhos difíceis, mas com às suas luzes me guiando foi possível
chegar até aqui. Espero que vocês tenham apreciado esse trajeto como eu também apreciei.

Gostaria de agradecer aos então alunos do Bacharelado em Ciência da Computação Matheus Bombassaro, Caroline Sala, Luiz Engler
e Cláudia Pimental com a ajuda na execução do experimento. Bem como do então aluno do Mestrado em Ensino de Ciências,
Matemática e Tecnologias na elaboração do dicionário de palavras-chave.

Agradeço aos professores da banca Dr. Rui Jorge Tramontin Junior, Dra. Rebeca Schroeder Freitas, Dr. Tiago T. Primo e
Dr. José Palazzo M. de Oliveira que tanto na qualificação quanto na defesa do trabalho contribuiram com comentários muito
relevantes que ajudou também a guiar o desenvolvimento dessa dissertação.

Agradeço aos demais professores da UDESC - CCT que participaram de alguma forma da minha trajetória, tanto na graduação
quanto no mestrado, e também aos demais amigos e familiares que me apoiaram e torceram por mim.
\end{agradecimentos}

% ---
% Epígrafe
% ---
% \begin{epigrafe}
% ``Independentemente das circunstâncias, devemos ser sempre humildes, recatados e despidos de orgulho.''
% \\
% \par
% Dalai Lama
% \end{epigrafe}

% ---
% RESUMOS
% ---

% Português
\begin{resumo}
  Sistemas de Recomendação (SR) são ferramentas de \textit{software} que sugerem itens para os usuários de forma automatizada e
  personalizada, sem a necessidade do usuário formular uma consulta para encontrar os itens do seu interesse. Esses
  sistemas são explorados em Ambientes Virtuais de Aprendizagem (AVA) com o objetivo de reduzir alguns problemas
  existentes nesses ambientes quando a quantidade de materiais disponíveis é grande, tais como: sobrecarga cognitiva,
  dificuldade de encontrar os materiais do seu interesse e muitos materiais nunca serem utilizados. Pesquisadores da
  área argumentam que os algoritmos de SRs tradicionais não são suficientes para os AVAs, sendo necessário um nível
  maior de personalização da situação do usuário, como considerar informações do seu contexto. O objetivo desse trabalho é
  avaliar se considerar a variação temporal dos interesses do usuário em itens acessados anteriormente em SRs voltados a AVAs
  influencia o desempenho da abordagem de recomendação e a percepção dos alunos sobre as recomendações recebidas. O
  algoritmo com decaimento proposto combina a (1) similaridade do perfil do usuário
  (representado pelos materiais acessados pelo usuário) com os itens disponíveis para a recomendação com a (2) recência
  do acesso ou uso desse materiais, além da (3) informação se aquele item disponível para a recomendação já foi acessado
  ou não. A proposta leva em conta que o ritmo de estudo dos alunos pode ser diferente, portanto a recência é
  considerada em relação à sequência de itens acessados e não ao tempo absoluto (em segundos) desde o acesso. A proposta
  desse trabalho foi incorporada ao ambiente \adaptwebspace e avaliada através de um experimento utilizando um Minicurso
  de Algoritmos e Linguagem de Programação ministrado no ambiente. O algoritmo proposto foi comparado à abordagem
  Baseada em Conteúdo Tradicional utilizando uma estratégia \textit{Between Subjects}. Os resultados mostraram
  que existe diferença significativa em relação a abordagem Tradicional na Cobertura, F-measure e em uma das questões
  sobre a percepção do usuário e que a abordagem com Decaimento teve resultados melhores nessas métricas. Portanto, as
  duas hipóteses alternativas definidas para o experimento foram aceitas e indicam que considerar o Decaimento em um SR
  para AVAs influencia positivamente o desempenho do algoritmo e a percepção dos alunos sobre as recomendações.

  \vspace{\onelineskip}

  \noindent
  \textbf{Palavras-chaves}: Sistema de Recomendação; Sensível ao Tempo; Sensível ao Contexto; Decaimento; Ambiente Virtual de Aprendizagem; \adaptweb.
\end{resumo}

% Inglês
\begin{resumo}[Abstract]
 \begin{otherlanguage*}{english}
  	Recommender Systems (RS) are software tools that provide items as suggestions to users automatically and personalized to his
    interests, without the need to formulate a search argument to achieve them. This systems are applied to Virtual Learning
    Environments (VLE) aiming to reduce some drawbacks existing in these enviroments when the number of available items is
    huge, e.g., cognitive overload, difficulty finding items of user's interest or some materials never get accessed. Researchers
    in this area argues that traditional RS approaches are not enough for VLE, being required a major level of personalization
    to user's current context. This work goals is to evaluate whether the use of Decay on user's interest for past items
    on RS for VLEs influence the algorithm performance and the user's perception of the recommendations. The proposed
    algorithm combines (1) similarity between user profile (represented by the materials accessed by the user) with the items
    available to recommendation with (2) the recency of materials accessed by the user and (3) the information about if
    the item available to be recommended was accessed or not. The proposal takes into account that each learner can have
    a different study rhythm, therefore the recency considers the sequence of items accessed and not the absolute time (in
    seconds) from the access. The proposal of this work was incorporated to the \adaptwebspace environment and evaluated
    through an experiment using an Algorithms and Programming Language's course. The proposal was compared to the Content-Based traditional approach
    through using a Between Subjects strategy. The results show that there is significant difference on the Coverage,
    F-measure and in one of the questions about the user's perception and that the approach with Decay had better results
    in these metrics. Therefore, both alternative hypothesis defined for the experiment were accepted and it indicates
    that consider the Decay on a RS for VLEs influences the algorithm performance and the user's perception of the recommendations.
    \vspace{\onelineskip}

    \noindent
    \textbf{Keywords}: Recommender System; Time-Aware; Context-Aware; Decay; Virtual Learning Environment; \adaptweb.
 \end{otherlanguage*}
\end{resumo}

% ---
% Lista de Figuras
% ---
\pdfbookmark[0]{\listfigurename}{lof}
\listoffigures*
\cleardoublepage
% ---

% ---
% Lista de Tabelas
% ---
\pdfbookmark[0]{\listtablename}{lot}
\listoftables*
\cleardoublepage


% ---
% Lista de Abreviaturas e Siglas
% ---
\begin{siglas}
  \SingleSpacing
  \item[\adaptweb]  Ambiente de Ensino-Aprendizagem Adaptativo na Web
  \item[AVA]       Ambiente Virtual de Aprendizagem
  \item[CCT]       Centro de Ciências Tecnológicas
  \item[CE]        Critério de Exclusão
  \item[CI]        Critério de Inclusão
  \item[CO]        Critério Objetivo
  \item[MBA]       Mecanismos de Busca Científica
  \item[PPGCA]     Programa de Pós-Graduação em Computação Aplicada
  \item[SR]        Sistema de Recomendação
  \item[TCLE]      Termo de Consentimento Livre e Esclarecido
  \item[UDESC]     Universidade do Estado de Santa Catarina
  \item[UFRGS]     Universidade Federal do Rio Grande do Sul
  \item[UFSM]      Universidade Federal de Santa Maria
  \item[XML]       Extensible Markup Language
\end{siglas}

% ---
% inserir o sumario
% ---

\pdfbookmark[0]{\contentsname}{toc}
\tableofcontents*
\cleardoublepage
% ---

\textual

%Retira o nome do capítulo do header
\pagestyle{eudesc}
\aliaspagestyle{chapter}{eudesc}

% ---

\chapter{Introdução}\label{introducao}

Um Ambiente Virtual de Aprendizagem (AVA) é um ambiente computacional com a finalidade de integrar diversas mídias
(e.g., vídeos, apresentações, textos) e dar suporte à educação online \cite{drachsler2015panorama}. Esses ambientes, além de simularem uma sala
de aula permitindo o relacionamento professor-aluno e aluno-aluno, disponibilizam conteúdos e materiais para os usuários
poderem acessar.

Quando a quantidade de materiais disponíveis nos AVAs é muito grande, existem alguns problemas que podem acontecer. São eles:

\begin{itemize}
\item O aluno sofrer uma sobrecarga cognitiva, aumentando o esforço necessário para compreender o ambiente
e encontrar os itens de seu interesse, atrapalhando o processo de aprendizagem;
\item O aluno não encontrar um material que seja de seu interesse, devido a enorme quantidade de materiais disponíveis;
\item Parte do material disponibilizado, que poderia auxiliar os alunos no processo de aprendizagem, nunca ser
utilizado.
\end{itemize}

Com o objetivo de reduzir esses problemas, pesquisadores têm aplicado técnicas de personalização para selecionar os
itens mais adequados para cada estudante, considerando o seu conhecimento, objetivos, preferências e necessidades
\cite{brusilovsky1998methods}. Os Sistemas de Recomendação são um alternativa para reduzir esses problemas, sugerindo
itens para o usuários utilizando informações sobre seus interesses e sobre os itens disponíveis \cite{adomavicius2005toward}.

Porém, pesquisadores da área argumentam que no domínio educacional os SRs tradicionais (aqueles que consideram apenas
informações sobre as interações do usuário com os itens para recomendar) não são suficientes \cite{verbert2012context, drachsler2015panorama}.
\citeonline{verbert2012context} afirmam que nessa área é necessário um nível maior de personalização, como utilizar informações
do contexto do usuário na recomendação.

Apesar de existir uma grande quantidade de trabalhos utilizando o contexto em SRs no domínio educacional, como pode ser visto
em \citeonline{verbert2012context} e \citeonline{drachsler2015panorama}, pouco foi encontrado da aplicação do contexto
temporal nesse domínio \cite{de2017time}. O contexto temporal é relevante, pois leva em consideração a variação dos
interesses do usuário com o passar do tempo. Além disso, os SR Sensíveis ao Tempo demonstraram bons resultados em outros
domínios de aplicação \cite{campos2014time}.

\section{Problema}

Como dito anteriormente, foram encontrados poucos trabalhos sobre o uso do contexto temporal em SRs para AVAs. Além disso,
nos trabalhos encontrados, as propostas não foram avaliadas em ambientes reais de uso, não sendo possível avaliar o
impacto dos SRs Sensíveis ao Tempo nesse domínio. Assim, a pergunta a ser respondida por este trabalho
é: "\textbf{Considerar a variação temporal dos interesses do aluno por conteúdos acessados anteriormente influencia o desempenho
da abordagem de recomendação e a percepção dos alunos sobre as recomendações recebidas?}".

\section{Objetivos}

Foram definidos objetivos geral e específicos para orientar o processo de pesquisa desse trabalho buscando responder a questão
de pesquisa definida acima.

\subsection{Objetivo Geral}

Avaliar em um ambiente real de uso se considerar a variação temporal dos interesses dos alunos em sistemas de recomendação
voltados a AVAs influencia o desempenho da abordagem de recomendação e a percepção dos alunos sobre as recomendações recebidas.

\subsection{Objetivos Específicos}

\begin{itemize}
\item Identificar as formas de utilizar os aspectos temporais do contexto do usuário em um algoritmo de recomendação;
\item Definir como o aspecto temporal pode ser utilizado em sistemas de recomendação para AVAs;
\item Conceber um algoritmo de recomendação considerando o Decaimento do interesse dos alunos por itens acessados
anteriormente no contexto educacional;
\item Realizar um experimento com usuários reais.
\end{itemize}

\section{Escopo}

Esse trabalho não considera outras dimensões do contexto além do tempo na recomendação, e dentro do uso do contexto
temporal apenas a categoria de Decaimento foi aplicada neste trabalho, que está relacionada à perda de
interesse por itens acessados anteriormente. Além disso, a única abordagem de recomendação utilizada é a Baseada em
Conteúdo, apesar de as categorias de Sistemas de Recomendação Sensíveis ao Tempo poderem ser aplicadas em quaisquer
abordagens de recomendação. A avaliação do Sistema de Recomendação proposto é feita apenas em um ambiente educacional,
mesmo sendo possível aplicá-lo em outros domínios de aplicação. Não é avaliado, nesse trabalho, o impacto da proposta
na aprendizagem dos alunos, pois esse tipo de avaliação
exigiria algum método confiável de medir a aprendizagem, algo que é ainda muito discutido por pedagogos \cite{luckesi2014avaliaccao}.

\section{Metodologia}

A pesquisa desse trabalho é de natureza Aplicada, pois busca gerar conhecimentos através da implementação e experimentação
de SRs em um ambiente real de uso. A abordagem do problema deste trabalho é tanto qualitativa, através do questionário
aplicado para extrair a percepção dos alunos, quanto quantitativa, através das métricas calculadas para medir o desempenho
dos algoritmos. Os objetivos dessa pesquisa têm caráter Explicativo, visando identificar fatores que influenciam o
desempenho de um algoritmo de recomendação e a percepção dos alunos sobre as recomendações. O procedimento utilizado
para o desenvolvimento dessa pesquisa é Experimental, onde os objetos de estudos são os dados coletados da interação dos
alunos com o SR e a percepção dos alunos sobre a qualidade das recomendações e a variável é o algoritmo de recomendação
utilizado.

\section{Estrutura}

Este trabalho está estruturado da seguinte forma: o Capítulo \ref{chapter:fundamentacao-teorica} conceitua os Sistemas de
Recomendação (SR), as suas abordagens, as formas de avaliação e a apresentação das recomendações; o Capítulo \ref{chapter:trabalhos-relacionados}
descreve os trabalhos relacionados que utilizam a categoria Decaimento nas recomendações e compara com a proposta desse
trabalho; o Capítulo \ref{chapter:proposta} apresenta em detalhe a proposta desse trabalho; o Capítulo \ref{chapter:experimento}
apresenta o experimento utilizado para avaliação dessa proposta e os resultados do experimento. Por último, o Capítulo \ref{chapter:conclusoes} apresenta
as considerações finais deste trabalho.



\chapter{Fundamentação Teórica}\label{chapter:fundamentacao-teorica}

Nesse capítulo são apresentados os principais conceitos relacionados à proposta desse trabalho. Primeiramente são apresentados
os Sistemas de Recomendação e as suas abordagens tradicionais, seguidos pelos Sistemas de Recomendação Sensíveis ao Contexto
e, mais especificamente, os Sistemas de Recomendação Sensíveis ao Tempo. Em seguida, são apresentados aspectos relacionados ao
projeto de interfaces de recomendação e formas de avaliação de Sistemas de Recomendação.

\section{Sistemas de Recomendação}

Sistemas de Recomendação (SRs) se tornaram uma importante área de pesquisa a partir dos anos 90, quando começaram a
surgir os primeiros trabalhos na área de filtragem colaborativa \cite{adomavicius2005toward}. Os SRs são ferramentas
computacionais que proveem sugestões de itens personalizadas aos usuários \cite{ricci2011introduction}. Isso significa
que o usuário recebe como recomendação um conjunto diferente de itens de acordo com as suas preferências e necessidades.
Nos últimos anos, o interesse na aplicação de SRs tem crescido fortemente \cite{adomavicius2005toward, beel2016towards}.
Exemplos dessas aplicações são: recomendação de Livros, CDs, DVDs, etc., em sites de \textit{e-commerce} como Amazon e EBAY;
recomendações de filmes em sites como MovieLens e Netflix; recomendação de músicas em sites de \textit{streaming} como Last.fm ou
Spotify; recomendação de amigos ou de postagens em redes sociais como Facebook ou Twitter; entre outras.

SRs estão representados formalmente na Equação \ref{eq:sr-tradicional}.

\begin{equation}
  F: U \times I \rightarrow R
  \label{eq:sr-tradicional}
\end{equation}

Onde $F$ é a função que busca prever o interesse do usuário pelos itens existentes, $U$ representa o conjunto dos usuários,
$I$ representa o conjunto dos itens e $R$ representa a lista ordenada dos itens pelo interesse previsto para o usuário ativo
(o usuário que irá receber a recomendação). O objetivo do SR então é conseguir prever de maneira mais correta, com as
informações disponíveis, os itens que serão de maior interesse do usuário.

Existem duas formas de capturar os interesses do usuário pelos itens acessados dentro do sistema: (1) Explícita, na
qual o usuário indica explicitamente o seu interesse pelo item que acabou de acessar, geralmente com uma nota 1 a 5 ou
apenas uma indicação de interesse positivo/negativo para o item; (2) Implícita, na qual o usuário não precisa indicar o
seu interesse pelo item, essa informação é capturada implicitamente através do seu comportamento e das suas interações
dentro do sistema.

Os SRs podem ser classificados de acordo com a forma como as recomendações são realizadas (abordagem). As principais
abordagens citadas na literatura são \cite{torres2004personalizaccao, adomavicius2005toward, ricci2011introduction, bobadilla2013recommender}:
Baseada em Conteúdo, Filtragem Colaborativa, Baseada em Conhecimento e Híbrida. Nas subseções a seguir são descritas
cada uma dessas abordagens.

\subsection{Baseada em Conteúdo}\label{subsection:baseada-em-conteudo}

Segundo \citeonline{ricci2011introduction}, essa é uma abordagem na qual o usuário recebe recomendações de itens
similares aos que se interessou no passado. Consiste em avaliar a semelhança entre um item e os interesses do usuário.
Os métodos dessa abordagem tentam prever o grau de utilidade de um item para um usuário com base na utilidade que o
usuário determinou para itens similares a este \cite{adomavicius2005toward}.

A abordagem Baseada em Conteúdo tem suas raízes na Recuperação da Informação \cite{adomavicius2005toward}. Na
abordagem Baseada em Conteúdo, tem-se um conjunto de atributos descrevendo um item e um conjunto de atributos
descrevendo os gostos e preferências do usuário. A descrição de um item frequentemente é realizada através de
palavras-chave definidas automaticamente por meio de algoritmos usados na área de Recuperação da Informação
\cite{adomavicius2005toward}. Já a descrição das preferências do usuário, como dito anteriormente, pode ser capturada de duas formas: implícita,
através do seu comportamento no ambiente e de itens que acessou; ou explícita, onde o usuário informa suas preferências
ao sistema, por exemplo, respondendo a questionários \cite{adomavicius2005toward}. Dessa forma, os SRs de itens
textuais (e.g., documentos) são os que mais utilizam a abordagem Baseada em Conteúdo, devido à facilidade da aplicação
das técnicas de Recuperação da Informação nesse tipo de item.

Dentro da área de Recuperação da Informação, uma forma de medir a similaridade de itens em um SR é o Cosseno. O cálculo
da similaridade por Cosseno foi definido por Salton nos anos 60 \cite{salton1964document}. Nessa técnica, cada documento
é representado por um vetor de termos $\vec{d_J} = (w_{1,j}, w_{2,j}, ..., w_{t,j})$. Os vetores são dispostos em um
espaço vetorial de $t$ dimensões, onde $t$ é o número de termos, e documentos próximos nesse espaço são considerados
semelhantes. Para verificar essa proximidade utiliza-se a Equação \ref{eq:cosseno} \cite{christopher2008introduction}.

\begin{equation}
  sim(d_1, d_2) = \frac{\sum_{i=1}^{t}{w_{1,i} \times w_{2,i}}}{\sqrt{\sum_{i=1}^{t}{w_{1,i}}^2 \sum_{i=1}^{t}{w_{2,i}}^2}}
  \label{eq:cosseno}
\end{equation}


Onde: $sim(d_1, d_2)$ é o resultado da distância dos vetores, variando de $[0,1]$; $w_{1,i}$ é o termo presente na
posição $i$ do item $1$; $w_{2,i}$ é o termo presente na posição $i$ do item $2$. Por exemplo, se tivermos três vetores: $u = \{1,1\}$
representando o usuário, $i_1 = \{0, 1\}$ e $i_2 = \{1, 1\}$ representando itens. Os vetores com $1$ na primeira posição
indicam que o item ou usuário que estão representando possuem o primeiro termo, enquanto o $0$ indica que não possuem. O mesmo
funciona para a segunda posição em diante. Ao calcular a similaridade entre esses itens, temos $sim(u, i_1) \approx 0.71$
e $sim(u, i_2) = 1$, identificando que o item representado por $i_2$ é mais similar às preferências do usuário $u$.

Outra técnica de Recuperação da Informação é o TF-IDF (\textit{Term-Frequency Inverse Document Frequency}). Essa técnica
é utilizada para identificar termos importantes em um documento \cite{christopher2008introduction} e pode ser utilizada
para a descoberta das palavras-chave que descrevem um item. É utilizada a fórmula da Equação \ref{eq:tf-idf} para o cálculo dos
pesos de cada termo do documento \cite{christopher2008introduction}.

\begin{equation}
  tf\hbox{-}idf_{t,d} = tf_{t,d} \times idf_{t,d}
  \label{eq:tf-idf}
\end{equation}

Onde: $tf\hbox{-}idf_{t,d}$ representa o peso do termo $t$ no documento $d$; $tf_{t,d}$ é o número de vezes que o termo
$t$ aparece no documento $d$; e $idf_{t,d}$ representa o \textit{Inverse Document Frequency} do termo $t$, sendo o responsável por
identificar termos que aparecem em muitos documentos diferentes \cite{christopher2008introduction}. Os termos que aparecem em muitos
documentos tendem a perder sua importância. O $idf_{t,d}$ é calculado através da Equação \ref{eq:idf} \cite{christopher2008introduction}.

\begin{equation}
  idf_{t,d} = \log(\frac{N}{d_f})
  \label{eq:idf}
\end{equation}

Onde: $N$ é o número total de documentos em uma coleção; e $d_f$ é o número de documentos onde aparece o termo $t$.

A principal vantagem da abordagem Baseada em Conteúdo é não necessitar da opinião de outros usuários para a recomendação
\cite{ricci2011introduction}. As principais desvantagens são: a Partida Fria, em que o sistema não terá informações
suficientes sobre os usuários novos para realizar uma boa recomendação; e a Superespecialização, na qual o
usuário recebe sempre itens semelhantes aos que já viu \cite{lops2011content}.

\subsection{Filtragem Colaborativa}

Nessa abordagem o usuário receberá como recomendação itens que usuários com os mesmos interesses que ele se
interessaram no passado, ou seja, é a automatização do processo de ''boca-a-boca'' \cite{jannach2010recommender}. A
técnica de Filtragem Colaborativa tenta prever a utilidade  do item para o usuário, com base na utilidade do mesmo
produto para um conjunto de usuários  possuidores de características semelhantes às suas \cite{jannach2010recommender}.

Existem duas variações básicas da Filtragem Colaborativa: Usuário-Usuário, onde a similaridade entre os usuários é analisada;
Item-Item, onde a similaridade entre itens a serem recomendados é analisada \cite{jannach2010recommender}.

Para \citeonline{torres2004personalizaccao}, que considera a variação Usuário-Usuário, a Filtragem Colaborativa ocorre,
resumidamente, da seguinte forma:

\begin{enumerate}
\item As opiniões das pessoas sobre itens são armazenadas;
\item Baseado nessas opiniões, pessoas com perfil semelhantes (vizinhos) são agrupados;
\item Itens bem avaliados pelos vizinhos são recomendados ao usuário.
\end{enumerate}

Existem duas estratégias para medir a similaridade entre os usuários: Coeficiente de Pearson e Cosseno
\cite{torres2004personalizaccao}. Levando em consideração que os usuários são representados pelas notas que deram aos
itens, utiliza-se um cálculo matemático para medir a similaridade entre o perfil dos usuários
\cite{torres2004personalizaccao}.

O Coeficiente de Pearson é um coeficiente bastante utilizado em modelos econômicos e mede a força do relacionamento
de duas variáveis \cite{torres2004personalizaccao}. Esse coeficiente varia no intervalo $[-1, 1]$, sendo $-1$ indica
ausência de correlação e $+1$ indica forte correlação. O cálculo é então feito de acordo com a Equação \ref{eq:pearson}
\cite{torres2004personalizaccao}.

\begin{equation}
  w_{a,u} = \frac{\sum_{i=1}^{m}(r_{a,i} - \overline{r_a})*(r_{u,i} - \overline{r_u}))}{\sqrt{\sum_{i=1}^{m}(r_{a,i} - \overline{r_a})^2} \sqrt{\sum_{i=1}^{m}(r_{u,i} - \overline{r_u})^2}}
  \label{eq:pearson}
\end{equation}

Na fórmula, $w_{a,u}$ representa a correlação entre o usuário $u$ e um determinado usuário $a$, onde: $r_{a,i}$ é a avaliação
do usuário $a$ para o item $i$; $\overline{r_a}$ é a média de todas as avaliações do usuário $a$; $r_{u,i}$ é a avaliação do usuário
$u$ para o item $i$; $\overline{r_u}$ é a média de todas as avaliações do usuário $u$. A similaridade é calculada apenas com
itens que os dois usuários avaliaram.

Com o aumento da quantidade de usuários e de itens, torna-se um desafio para a Filtragem Colaborativa Usuário-Usuário
realizar uma recomendação, principalmente pela dificuldade de identificar a vizinhança com tantos usuários
\cite{jannach2010recommender}. A estratégia Item-Item é uma solução para ser utilizada nesse contexto, permitindo a
computação das similaridades a acontecer \textit{off-line} (JANNACH et al., 2011). A ideia principal da estratégia Item-Item
é prever a nota que o usuário daria para um item com base nas notas que ele deu para itens semelhantes àquele. Para
essa estratégia, o cálculo da similaridade pelo Cosseno, semelhante ao já citado, é uma métrica padrão e a que
apresenta os melhores resultados \cite{jannach2010recommender}. Esse cálculo da similaridade, ao invés de comparar
as notas de cada um dos usuários, considera vetores com as notas de cada item para identificar essa similaridade.

As pessoas que apresentaram preferências similares no passado tendem a concordar no futuro \cite{ricci2011introduction}.
Por isso essa abordagem tende a realizar recomendações que serão bem aceitas pelos usuários.

Como essa abordagem não considera a descrição dos itens e sim as notas desses, uma vantagem dessa abordagem é que as
recomendações realizadas podem ser bastante interessantes e inesperadas ao usuário \cite{ricci2011introduction}.

Por outro lado, a abordagem colaborativa também possui a desvantagem da Partida Fria. Existem dois tipos de Partida Fria
nessa abordagem \cite{adomavicius2005toward}: a Partida Fria do Usuário e do Item. A Partida Fria do Usuário é a dificuldade
que o sistema encontra para recomendar um item para um usuário que não avaliou nenhum item ainda. A Partida Fria do Item
ocorre para um novo item no sistema, que não será recomendado enquanto não for avaliado por algum usuário.

Além disso, outras desvantagens são \cite{adomavicius2005toward}:

\begin{itemize}
\item Esparsidade: quanto maior a quantidade de usuários e de itens disponíveis, mais esparsa ficará a tabela com as
notas dos usuários e mais difícil será realizar as comparações. Pode ser difícil prever com precisão usuários com os
mesmos gostos, pois cada usuário poderá avaliar conjuntos muito diferentes de itens;
\item Necessidade de uma comunidade de usuários ativa: para essa abordagem, é necessário ter uma grande quantidade de
usuários ativos no sistema ao mesmo. No caso de um sistema com poucos usuários, pode acontecer também a esparsidade,
pois os usuários acessam e avaliam itens diferentes e não é possível calcular a similaridade entre eles;
\item Ovelha Cinza: para usuários que possuem gostos distintos demais, torna-se um desafio realizar recomendações
interessantes para ele. O principal motivo é que o sistema não consegue definir outros usuários semelhantes a ele para
gerar recomendações;
\item Escalabilidade: com o aumento do número de usuários, o custo computacional se torna alto;
\item Confiabilidade: essa abordagem é dependente da confiabilidade das avaliações realizadas pelos usuários, se estas forem
realizadas de forma incorreta irão diminuir a eficiência da abordagem. Outra coisa a ser considerada é a reputação dos
usuários: usuários com maior reputação podem ter suas avaliações mais consideradas (maior peso) que as avaliações de
outros usuários.
\end{itemize}

\subsection{Baseada em Conhecimento}

A abordagem Baseada em Conhecimento recomenda itens aos usuários com base no conhecimento que o sistema possui sobre
como características de um item se encaixam nas necessidades de um usuário e o quão útil esse item será
\cite{ricci2011introduction}. Geralmente são utilizadas formas de representar esse conhecimento que sejam de fácil interpretação
por computadores, como Ontologias, por exemplo\cite{burke2002hybrid}. O sistema então recebe como entrada a descrição das
necessidades e interesses do usuário, e o papel do sistema é realizar uma combinação entre essas necessidades e os itens.

Os SRs Baseados em Caso (\textit{Case-Based}) são um exemplo de SR da abordagem Baseada em Conhecimento. Nesse sistema, uma
função de similaridade estima o quanto a necessidade de um usuário (descrição de um problema) combina com uma
determinada recomendação (solução do problema) \cite{ricci2011introduction}. Essa similaridade é o grau de utilidade
da recomendação.

Outro exemplo da abordagem Baseada em Conhecimento são os SR Baseados em Restrição. Nessa abordagem, os itens que não
atendam a certas restrições são automaticamente eliminados dos itens a serem recomendados. Segundo
\citeonline{ricci2011introduction}, a principal diferença entre um SR Baseado em Caso e um Baseado em Restrição está
no fato de o Baseado em Caso considerar a similaridade entre as necessidades do usuário e o item enquanto a baseada
em restrições possui regras específicas para tratar cada uma das necessidades do usuário.

A abordagem Baseada em Conhecimento costuma funcionar melhor que outras (e.g., Filtragem Colaborativa ou Baseada em
Conteúdo) no início do desenvolvimento, porém se ela não for equipada com a capacidade de aprender mais sobre o usuário,
ela será rapidamente ultrapassada por métodos mais simples que exploram a interação do usuário com o sistema
\cite{ricci2011introduction}. Essa abordagem é empregada em conjunto com as outras abordagens com o objetivo de aprimorar
a qualidade das recomendações \cite{burke2002hybrid}.

\subsection{Híbrida}

Essa abordagem utiliza uma combinação das diversas abordagens para recomendar itens ao usuário. O objetivo é reunir as
vantagens das abordagens e tentar eliminar suas desvantagens \cite{burke2002hybrid}. Alguns exemplos de algoritmos que
utilizam a abordagem híbrida foram dados por \citeonline{burke2002hybrid}:

\begin{itemize}
\item \textit{Weighted}: a recomendação é o resultado da execução das abordagens de recomendação em conjunto. Essas abordagens podem
ser executadas linearmente, uma após a outra, para definir os melhores itens a serem recomendados, ou cada abordagem
pode ter pesos diferentes, tornando o resultado de um mais importante que o resultado do outro.
\item \textit{Switching}: ocorre uma alternância entre as abordagens, em certos momentos uma delas é utilizada e em outros
momentos outra é utilizada. O sistema deverá possuir alguns critérios para definir qual abordagem irá utilizar.
\item \textit{Mixed}: as mencionadas são utilizadas separadamente e os resultados aparecem em um mesmo ranking. Esse tipo de
abordagem é utilizado quando se deseja realizar um grande número de recomendações diferentes simultaneamente.
\item \textit{Feature combination}: considera as informações da colaboração como uma característica e utiliza a abordagem
Baseada em Conteúdo para realizar a recomendação.
\item \textit{Cascade}: uma abordagem é utilizada primeiro para gerar um ranking e outra abordagem refina o resultado dado
por esta.
\item \textit{Feature augmentation}: uma abordagem é utiliza para produzir um ranking ou uma classificação para cada item e o
resultado será considerado na execução de outra abordagem.
\end{itemize}

\section{Sistemas de Recomendação Sensíveis ao Contexto}\label{section:sr-sensivel-contexto}

SRs tradicionais consideram apenas as relações entre os usuários e os itens para recomendar, mas não consideram o
contexto em que os usuários estão. De acordo com \citeonline{dey2001understanding}, o contexto é qualquer informação
que pode ser usada para caracterizar a situação de uma entidade. As principais entidades em SRs são o usuário que
irá receber uma recomendação e os itens que serão recomendados.

SRs Sensíveis ao Contexto estão formalmente definidos na Equação \ref{eq:context-aware}.

\begin{equation}
  F: U \times I \times C \rightarrow R
  \label{eq:context-aware}
\end{equation}

Onde $F$ é a função que prediz o interesse em um item ainda não utilizado pelo usuário, $U$ representa o conjunto do
usuários, $I$ representa o o conjunto dos itens, $C$ representa o contexto da interação e $R$ representa o conjunto de itens
ordenado pelo interesse previsto do usuário para os itens disponíveis.

Vários autores definem conjuntos de dimensões que podem representar o contexto
\cite{schilit1994context, chen2000survey, zimmermann2007operational} e que diferem pouco entre si. Nesse trabalho,
adotou-se a definição de \citeonline{schmidt1999there}, que é uma das mais completas encontradas. Os autores definem essas
sete dimensões para representar o contexto \cite{schmidt1999there}:

\begin{itemize}
\item Informações sobre o usuário, e.g., hábitos do usuário, estado emocional, etc.;
\item Ambiente social do usuário, e.g., co-localização com outros usuários, interação em redes sociais, etc.;
\item Tarefas do usuários, e.g., objetivos gerais, se é uma tarefa definida previamente (pelo professor, por exemplo)
ou aleatória, etc.;
\item Localização, e.g., posição absoluta, se o usuário está em casa, no trabalho ou na universidade, etc.;
\item Condições do ambiente, e.g., barulho, luminosidade, etc.;
\item Infraestrutura, e.g., velocidade da internet, tipo de dispositivo utilizado, etc.;
\item Tempo, e.g., \textit{timestamp} de ocorrência de uma interação, dia da semana no qual o usuário pede uma recomendação, etc.
\end{itemize}

Sobre a aplicação do contexto em SRs, \citeonline{adomavicius2011context} definem três paradigmas de uso das dimensões
do contexto no processo de recomendação:

\begin{itemize}
\item Pré-Filtragem Contextual: o contexto filtra os dados que representam o usuário e esses dados servem
como entrada para um algoritmo tradicional de recomendação;
\item Pós-Filtragem Contextual: uma abordagem tradicional de recomendação é utilizada para gerar uma lista de
itens a serem recomendados e depois esses itens são filtrados de acordo com o contexto do usuário;
\item Modelagem Contextual: o contexto é aplicado diretamente no algoritmo de recomendação, gerando um
algoritmo diferente dos tradicionais.
\end{itemize}

\citeonline{verbert2012context} dizem que, em ambientes educacionais, as abordagens tradicionais de SRs não são
suficientes para recomendar de forma apropriada para os estudantes, porque esse domínio oferece algumas características
específicas que não são cobertas por essas abordagens. Por exemplo, é muito mais perigoso recomendar um item ruim para
um estudante, que pode desmotivá-lo nos seus estudos, do que recomendar um produto ruim em um site de \textit{e-commerce}.
De acordo com \citeonline{verbert2012context} esse domínio requer um nível maior de personalização.

Aplicar algumas dimensões do contexto é uma alternativa para melhorar a personalização em ambientes educacionais,
recomendando materiais adequados para a situação atual do usuário. Por exemplo, considerar o histórico de aprendizagem
do aluno, as condições do ambiente e a acessibilidade dos recursos \cite{verbert2012context}.

Na Seção \ref{section:sr-sensivel-tempo} é apresentado um tipo específico de SRs Sensíveis ao Contexto que utilizam a dimensão temporal para
recomendar, chamados de SRs Sensíveis ao Tempo. Esse tipo de SR pode também aplicar outras dimensões do contexto em
conjunto à questão temporal.

\section{Sistemas de Recomendação Sensíveis ao Tempo}\label{section:sr-sensivel-tempo}

Dentre as dimensões do contexto citadas na seção \ref{section:sr-sensivel-contexto}, o tempo tem uma vantagem de ser
fácil de capturar, considerando que praticamente todos os dispositivos têm um relógio que pode capturar o tempo no qual
alguma interação ocorreu. Além disso, trabalhos na área demonstraram que o contexto temporal tem potencial para melhorar
a qualidade das recomendações \cite{campos2014time}. Esse tipo de SR é chamado de SR Sensível ao Tempo.

SRs Sensíveis ao Tempo estão formalmente definidos na Equação \ref{eq:time-aware}.

\begin{equation}
  F: U \times I \times T \rightarrow R
  \label{eq:time-aware}
\end{equation}

Onde $F$ é a função que prediz o interesse do usuário por item ainda não utilizado, $U$ representa o conjunto de usuários,
$I$ representa o conjunto de itens, $T$ representa o contexto temporal e $R$ representa o conjunto de itens ordenado pelo
interesse previsto do usuário para os itens disponíveis.

De acordo com o dicionário \citeonline{michaelis2011disponivel}, o tempo é um  ''Período de momentos, de horas, de dias,
de semanas, de meses, de anos, etc. no qual os eventos se sucedem, dando-se a noção de presente, passado e futuro''.
Com essa informação é possível para um sistema computacional estabelecer uma ordem para os eventos que ocorrem.

O Tempo pode ser representado de uma variável contínua ou categórica. A representação contínua utiliza o exato momento
em que os itens foram consumidos/avaliados \cite{campos2014time}, por exemplo: ''8 de outubro de 2017, 16:15:03''.
Enquanto na representação categórica as variáveis são calculadas em relação a períodos de interesse \cite{campos2014time},
e.g., Dias da semana: {Domingo, Segunda, Terça, ...} ou Estações do ano: {Primavera, Verão, Outono, Inverno}. Além
disso, o tempo pode ser representado por diferentes unidades de tempo, e.g., segundos, minutos, horas, meses, anos,
etc., e as unidades de tempo são hierárquicas, e.g., um dia tem 24 horas, uma hora tem 60 minutos e 1 minuto tem 60
segundos.

Em \citeonline{de2017time} um mapeamento sistemático foi conduzido sobre os SR Sensíveis ao Tempo utilizando a metodologia de
\citeonline{petersen2008systematic}. Esse mapeamento sistemático considerou artigos de qualquer domínio de aplicação,
e não apenas trabalhos na área educacional. A principal questão de pesquisa desse mapeamento é ''Como o contexto
temporal é utilizado em SRs Sensíveis ao Contexto?''. Para responder a essa questão de pesquisa principal, três questões
de pesquisa secundárias foram definidas, são elas ''Como os algoritmos de recomendação utilizam o tempo?''; ''Qual é a
diferença entre o uso do tempo em diferentes domínios de aplicação?''; e ''Que outras dimensões são utilizadas
juntamente com o contexto temporal?''.

Com base nas questões de pesquisa que buscamos responder, foi definida o seguinte argumento de busca
\textit{(time-aware OR context-aware) AND (''recommender system'').}, que tem o objetivo de encontrar artigos sobre
SRs Sensíveis ao Tempo ou SRs Sensíveis ao Contexto que utilizem o tempo como uma de suas dimensões.
O argumento de busca foi aplicado em três Mecanismos de Busca Acadêmica (MBA): \textit{IEEE Xplorer}, \textit{Scopus} e
\textit{Springer Link}. Esses MBAs foram os escolhidos por terem um grande acervo
de artigos da área da Computação e possuírem mecanismos de busca e de filtro necessários \cite{de2017time}. A busca
foi realizado procurando pelos argumentos de busca no Título, Resumo ou Palavras-chave.

Três Critérios Objetivos (CO) foram definidos para filtrar artigos mais relevantes para a pesquisa. O primeiro CO é que os
artigos tenham sido publicados nos 10 anos anteriores à realização do mapeamento, i.e., de 2006 à 2016. O segundo CO é
que apenas artigos que estiverem disponíveis para o download completo foram utilizados e o último CO é que apenas artigos
em inglês foram considerados. Após o processo de filtragem utilizando os COs, 556 foram baixados e analisados individualmente
conforme os seguintes Critérios de Inclusão (CI) e Exclusão (CE):

\begin{itemize}
\item CI1: Incluir apenas artigos que tenha como objetivo descrever uma estratégia (i.e., algoritmo, \textit{framework},
método, modelo, etc.) para recomendar.
\item CE1: Excluir artigos que não utilizem o tempo para recomendar ou não expliquem detalhadamente como o tempo é utilizado.
\item CE2: Excluir artigos duplicados ou artigos diferentes relativos ao mesmo trabalho.
\end{itemize}

Após a última filtragem, 88 trabalhos fizeram parte do estudo e foram considerados para responder as
questões de pesquisa. Entre os resultados do mapeamento sistemático desenvolvido, o principal
foi a definição de sete categorias de SRs Sensíveis ao Tempo. Essa categorização foi feita a partir do agrupamento
dos artigos que utilizam o tempo de forma semelhante. A partir disso, foi possível identificar as sete principais
formas de utilizar o tempo nos algoritmos de recomendação: \textit{Restriction}, \textit{Micro-profiles}, \textit{Bias},
\textit{Decay}, \textit{Time Rating}, \textit{Novelty} e \textit{Sequence}. Essas categorias são descritas em detalhes
nas Subseções \ref{subsection:restriction}-\ref{subsection:sequence}.

Além disso, os resultados mostraram que as aplicações do Tempo mais comuns são através de \textit{Restriction},
\textit{Micro-profiles} e \textit{Bias}. O formato do tempo mais utilizado é o contínuo e a dimensão do contexto mais
utilizada em conjunto com o tempo é a localização. Foi observado também que em cada domínio de aplicação o Tempo costuma ser
aplicado de forma diferente, e.g., na recomendação de Pontos de Interesse o uso de \textit{Restriction} é mais comum enquanto
na recomendação de Multimídia o Tempo é mais aplicadoa através dos \textit{Micro-profiles}. Dentre os 88 artigos analisados,
apenas quatro são da área educacional e utilizam \textit{Decay} (dois artigos) e \textit{Restriction} (dois artigos).

\subsection{Restriction}\label{subsection:restriction}

Na categoria \textit{Restriction}, o tempo é utilizado para restringir que itens serão utilizados. Isso significa que o SR
compara variáveis de tempo relacionadas aos itens e ao usuário para restringir quais itens irão aparecer na lista de
recomendações. Existe pelo menos duas formas de restrição para se utilizar: (1) o SR compara o tempo disponível pelo
usuário com o tempo necessário para consumir um determinado item, e.g., a duração dos filmes que serão recomendados e
o tempo que o usuário tem até o seu próximo compromisso; (2) o SR compara o tempo atual (data e hora) com o horário de
funcionamento dos itens que serão recomendados, e.g., na recomendação de restaurantes onde só faz sentido recomendar
locais que estejam servindo no momento.

\subsection{Micro-profile}

Na categoria \textit{Micro-Profile}, o usuário possui perfis distintos para cada período de tempo. Nessa categoria, o tempo
deve ser utilizado de forma categórica, onde as categorias que serão utilizadas dependem da aplicação onde for aplicada.
É possível, por exemplo, que o usuário possua um perfil para dias da semana e outro perfil para finais de semana, ou
então um perfil para a manhã, outro para a tarde e outro para a noite. O objetivo é que as recomendações serão
realizadas considerando apenas as interações do usuário que aconteceram no mesmo contexto temporal em que ele está no
momento, e.g., recomendar programas de TV para o usuário em um domingo a noite considerando apenas quais programas ele
costuma acessar em um domingo a noite.

\subsection{Bias}

Na categoria \textit{Bias}, o tempo é utilizado para agregar informação na matriz Usuários x Itens normalmente utllizada pela
Filtragem Colaborativa. Essa matriz é comumente utilizada com apenas duas dimensões que são os Usuários e os Itens e os
valores dessa matriz são as notas dadas pelos usuários para os itens. Ao incorporar o tempo nessa matriz, é possível
realizar uma comparação mais precisa entre os usuários do sistema e assim prever o interesse do usuário ativo para os
itens ainda não acessados. Dessa forma, usuários que avaliaram os mesmos itens com notas semelhantes e em contextos
temporais semelhantes serão considerados vizinhos do usuário ativo e o algoritmo de recomendação tem uma maior chance
de acertar nos interesses do usuário.

\subsection{Decay}\label{section:decay}

Na categoria \textit{Decay}, o tempo é utilizado como um fator de decaimento na importância das interações do usuário, i.e.,
interações (itens consumidos, avaliações, etc.) mais antigas têm um peso menor para o algoritmo de recomendação do que
as interações mais atuais. Os algoritmos dessa categoria consideram que o interesse do usuário varia com o tempo e é
importante considerar que os interesses mais atuais do usuário representam melhor o seu perfil do que interesses mais
antigos. É importante notar que as interações antigas não são ignoradas pelo algoritmo de recomendação com \textit{Decay}, é
apenas dado um peso menor para essas interações.

\subsection{Time Rating}

Na categoria \textit{Time Rating}, o tempo é considerado pelo SR para inferir as preferências do usuário. Nessa categoria, o SR
utiliza uma estratégia implícita para capturar o interesse do usuário que considera o tempo que o usuário passou em
determinado item. A categoria toma como princípio que itens no qual o usuário passou pouco tempo não são do seu
interesse, enquanto itens em que ele passou mais tempo indicam os seus interesses. Essa forma de capturar é interessante
pois o usuário não precisa explicitamente dar notas ao itens, dessa forma é possível capturar um feedback do usuário
para todos os itens acessados por ele.

\subsection{Novelty}

Na categoria \textit{Novelty}, o SR considera que itens mais novos serão mais relavantes para os usuários do que itens mais
antigos. Nessa catoria, existem pelo menos duas estratégias que podem ser utilizadas: (1) o SR possui uma idade limite
definida (por exemplo, duas semanas) e itens que sejam mais velhos que isso serão retirados da lista de recomendação;
(2) o SR não ignora itens antigos, porém os itens novos possuem um peso maior e, se dois itens similares estiverem para
ser recomendados, o mais novo é o escolhido mesmo que o mais antigo esteja mais de acordo com o perfil do usuário.
Essa categoria é mais comum em domínios onde novos itens tendem a ser mais relevantes que itens antigos, e.g., redes
sociais, notícias, etc.

\subsection{Sequence}\label{subsection:sequence}

Na categoria \textit{Sequence}, o SR observa itens que são geralmente consumidos juntos em uma determinada ordem e utiliza essa
informação para recomendar. Dessa forma, quando o SR encontra um padrão nos acessos de um usuário que já é conhecido,
é possível utilizar os próximos itens da sequência como recomendações para o usuário. Essa categoria considera que os usuários
tendem a seguir algum padrão de acesso (trajetória) enquanto interagem com o sistema.

\section{Projeto de Interface de Recomendações}\label{section:fundamentacao-apresentacao-recomendacao}

No trabalho de \citeonline{pu2012evaluating} os autores argumentam que apenas a eficiência do algoritmo não garante
que o usuário estará satisfeito com o sistema, será leal e continuará utilizando-o ou que os itens serão ''convertidos''
(nesse sentido, os autores se referem à conversão como a aceitar a recomendação dada e consumir o item recomendado).
Os autores afirmam que percepção do usuário sobre a qualidade da recomendação é afetada tanto pela
qualidade das recomendações, que é responsabilidade do algoritmo de recomendação, quanto pela eficiência na apresentação
das recomendações, explicando a razão daquelas recomendações e inspirando a confiança do usuário nas suas decisões.
Para isso, os autores defendem uma avaliação do SR pela perspectiva do usuário, de forma a avaliar não somente o algoritmo de
recomendação, mas o SR como um todo \cite{pu2012evaluating}.

Além disso, \citeonline{pu2012evaluating} definem um conjunto de vinte diretrizes para o design de um SR bem aceito
pelos usuários. Essas diretrizes foram criadas a partir da combinação do resultado de vários trabalhos que executaram
experimentos com participação de usuários (i.e., Estudos com usuários) para avaliar a interface de SRs. As principais
diretrizes levadas em conta por esse trabalho são \cite{pu2012evaluating}:

\begin{itemize}
\item Diretriz 14: Considere aprimorar a acurácia percebida pelo usuário com um \textit{layout} mais atrativo, rótulos mais
efetivos, e explicando como o sistema gerou as recomendações. Fazendo isso, pode-se aumentar a percepção do usuário sobre a
eficiência do sistema, sua satisfação com o sistema em geral, sua prontidão para aceitar os itens recomendados e a sua
confiança no sistema.
\item Diretriz 18: Considere fornecer como recomendação itens compatíveis ao contexto do usuário. Essa característica
pode estar altamente relacionada com a percepção de utilidade do sistema e da satisfação do usuário.
\item Diretriz 19: Considere explicar porque o sistema recomendou determinados itens. Esses aspectos podem estar
altamente relacionados com a satisfação do usuário, a percepção de controle, as intenções do usuário inspiradas pela
confiança do usuário, como a intenção de retornar ao sistema.
\item Diretriz 20: Considere fornecer informação suficiente relacionadas aos itens recomendados, controlar a qualidade
das informações e da estrutura de navegação.
\end{itemize}

\section{Avaliação de Sistemas de Recomendação}\label{section:fundamentacao-avaliacao-sr}

Para avaliar o desempenho dos algoritmos de recomendação, as métricas (quantitativas) tradicionalmente utilizadas são:

\begin{itemize}
\item Erro Absoluto Médio (do inglês \textit{Mean Absolute Error} e \textit{Root Mean Square Error} (RMSE): utilizadas
para calcular o quão próximas as previsões do algoritmo de recomendação estão da realidade;
\item Precisão: definida pela divisão do número de itens relevantes recomendados pelo número total de itens recomendados;
\item \textit{Recall}: definida pela divisão do número de itens relevantes recomendados pelo número total de itens
relevantes existentes;
\item Cobertura (do inglês \textit{Coverage}): definida pela união de todas a listas de recomendação geradas (i.e., todos os itens
distintos recomendados) pela quantidade de itens disponíveis no sistema. Também chamada de Cobertura do Catálogo \cite{ge2010beyond};
\item \textit{F-measure}: definida pela média harmônica entre Precisão e Recall.
\end{itemize}

A avaliação dos SRs é comumente realizada através de experimentos, comparando dois ou mais algoritmos de recomendação. A
avaliação pode ter por objetivo medir as métricas quantitativas citadas anteriormente ou fazer uma avaliação pela
perspectiva do usuário onde informações qualitativas são extraídas e analisadas. Os métodos de avaliação de SRs podem ser
divididos em três categorias \cite{shani2011evaluating}:

\begin{itemize}
\item \textit{Offline}: avaliação do método de recomendação através de uma base de dados, simulando as ações
dos usuários sem necessitar da participação dos mesmos. Essa avaliação geralmente utiliza uma estratégia onde a base de dados
é separada em base de treinamento e base de testes. A base de treinamento terá as notas dadas pelo usuário que serão
repassadas ao algoritmo de recomendação como forma de construir o perfil dos usuários. A base de teste contém os itens para os
quais o algoritmo de recomendação irá buscar prever o interesse do usuário. As métricas apresentadas anteriormente são
utilizadas para medir a eficiência do algoritmo, comparando a recomendação e/ou predições realizadas pelo algoritmo de
recomendação com o resultado real presente na base de teste;
\item Estudos com os usuários: um pequeno grupo de usuários é convidado a participar de um experimento e realiza
tarefas específicas relacionadas ao SR. As métricas tradicionais de SR podem ser utilizadas em conjunto com medidas
qualitativas para mensurar a satisfação dos usuários, por exemplo através da observação, questionários, entrevistas, etc.;
\item Uso real do sistema: o SR é avaliado em situações reais de uso, com uma quantidade grande de usuários, e
os dados para a avaliação (quantitativa) são capturados de forma automática, por exemplo através de ferramentas de
\textit{Web Analytics}, registros de \textit{Logs}, notas dadas pelos usuários, etc.
\end{itemize}

\citeonline{pu2011user} propõem um \textit{framework} para a avaliação de SRs utilizando \textbf{Estudos com os usuários}, com o objetivo
de realizar uma avaliação do SR através da perspectiva do usuário. Esse \textit{framework} se chama \textit{Recommender
Systems' Quality of User Experience}  (ResQue) e foi proposto com base em outras ferramentas para avaliação centrada no
usuário não exclusivas de SR: \textit{Technology Acceptance Model} (TAM), que define três construtos (Facilidade de Uso
Percebida, Utilidade Percebida e Intenções do Usuário em Utilizar o Sistema) e \textit{Software Usability Measurement
Inventory} (SUMI), que consiste de cinco construtos (Eficiência, Influência, Ajuda, Controle, Capacidade de Aprendizado)
e um questionário de 50 questões.

O \textit{framework} proposto por \citeonline{pu2011user} consiste em quatro construtos: (1) Qualidades Percebidas pelos
Usuários, (2) Crenças/Opiniões do Usuário, (3) Atitudes/Propósitos do Usuário, (4) Intenções Comportamentais.
Para cada um dos construtos, vários aspectos são avaliados, como pode ser visto na Figura \ref{fig:resque-framework}.
Os autores definem ainda um conjunto de 60 questões para aplicar nessa avaliação como pode ser visto no Anexo
\ref{ane:questoes-framework}. Nesse questionário as questões são afirmações nas quais o usuário deve ser posicionar em
um escala de Likert de 5 pontos, de ''Discordo totalmente'' até ''Concordo totalmente''. Os autores ainda afirmam que o
conjunto de questões aplicado pode ser reduzido para um subconjunto com 15 questões (questões com asterisco no Anexo
\ref{ane:questoes-framework}).

\begin{figure}[htb]
  \caption{\label{fig:resque-framework}Construtos do \textit{framework} de avaliação de SRs pela perspectiva do usuário}
  \begin{center}
      \includegraphics[scale=0.4]{./Figuras/resque-framework-traduzido.png}
  \end{center}
  \legend{Fonte: \citeonline{pu2011user}}
\end{figure}

\section{Considerações sobre o capítulo}

Nesse capítulo foram apresentados os principais conceitos relacionados a Sistemas de Recomendação (SRs) Sensíveis ao Tempo.
Foram apresentadas desde as abordagens tradicionais, passando pelos SR Sensíveis Contexto e os seus paradigmas até as
categorias de SRs Sensíveis ao Tempo definidas em \citeonline{de2017time}.

Dentre as categorias de SRs Sensíveis ao Tempo apresentadas na Seção \ref{section:sr-sensivel-tempo}, a utilizada por esse
trabalho é o \textit{Decay}. Nessa categoria é considerado que o interesse do usuário por um item acessado diminui com o passar
do tempo. Neste sentido, considerando que o acesso a um item é um indicativo do interesse do usuário, os itens acessados
mais recentemente têm um peso maior na identificação dos interesses do usuário, i.e., na definição do perfil do usuário.

Além disso, foram apresentadas as formas de avaliação de um SR como definido por \citeonline{shani2011evaluating}. Neste
trabalho será utilizada uma avaliação pela perspectiva do usuário, como orientado por \citeonline{pu2012evaluating}. Por isso, é
apresentado o \textit{framework} definido por \citeonline{pu2011user} que busca avaliar a experiência do usuário através de um
questionário com 60 questões dividas nos quatro construtos do \textit{framework}. Para ser utilizado, é necessário
selecionar quais das questões serão utilizadas, pois como citado por \citeonline{pu2011user} nem todas as questões se
aplicam a todos os SRs.

\include{Partes/3-trabalhos-relacionados}
\include{Partes/4-sr-sensivel-ao-tempo}
\chapter{Experimento}\label{chapter:experimento}

Neste capítulo é apresentado o experimento realizado para a avaliação da proposta apresentada no Capítulo \ref{chapter:proposta}.
Para avaliar a proposta deste trabalho foi utilizada uma avaliação pela perspectiva do usuário, que segundo a definição de
\citeonline{shani2011evaluating} se encaixa em um Estudo com usuários. O algoritmo proposto foi incorporado ao ambiente
\adaptwebspace e avaliado em uma situação real de uso em um Minicurso de Algoritmos desenvolvido por \citeonline{santos2017addie}.
Na Seção \ref{section:planejamento-experimento} é descrito o ambiente \adaptwebspace no qual a proposta foi incorporada,
as mudanças realizadas no ambiente para o experimento, o objetivo do experimento e o teste piloto realizado. A Seção
\ref{section:execucao-experimento} descreve o experimento que foi realizado e a Seção \ref{section:analise-experimento} apresenta
as análises realizadas no uso do Sistema de Recomendação (SR) e no questionário aplicado.

\section{Planejamento}\label{section:planejamento-experimento}

\subsection{Descrição do Ambiente \adaptweb}

O \adaptwebspace (Ambiente de Ensino-Aprendizagem Adaptativo na Web) é um sistema open source
que consiste em um AVA capaz de adaptar o conteúdo, a apresentação e a navegação em determinado curso às características
e preferências do aluno \cite{gasparini2009adaptweb}. A Seção \ref{subsection:estrutura-adaptweb} apresenta a Estrutura Geral do
\adaptweb.

\subsubsection{Estrutura do \adaptweb}\label{subsection:estrutura-adaptweb}

A estrutura do \adaptwebspace é composta por quatro módulos: (1) o módulo de autoria; (2) o
módulo de armazenamento em XML (Extensible Markup Language); (3) o módulo de adaptação do conteúdo baseado no modelo do
usuário e (4) o módulo de interface adaptativa \cite{gasparini2003interface}, conforme pode ser visto na Figura
\ref{fig:adaptweb-arquitetura}.

\begin{figure}[htb]
  \caption{\label{fig:adaptweb-arquitetura}Estrutura do \adaptweb}
  \begin{center}
      \includegraphics[scale=1.0]{./Figuras/adaptweb-arquitetura.png}
  \end{center}
  \legend{Fonte: \citeonline{gasparini2003interface}}
\end{figure}

O módulo de autoria (1) consiste na organização do conteúdo instrucional a ser disponibilizado para o aluno, sendo que
este conteúdo pode ter arquivos classificados como conceito, exemplos, exercícios e materiais complementares
\cite{gasparini2003interface}. Ao criar um conteúdo no sistema, o autor pode definir para quais cursos e disciplinas
deseja que o conteúdo ou arquivo esteja disponível. Isto significa que um aluno de um Curso X e de outro Curso Y,
matriculados em uma mesma disciplina, podem ter conteúdos distintos, conforme definido pelo professor. Por exemplo, a
disciplina de Cálculo I pode ser oferecida para os cursos de Ciência da Computação e Engenharia Elétrica e sua
abrangência e profundidade pode ser distinta para cada curso.

Em \citeonline{de2015sistema}, foi proposto uma nova categoria para os conteúdos chamada Links de Apoio. Esses Links de Apoio
são links externos ao ambiente \adaptwebspace que são cadastrados pelo professor como um
material alternativo de estudo e não estão diretamente atrelados á nenhum conceito em específico. O objetivo foi criar
uma nova categoria de materiais que poderia ser recomendada para o usuário a qualquer momento de sua interação.

O módulo de armazenamento em XML (2) é responsável por organizar os conteúdos e arquivos disponibilizados pelo autor em
um arquivo XML \cite{gasparini2003interface}. É utilizada a representação através de XML devido à sua alta
flexibilidade, oferecendo a estruturação dos documentos de forma independente da apresentação.

O módulo de adaptação do conteúdo baseado no modelo do aluno (3) é responsável por adaptar o conteúdo da disciplina
para cada curso. Por fim, o módulo de interface adaptativa (4) é responsável pela adaptação da navegação e da
apresentação da interface do ambiente de acordo com o curso, preferências do modo de navegação (modo tutorial ou livre)
e o conhecimento do usuário \cite{gasparini2003interface}.

\subsubsection{Sistema de Recomendação no \adaptweb}

Na Seção \ref{subsection:estrutura-adaptweb} foi apresentada a estrutura do ambiente \adaptweb, que possui quatro categorias
de materiais para cada conteúdo. Além disso, existe uma outra categoria chamada Links de Apoio com o propósito de ser um
material auxiliar e que pode ser recomendado a qualquer momento para o usuário.

Fazendo uma relação da estrutura do \adaptwebspace com o algoritmo proposto no Capítulo \ref{chapter:proposta}, os itens de
das categorias Conceito, Materiais Complementares e os próprios Links de Apoio serão considerados para a composição do perfil do usuário. Todos os
materiais acessados em cada uma das categorias é representado através das palavras-chave, e
essas palavras-chave farão parte do perfil do aluno a partir do momento em que este acessar o material. Já para os itens
recomendados, apenas os Links de Apoio serão utilizados.

Como no ambiente \adaptwebspace as palavras-chave para o itens podem ser cadastradas pelo professor da disciplina, não será
utilizada nenhuma técnica para captura automática das palavras-chave. Para a representação dos materiais envolvidos no
processo de recomedanção (i.e., Conceitos, Materiais Complementares e Links de Apoio) foi criado um Dicionário de Palavras-Chave
com as possíveis palavras a ser utilizadas. Esse Dicionário foi essencial para o funcionamento do SR, pois garante que as
palavras-chave presentes nos materiais que são similares também serão similares. Isso evita a necessidade de lidar com palavras-chave
sinônimas ou a variação de singular e plural, já que as palavras-chave que podem ser utilizadas para representar os materiais
são restritas às presentes no dicionário.

O Dicionário completo pode ser visto no Apêndice \ref{ape:dicionario-palavras-chave}. A criação do Dicionário foi validado
por um aluno do Mestrado em Ensino de Ciências, Matemática e Tecnologias que também é professor da disciplina de
Algoritmos no SENAC-SC. O professor teve acesso ao Minicurso de Algoritmos utilizado no experimento
e teve o papel de avaliar o Dicionário criado e acrescentar mais palavras para agregar conjunto de palavras-chave.

Depois de criado e validado o Dicionário, foi realizada a associação de forma manual entre cada material dos Conceitos,
Materiais Complementares e Links de Apoio com as palavras-chave do Dicionário. No total, 51 Conceitos, 28 Materiais
Complementares e 108 Links de Apoio foram analisados. O resultado dessa associação pode ser visto no Apêndice \ref{ape:palavras-chave-materiais}.

O SR irá buscar, com base nos itens acessados pelo aluno, os Links de Apoio mais adequados
para a recomendação e irá apresentar através de uma lista de itens. A forma de apresentação das Recomendações é discutida
em mais detalhe na Seção \ref{subsection:apresentacao-recomendacoes}.

\subsubsection{Apresentação das Recomendações}\label{subsection:apresentacao-recomendacoes}

A lista de recomendações neste trabalho será apresentada ao aluno na tela principal do ambiente do aluno. Dessa forma,
quando o SR possuir itens para recomendar para o usuário esses itens aparecem em uma lista logo abaixo do conteúdo que
ele estiver visualizando no momento, independente se o aluno estiver na tela de Conceito, Exercícios, Exemplos ou
Materiais Complementares. Na Figura \ref{fig:adaptweb-proposta-recomendacao} pode-se observar a tela inicial do ambiente do
aluno, onde estão destacadas as seguintes áreas: (1) Menu de navegação pelos tópicos; (2) Categorias dos materiais dentro
do ambiente; (3) Interface das recomendações; (4) Mapa da disciplina; (5) Ajuda. As recomendações podem ser apresentadas ao usuário no momento em que este estiver acessando
quaisquer itens que sejam da categoria Conceito e Materiais Complementares, assim que o SR possuir itens relevantes para recomendar.

\begin{figure}[htb]
  \caption{\label{fig:adaptweb-proposta-recomendacao}Proposta de Interface de Recomendação}
  \begin{center}
      \includegraphics[scale=0.4]{./Figuras/interface-recomendacao.png}
  \end{center}
  \legend{Fonte: O autor.}
\end{figure}

Como visto na imagem, as principais informações dos Links de Apoio recomendados apresentados para o aluno são: Link, Nome do Link, Descrição e a possibilidade de avaliar o item
positivamente ou negativamente. A avaliação feita pelo usuário não é considerada pelo algoritmo de recomendação, sendo que os itens acessados pelo
usuário são considerados como do seu interesse. Como trabalho futuro é possível incorporar a avaliação na recomendação, e.g.,
não tornar a recomendar itens que foram avaliado com notas baixas ou apenas considerar para o algoritmo Baseado em Conteúdo
os itens avaliados positivamente. \citeonline{pu2012evaluating} afirmam que enquanto recomendar um item apenas é pouco, recomendar mais do que cinco itens
aumenta a dificuldade de escolhar do usuário. Por isso, a quantidade máxima de itens recomendadas para o usuário em cada
recomendação é de cinco itens.

Para cumprir o requisito de Explicação das recomendações citada por \citeonline{pu2012evaluating}, foi adicionado o
botão de ''Entenda melhor'' que tem por objetivo explicar ao usuário como a lista de itens foi gerada. Ao entender o
funcionamento do algoritmo de recomendação o usuário tem a possibilidade aprimorar o seu perfil para personalizar as
recomendações recebidas. Na Figura \ref{fig:adaptweb-proposta-explicacao} está um protótipo da explicação da recomendação
mostrada para o aluno.

\begin{figure}[htb]
  \caption{\label{fig:adaptweb-proposta-explicacao}Explicação da recomendação}
  \begin{center}
      \includegraphics[scale=0.6]{./Figuras/explicacao-das-recomendacoes.png}
  \end{center}
  \legend{Fonte: O autor.}
\end{figure}

\subsection{Definição do experimento}\label{subsection:definicao-experimento}

Para a execução do experimento, o SR proposto foi comparado a abordagem Baseada em Conteúdo tradicional utilizando uma
estratégia \textit{Between Subjects}, i.e., os alunos foram divididos em dois grupos e cada grupo utilizou apenas
um dos sistemas. Para garantir que a única variável seja o SR utilizado, ambos os grupos utilizaram a mesma
interface proposta para as recomendações.

O experimento proposto neste trabalho visa avaliar o desempenho e a percepção do usuário dontr o SR proposto quando
comparado à abordagem Baseada em Conteúdo Tradicional. Para avaliar o desempenho do algoritmo de recomendação em relação
à variável independente \textit{Algoritmo de Recomendação} foram adotadas as seguintes hipóteses:

\begin{itemize}
\item \textbf{H\textsubscript{0}:} Não há diferenças entre o desempenho da abordagem Baseada em Conteúdo
Tradicional e a proposta desse trabalho.
\item \textbf{H\textsubscript{1}:} Há diferenças entre o desempenho da abordagem Baseada em Conteúdo
Tradicional e a proposta desse trabalho.
\end{itemize}

Para avaliar a percepção do usuário sobre a qualidades das recomendações recebidas em relação
à variável independente \textit{Algoritmo de Recomendação} foram adotadas as seguintes hipóteses:

\begin{itemize}
\item \textbf{H\textsubscript{0}:} Não há diferenças na percepção do usuário da qualidade das recomendações recebidas utilizando a abordagem
Baseada em Conteúdo tradicional e a proposta desse trabalho.
\item \textbf{H\textsubscript{1}:} Há diferenças na percepção do usuário da qualidade das recomendações recebidas utilizando a abordagem
Baseada em Conteúdo tradicional e a proposta desse trabalho.
\end{itemize}

O experimento foi realizado através do Minicurso de Algoritmos e Linguagem de Programação, o qual teve seu design
instrucional realizado por \citeonline{santos2017addie}. Foram convidados a participar alunos de todos os cursos do
Centro de Ciências Tecnológicas (CCT) da Universidade do Estado de Santa Catarina (UDESC), sendo que todos os cursos do
CCT possuem essa disciplina na grade curricular. Os convites foram realizados em todas as salas das disciplinas de
Algoritmos (ALG), Algoritmos e Linguagem de Programação (ALP), Linguagem de Programação (LPG) e Iniciação a Ciência da
Computação (ICC). Além disso, foi enviado um convite para todos os alunos do campus por e-mail através da Assessoria de
Comunicação e foi divulgado na página do Facebook da UDESC Joinville.

Os usuários que se matricularam no Minicurso foram aleatoriamente divididos em dois grupos, da forma mais igualitária possível
pelos mesmos critérios utilizados por \citeonline{klockanalise}: Professor, Curso, Sexo e Idade. Isso foi possível
porque durante o processo de matrícula os alunos responderam um questionário para montar o seu perfil. Também durante a
matrícula os alunos tiveram acesso ao Termo de Consentimento Livre e Esclarecido (TCLE), presente no Apêndice
\ref{ape:termo-de-consentimento}, que explica o objetivo do experimento e no qual eles consentiram em participar e em
permitir o uso dos resultados para essa pesquisa, sempre garantindo a anonimidade dos participantes.

Ao final do minicurso, os alunos puderam acessar a avaliação do Minicurso, composta de 10 questões, e o questionário
sobre a percepção do usuário sobre o SR. As questões foram selecionadas do
conjunto de questões definidas por \citeonline{pu2011user} (presente no Anexo \ref{ane:questoes-framework}), de acordo
com o objetivo desse experimento que é medir a percepção do usuário sobre a qualidade das recomendações recebidas.
As questões selecionadas foram traduzidas para o Português e estão presentes no Apêndice \ref{ape:questionario-de-satisfacao}.
Além dessas questões, foi adicionada uma questão aberta perguntando sobre os pontos positivos e negativos do SR utilizado.

Durante o desenvolvimento do Minicurso foram utilizadas as Intervenções definidas por \citeonline{santos2017addie}, para
fazer com que os alunos fiquem engajados no curso. As Intervenções são e-mails combinadas com postagens no Fórum de Discussão
que guiam os alunos no seu estudo e também propõe desafios para os estudantes. As Intervenções propostas por \citeonline{santos2017addie}
consideram o Minicurso com uma duração de dois meses, por isso foi necessário uma adaptação dessas Intervenções para o período
mais reduzido no qual foi realizado esse experimento. As Intervenções adaptadas de \citeonline{santos2017addie} podem
ser vistar no Apêndice \ref{ape:intervencoes} e os Desafios postados no Fórum de Discussão estão presentes no Apêndice \ref{ane:desafios}.

\subsection{Teste piloto}\label{section:planejamento-teste-piloto}

Antes do experimento ser realizado com os alunos no Minicurso de Algoritmos, foi realizado um teste piloto com
quatro alunos que já realizaram essa disciplina. O objetivo do teste piloto foi avaliar os instrumentos do experimento,
além de permitir encontrar problemas na experiência do usuário para serem corrigidos antes da execução do minicurso. O teste piloto
foi realizado no dia 06 de Abril de 2018.

Durante o teste piloto os alunos foram divididos em dois grupos aleatóriamente, sendo que dois alunos utilizaram o SR
Baseado em Conteúdo Tradicional e os outros dois utilizaram o SR com o Decaimento. Os alunos receberam
um protocolo de atividades para realizar, presente no Apêndice \ref{ape:teste-piloto}. As tarefas envolvem
realizar a matrícula na disciplina, na qual eles leram e aceitaram o TCLE, realizar o acesso a alguns conceitos e
materiais complementares, utilizar o SR, realizar a avaliação da disciplina e responder ao questionário sobre a experiência
com o SR. Durante o teste, os comentários e observações feitas pelos alunos foram anotadas para posterior análise.

Os quatro participantes do teste piloto foram identificados como Participante 1, Participante 2,
Participante 3 e Participante 4. Os Partipantes 1 e 2 utilizaram o algoritmo tradicional de recomendação, enquanto os
participantes 3 e 4 utilizaram a proposta desse trabalho, porém eles não sabiam disso durante a realização do Teste. Os
participantes foram livres na escolha do Sistema Operacional e Navegador utilizados, bem como no Modo de Navegação
escolhido (Livre ou Tutorial). A Tabela \ref{tab:participantes-teste-piloto} apresenta o Modo de Navegação e o Algoritmo
de Recomendação utilizado por cada participante.

\begin{table}[h]
\footnotesize
\caption[Dados dos Participantes do Teste Piloto]{Dados dos Participantes do Teste Piloto}
\label{tab:participantes-teste-piloto}
\centering
\begin{tabular}{|p{2cm}|p{3.5cm}|p{6.5cm}|}
  \hline
  \textbf{Participante} & \textbf{Modo de Navegação} & \textbf{Algoritmo de Recomendação} \\
  \hline
  1 & Tutorial & Baseado Em Conteúdo Tradicional \\
  \hline
  2 & Livre & Baseado Em Conteúdo Tradicional \\
  \hline
  3 & Tutorial & Baseado Em Conteúdo com Decaimento \\
  \hline
  4 & Livre & Baseado Em Conteúdo com Decaimento \\
  \hline
\end{tabular}
\legend{Fonte: O autor.}
\end{table}

Durante a execução do Teste Piloto, os participantes encontraram erros de digitação no Termo de Consentimento Livre e Esclarecido (TCLE)
e alguns problemas na prova da disciplina, como perguntas de Verdadeiro e Falso com questões duplicadas e uma
pergunta que não possuía nenhuma resposta certa. Esses problemas foram todos resolvidos antes do início do experimento.

Sobre as recomendações, os Participantes 1 e 2 comentaram que aparentemente os Links de Apoio recomendados para eles não
mudaram muito durante toda a interação - as vezes mudavam de ordem apenas, porém continuavam os mesmo itens. Já os Participantes
3 e 4, que utilizaram a proposta desse trabalho, comentaram que ao acessar um novo conceito pelo menos 3 novos Links eram
recomendados. Esse resultado demonstra o problema da Superespecilização presente na Abordagem Baseada em Conteúdo Tradicional,
e mostra que a proposta desse trabalho utilizando o Decaimento ajuda a diminuir esse problema.

Os participantes também observaram que desde o primeiro acesso a disciplina eles receberam cinco links com recomendação, i.e.,
o número máximo possível. Isso mostra que, como não foi definido um limiar mínimo para a similaridade entre o perfil do usuário
e os Links de Apoio, mesmo que a similaridade seja muito pequena o algoritmo sempre irá recomendar algo para o aluno. Por
outro lado, não seria interessante adicionar um limiar para o experimento deste trabalho pois estaria adicionando
mais uma variável independente ao experimento. Como trabalho futuro é possível analisar como o limiar mínimo para a
similaridade pode afetar a qualidade percebida das recomendações.

Ao final do Teste Piloto foi realizada uma pequena entrevista com os participantes onde eles afirmaram ter entendido
a interface do SR. Quando revelado que cada participante existiam dois SRs e que duas pessoas estavam usando a abordagem
Tradicional e os outros dois estavam usando a proposta do trabalho, os participantes associaram essa informação com os
comentários que feitos sobre a repetição dos itens, pelos alunos da abordagem Tradicional, e a novidade nas recomendações,
pelos alunos da Proposta.

\section{Execução}\label{section:execucao-experimento}

O período de matrícula para o minicurso foi de 09 de Abril de 2018 até 13 de Abril de 2018. Durante esse período, todas as turmas
das disciplinas mencionadas na Seção \ref{subsection:definicao-experimento} foram visitadas, convidando os alunos a se matricular no minicurso.

No total, 208 alunos se matricularam no Minicurso. Esses alunos foram homogeneamente divididos em dois grupos (AlgoritmoTradicional
e AlgoritmoProposta) utilizando os seguintes critérios: Professor, Curso, Sexo e Idade. A Tabela \ref{tab:divisao-alunos-experimento} mostra o resultado da divisão dos alunos.

\begin{table}[h]
\footnotesize
\caption[Divisão dos alunos de acordo com os critérios]{Divisão dos alunos de acordo com os critérios}
\label{tab:divisao-alunos-experimento}
\centering
\begin{tabular}{|c|c|c|c|c|}
  \hline
  \multicolumn{3}{|c|}{}                                                                         & \multicolumn{2}{c|}{\textbf{Algoritmo de Recomendação}} \\ \hline
  \multicolumn{2}{|c|}{\textbf{Critério}}                  & \multicolumn{1}{c|}{\textbf{Total}} & AlgoritmoTradicional & AlgoritmoProposta                \\ \hline
  \multirow{12}{*}{Professor}           & Professor A      & 12                                  & 6                    & 6                                \\
                                        & Professor B      & 18                                  & 9                    & 9                                \\
                                        & Professor C      & 7                                   & 4                    & 3                                \\
                                        & Professor D      & 5                                   & 3                    & 2                                \\
                                        & Professor E      & 38                                  & 19                   & 19                               \\
                                        & Professor F      & 4                                   & 2                    & 2                                \\
                                        & Professor G      & 10                                  & 5                    & 5                                \\
                                        & Professor H      & 6                                   & 3                    & 3                                \\
                                        & Professor I      & 6                                   & 3                    & 3                                \\
                                        & Professor J      & 26                                  & 13                   & 13                               \\
                                        & Professor K      & 14                                  & 7                    & 7                                \\
                                        & Outro            & 52                                  & 30                   & 32                               \\ \hline
  \multirow{9}{*}{Curso}                & Computação       & 55                                  & 23                   & 22                               \\
                                        & TADS             & 36                                  & 18                   & 18                               \\
                                        & Elétrica         & 36                                  & 18                   & 18                               \\
                                        & Física           & 10                                  & 5                    & 5                                \\
                                        & Mecânica         & 31                                  & 15                   & 16                               \\
                                        & Química          & 3                                   & 2                    & 1                                \\
                                        & Produção         & 20                                  & 10                   & 10                               \\
                                        & Civil            & 15                                  & 7                    & 8                                \\
                                        & Matematica       & 12                                  & 6                    & 6                                \\ \hline
  \multirow{3}{*}{Sexo}                 & Masculino        & 135                                 & 67                   & 68                               \\
                                        & Feminino         & 72                                  & 36                   & 36                               \\
                                        & Não informado    & 1                                   & 1                    & 0                                \\ \hline
  \multirow{6}{*}{Idade}                & Até 17 anos      & 21                                  & 10                   & 11                               \\
                                        & 18 ou 19 anos    & 75                                  & 37                   & 38                               \\
                                        & 20 ou 21 anos    & 35                                  & 18                   & 17                               \\
                                        & 22 ou 23 anos    & 11                                  & 6                    & 5                                \\
                                        & 24 ou 25 anos    & 26                                  & 13                   & 13                               \\
                                        & 26 anos ou mais  & 40                                  & 20                   & 20                               \\ \hline
                                        & Total            & 208                                 & 104                  & 104                              \\ \hline
\end{tabular}
\end{table}

O período de execução do Minicurso foi de 16 de Abril de 2018 até 10 de Maio de 2018. Nesse período, dos 208 alunos
matriculados no Minicurso, 145 acessaram a disciplina pelo menos uma vez, sendo 76 do grupo AlgoritmoTradicional e 69
do grupo AlgoritmoProposta.

O período para realizar a prova da disciplina e responder ao questionário foi de 11 de Maio de 2018 até
14 de Maio de 2018. Nesse período, dos 145 alunos que acessaram a disciplina pelo menos uma vez, 85 realizaram a prova final e responderam ao
questionário, sendo 48 do grupo AlgoritmoTradicional e 37 do grupo AlgoritmoProposta. Dos 85 alunos que finalizaram a
disciplina (i.e., realizaram a prova e responderam ao questionário), 47 acessaram pelo menos uma recomendação que
recebeu, sendo 25 do grupo AlgoritmoTradicional e 22 do grupo AlgoritmoProposta. A Tabela \ref{tab:uso-minicurso-sr}
apresenta mais informações sobre as ações dos alunos dentro do \adaptweb.

\begin{table}[h]
\footnotesize
\centering
\caption{Uso do Minicurso de Algoritmos e do SR}
\label{tab:uso-minicurso-sr}
\begin{tabular}{|p{9.5cm}|p{2cm}|p{1.5cm}|p{1cm}|}
\hline
\textbf{Quantidade de:}                                                                     & \textbf{Tradicional} & \textbf{Proposta}    & \textbf{Total}    \\
\hline
Alunos que se matricularam                                                                  & 104                  & 104                  & 208      \\
\hline
Alunos que entraram pelo menos uma vez no curso                                             & 76                   & 69                   & 145      \\
\hline
Alunos que acessaram pelo menos uma recomendação                                            & 46                   & 39                   & 85       \\
\hline
Alunos que avaliaram pelo menos uma recomendação                                            & 30                   & 22                   & 52       \\
\hline
Alunos que realizaram a prova                                                          & 48                   & 38                   & 86       \\
\hline
Alunos que responderam o questionário                                         & 48                   & 37                   & 85       \\
\hline
Alunos que responderam o questionário e acessaram pelo menos uma recomendação & 25                   & 22                   & 47       \\
\hline
Alunos que responderam o questionário e avaliaram pelo menos uma recomendação & 16                   & 13                   & 29       \\
\hline
Acessos aos itens recomendados                                                              & 227                  & 396                  & 623      \\
\hline
Avaliações positivas ao itens recomendados                                                  & 141                  & 234                  & 375      \\
\hline
Avaliações negativas ao itens recomendados                                                  & 5                    & 4                    & 9        \\
\hline
\end{tabular}
\end{table}

Pela Tabela \ref{tab:uso-minicurso-sr} podemos perceber que 38 alunos que responderam ao questionário não acessaram nenhuma
recomendação, sendo 23 do grupo AlgoritmoTradicional e 15 do grupo AlgoritmoProposta. Na Seção
\ref{subsection:analise-uso-sr} os dados de acesso e avaliação das recomendações são analisadas mais profundamente,
comparando os resultados dos dois algoritmos utilizando técnicas estatísticas descritas na Seção \ref{subsection:tecnicas-analise-estatistica}.

\section{Análise dos Resultados}\label{section:analise-experimento}

Nessa Seção são descritas as análises estatísticas realizadas sobre os dados coletados durante a execução do experimento.
As técnicas estatísticas utilizadas para fazer a análise são apresentadas na Seção \ref{subsection:tecnicas-analise-estatistica}
e nas Subseções subsequentes são apresentas as análises realizadas sobre o desempenho dos algoritmos de recomendação e sobre
a percepção dos alunos sobre a qualidade do SR utilizado.

\subsection{Técnicas de Análise Estatística}\label{subsection:tecnicas-analise-estatistica}

Para realizar as análises estatísticas, primeiro foi necessário entender os tipos de variáveis que podem ser analisadas.
As variáveis são as informações coletadas durante o experimento e podem ser quantitativas ou qualitativas \cite{bussab2012morettin}.
As variváveis quantitativas são resultado de uma contagem e/ou mensuração \cite{bussab2012morettin} e podem ser discretas
(conjunto finito ou enumerável de números) ou contínuas (pertencem a um intervalo de números reais). Exemplos de variáveis
quantitativas são número de filhos, número de cômodos em uma casa, número de pessoas presentes em uma sala, altura e peso.
As variáveis qualitativas são aquelas que descrevem um aspecto relacionado ao objeto estudado \cite{bussab2012morettin} e podem ser
nominais (quando não possuem ordenação) ou ordinais (quando há ordem entre os valores). Exemplos de variáveis qualitativas
são sexo, estado civil e grau de instrução.

A variável independente desse trabalho é o algoritmo de recomendação utilizado pelo alunos, que pode assumir os valores
AlgoritmoTradicional ou AlgoritmoProposta. As variáveis dependentes analisadas são o
desempenho do algoritmo de recomendação e a percepção do usuário sobre as recomendações. O desempenho do SR
foi medido através da quantidade de recomendações acessadas, quantidade de recomendações avaliadas, Precisão e Cobertura do
algoritmo. A quantidade de recomendações acessadas e/ou avaliadas são variáveis quantitativas discretas,
enquanto a Precisão e a Cobertura do algoritmo de recomendação são variáveis quantitativas contínuas. A percepção do usuário sobre
as recomendações recebidas, que foi medida através do questionário, pode ser categorizado como uma variável
qualitativa ordinal, pois as respostas da escala de Likert podem ser ordenadas de Discordo Totalmente até Concordo Totalmente.

De acordo com o tipo de variável analisado e a distribuição dos valores dessa variável é possível definir qual técnica
estatística será utilizada. O fluxograma da Figura \ref{fig:fluxograma-tecnica-moissa} produzido por \citeonline{moissa2016influencia}
define como decidir as técnicas estatísticas utilizadas. Todos os testes mencionados na Figura \ref{fig:fluxograma-tecnica-moissa}
estão disponíveis na ferramenta estatística R \cite{rstatisticalcomputing2018}, que foi a ferramenta utilizada para todas as análises realizadas nas
Subseções a seguir.

\begin{figure}[htb]
  \caption{\label{fig:fluxograma-tecnica-moissa}Fluxograma de uso das técnicas estatísticas}
  \begin{center}
      \includegraphics[scale=1.0]{./Figuras/fluxograma-tecnicas-moissa.png}
  \end{center}
  \legend{Fonte: \citeonline{moissa2016influencia}}
\end{figure}

\subsection{Análise do Desempenho da Abordagem de Recomendação}\label{subsection:analise-uso-sr}

Nessa seção é realizada uma análise estatística sobre os acessos e avaliações dos usuários aos Links de Apoio recomendados.
O objetivo é extrair informações dos dados capturados durante a interação para avaliar o desempenho do algoritmo em
relação a diversas métricas. As análises feitas foram da (1) Quantidade de recomendações acessadas,
(2) Quantidade de avaliações positivas para as recomendações, (3) Quantidade de avaliações negativas para as recomendações,
(4) Precisão do algoritmo de recomendação, (5) Cobertura do algoritmo de recomendação e (6) F-measure do algoritmo de
recomendação, comparando os alunos que utilizaram a abordagem Baseada em Conteúdo Tradicional com os alunos
que utilizaram a Proposta desse trabalho. Os resultados das análises realizadas nessa Seção estão presentes no Apêndice \ref{ape:analise-estatistica-do-uso}.

Para a análise da quantidade de recomendações acessadas, foram considerados apenas os alunos que acessaram pelo
menos uma recomendação no ambiente (i.e., 85 alunos). A média do grupo AlgoritmoTradicional foi de 4,935 itens
acessados por aluno, enquanto no grupo AlgoritmoProposta foi de 6,5 itens. A análise utilizando o Teste de normalidade de
Shapiro-Wilk mostrou que a distribuição dos dados não é normal e portanto o Teste não-paramétrico de Mann-Whitney U foi utilizado
para comparar os dois fatores. Esse Teste mostrou que não existe diferença significativa entre as amostras e portanto
não podemos que afirmar que existe diferença na quantidade de recomendações acessadas nos dois grupos analisados.

A análise realizada sobre as avaliações positivas para as recomendações utilizou apenas os alunos que avaliaram positivamente
alguma recomendação (i.e., 52 alunos). A média da quantidade de avaliações positivas para o grupo AlgoritmoTradicional
foi de 4,7 avaliações por aluno, enquanto no grupo AlgoritmoProposta foi de 10,64 avaliações por aluno.
Essa análise teve um resultado similar ao anterior, onde o Teste de normalidade de Shapiro-Wilk também mostrou que a
distribuição é não-normal e o Teste de Mann-Whitney U mostrou que não existem diferença signifitiva entre
as amostras dos dois grupos.

A análise da quantidade de avaliações negativas para as recomendações utilizou apenas os alunos que avaliaram negativamente
os itens (i.e., 7 alunos). A média de avaliações para o grupo que AlgoritmoTradicional foi de 1 avaliação negativa
por aluno, enquanto para o grupo AlgoritmoProposta foi de 2 avaliações negativas por aluno. Ao realizar
a análise estatística, o Teste de normalidade de Shapiro-Wilk mostrou que a distribuição dos dados é não-normal. O Teste
não-paramétrico de Mann-Whitney U foi aplicado e mostrou que não existe diferença significativa para esse critério entre
os dois grupos.

A Precisão dos algoritmos de recomendação foi calculada para cada aluno que acessou pelo menos uma recomendação dentro do
ambiente (i.e., 85 alunos) dividindo a quantidade de recomendações distintas acessadas pelo usuário pela quantidade de
itens distintos que foram recomendados para o usuário durante todo o minicurso. Essa métrica resulta em um valor entre
0 e 1, onde um valor próximo de 0 indica que poucos dos itens recomendados foram acessados e um valor próximo de 1
mostra que a maioria dos itens recomendados foram acessados, independente da quantidade de vezes que um item foi
acessado e da quantidade de vezes que o item foi recomendado. A média da Precisão para o grupo AlgoritmoTradicional foi
0,49679 e para o AlgoritmoProposta foi 0,39859. O Teste de normalidade de Shapiro-Wilk mostra que a distruibuição dos
dados da Precisão é não-normal e o Teste de Mann-Whitney U mostrou que não existem diferença signifitiva entre as
precisão calculada para cada algoritmo.

A Cobertura dos algoritmos da recomendação foi calculada considerando os alunos que acessaram pelo menos uma recomendação
dentro do Minicurso (i.e., 85 alunos). Essa métrica foi calculada dividindo a quantidade de itens distintos recebidos
como recomendação pelo usuário pela quantidade de itens disponíveis para recomendação na disciplina (i.e., 108 itens) e mostra
a porcentagem dos itens disponíveis para recomendação que foram efetivamente recomendados. A média da Cobertura para o
grupo AlgoritmoTradicional é de 0,10064 e para o AlgoritmoProposta é de 0,17450. Ao realizar o Teste de
normalidade de Shapiro-Wilk o resultado mostrou que a distribuição dos dados é não-normal. Quando aplicado o Teste de Mann-Whitney
U o resultado mostrou que existe diferença significativa entre os resultados dos dois grupos com relação à Cobertura e, nesse caso,
podemos afirmar que o algoritmo Com Decaimento tem uma Cobertura melhor que a algoritmo Tradicional. Essa diferença significativa
mostra que a Proposta desse trabalho ajuda a diminuir o problema da Superespecialização presente na abordagem Tradicional.

Como visto na Seção \ref{section:fundamentacao-avaliacao-sr}, o \textit{F-measure} é calculado como a média harmônica entre
a Precisão e o Recall do algoritmo. Essa métrica é interessante para garantir que o algoritmo não diminua o Recall
para aumentar a Precisão, e vice-versa. Como no experimento realizado o Recall não pode ser determinado, pois não
é possível saber todos os itens relevantes existentes para cada usuário (essa métrica é mais comum em experimentos
\textit{offline}), nesse trabalho foi utilizado a Cobertura no cálculo do \textit{F-measure} ao invés do Recall.
Com isso podemos mostrar se o algoritmo Proposto teve uma melhor Cobertura que a abordagem Tradicional sem diminuir
a sua Precisão. A análise foi realizada utilizando os 85 alunos que acessaram pelo menos uma recomendação e o valor médio
dessa métrica proposta foi de 0,13852 para o grupo AlgoritmoTradicional e de 0,21719 o grupo AlgoritmoProposta.
O Teste de Normalidade de Shapiro-Wilk mostrou que a distribuição dos dados é não-normal e portanto o Teste de
Mann-Whitney U foi utilizado para verificar a diferença significativa entre os dois grupos. Esse teste mostrou que
existe diferença significativa entre os dois grupos e portanto podemos afirmar que F-measure, com a Cobertura ao invés
do Recall, teve um resultado melhor para o algoritmo proposto do que na abordagem tradicional. Isso mostra
que o uso do Decaimento aumentou a Cobertura do algoritmo sem influenciar negativamente outros aspectos, como a Precisão.

Como a Cobertura e F-measure deram diferença significativa, podemos aceitar a hipótese alternativa definida
que diz que ''Há diferenças entre o desempenho da abordagem Baseada em Conteúdo Tradicional e a proposta desse trabalho''
e a diferença significativa nos dois casos foi favorável a proposta desse trabalho.

\subsection{Análise da Percepção do Usuário}\label{subsection:analise-questionario-satisfacao}

A percepção dos usuários da qualidade das recomendações recebidas foi medida por meio do questionário presente no
Apêndice \ref{ape:questionario-de-satisfacao}. O questionário possui 13 questões que foram selecionadas do questionário
definido por \citeonline{pu2011user} (presente no Anexo \ref{ane:questoes-framework}) e traduzidas para o Português.
Essas questões são afirmações nas quais os alunos devem se posicionar em uma escala de
Likert de cinco pontos, de ''Discordo Totalmente'' até ''Concordo Totalmente''. Também foi adicionada uma opção de ''Não utilizei'' para
os alunos que não utilizaram o SR (ou usaram sem perceber). A Tabela \ref{tab:questionario-satisfacao-respostas}
apresenta as respostas agrupadas dos alunos para cada uma das perguntas.

\begin{table}[h]
\footnotesize
\caption{Respostas ao Questionário}
\label{tab:questionario-satisfacao-respostas}
\centering
\begin{tabular}{|p{1.5cm}|p{1.8cm}|p{2.2cm}|p{0.6cm}|p{0.6cm}|p{0.6cm}|p{2.3cm}|p{2cm}|}
\hline
\textbf{Questão} & \textbf{Algoritmo}   & \textbf{Discordo tot.} & \textbf{-} & \textbf{-} & \textbf{-} & \textbf{Concordo tot.} & \textbf{Não utilizei} \\
\hline
\multirow{2}{*}{1}          & Tradicional & 2                   & 1                     & 2                         & 15                    & 11                  & 17           \\
                            & Proposta    & 0                   & 0                     & 6                         & 11                    & 7                   & 13           \\
\hline
\multirow{2}{*}{2}          & Tradicional & 1                   & 0                     & 5                         & 18                    & 6                   & 18           \\
                            & Proposta    & 0                   & 1                     & 7                         & 9                     & 9                   & 11           \\
\hline
\multirow{2}{*}{3}          & Tradicional & 2                   & 0                     & 8                         & 13                    & 8                   & 17           \\
                            & Proposta    & 0                   & 2                     & 4                         & 12                    & 8                   & 11           \\
\hline
\multirow{2}{*}{4}          & Tradicional & 1                   & 0                     & 5                         & 17                    & 6                   & 19           \\
                            & Proposta    & 0                   & 1                     & 8                         & 8                     & 10                  & 10           \\
\hline
\multirow{2}{*}{5}          & Tradicional & 3                   & 0                     & 4                         & 10                    & 10                  & 21           \\
                            & Proposta    & 0                   & 2                     & 9                         & 8                     & 6                   & 12           \\
\hline
\multirow{2}{*}{6}          & Tradicional & 3                   & 3                     & 3                         & 12                    & 8                   & 19           \\
                            & Proposta    & 0                   & 0                     & 5                         & 11                    & 10                  & 11           \\
\hline
\multirow{2}{*}{7}          & Tradicional & 2                   & 5                     & 4                         & 14                    & 9                   & 14           \\
                            & Proposta    & 1                   & 1                     & 7                         & 9                     & 7                   & 12           \\
\hline
\multirow{2}{*}{8}          & Tradicional & 0                   & 6                     & 3                         & 12                    & 9                   & 18           \\
                            & Proposta    & 0                   & 2                     & 8                         & 8                     & 6                   & 13           \\
\hline
\multirow{2}{*}{9}          & Tradicional & 2                   & 2                     & 7                         & 10                    & 7                   & 20           \\
                            & Proposta    & 0                   & 0                     & 11                        & 5                     & 8                   & 13           \\
\hline
\multirow{2}{*}{10}         & Tradicional & 1                   & 0                     & 10                        & 14                    & 4                   & 19           \\
                            & Proposta    & 0                   & 2                     & 4                         & 12                    & 8                   & 11           \\
\hline
\multirow{2}{*}{11}         & Tradicional & 2                   & 2                     & 6                         & 13                    & 5                   & 20           \\
                            & Proposta    & 0                   & 1                     & 6                         & 10                    & 8                   & 12           \\
\hline
\multirow{2}{*}{12}         & Tradicional & 4                   & 1                     & 6                         & 17                    & 2                   & 18           \\
                            & Proposta    & 0                   & 1                     & 5                         & 9                     & 10                  & 12           \\
\hline
\multirow{2}{*}{13}         & Tradicional & 2                   & 0                     & 6                         & 13                    & 12                  & 15           \\
                            & Proposta    & 0                   & 1                     & 4                         & 7                     & 14                  & 11           \\
\hline
\end{tabular}
\end{table}

Através das respostas dos alunos presentes na Tabela \ref{tab:questionario-satisfacao-respostas}, podemos perceber que
o grupo AlgoritmoTradicional tem mais respostas de ''Discordo Totalmente'' do
que o grupo AlgoritmoProposta, com 25 para o AlgoritmoTradicional e uma para o AlgoritmoProposta.
Além disso, pode-se perceber que 12 questões tiveram pelo menos um ''Discordo Totalmente'' para o grupo AlgoritmoTradicional
e uma questão para o grupo AlgoritmoProposta. Com relação a resposta ''Concordo Totalmente'',
97 respostas foram dadas pelos alunos do grupo AlgoritmoTradicional e 111
pelos alunos do grupo AlgoritmoProposta. Por outro lado, tiveram mais respostas ''Discordo Parcialmente'',
''Não Concordo e Nem Discordo'' e ''Concordo Parcialmente'' para o grupo AlgoritmoTradicional do que para o grupo
AlgoritmoProposta, com 267 para o primeiro e 217 para o segundo.

O resultado das respostas ''Não utilizei'' mostram que os alunos responderam que não utilizaram para apenas algumas
perguntas e não para todas, sendo que para o grupo AlgoritmoTradicional o valor varia entre 17 e 21 alunos e para o grupo
AlgoritmoProposta varia entre 10 e 13 alunos. O fato desse valor ter variado tanto não foi esperado inicialmente, mas pode-se
afirmar que pelo menos 17 alunos do primeiro grupo e 10 alunos do segundo grupo não utilizaram o SR. Ao comparar essa
informação com os dados de acesso dos alunos, que mostra que 23 alunos do grupo AlgoritmoTradicional e 15 alunos do
grupo AlgoritmoProposta não acessaram nenhuma recomendação, é provável que apesar de não terem acessado nenhuma recomendação os
alunos perceberam a existência do SR, analisaram as recomendações recebidas e com base nisso responderam às questões do
questionário.

As respostas dadas ao questionário foram analisadas utilizando o Teste não-paramétrico de
Mann-Whitney U para verificar se existe diferença estatisticamente significativa entre as respostas dos dois grupos (i.e., \textit{p} < 0.05).
As respostas de ''Não utilizei'' não foram consideradas nessa análise. O resultado das análises está presente no Apêndice
\ref{ape:analise-estatistica-questionario}.

A análise mostrou que a questão 12 (i.e., ''Eu entendi porque os itens foram recomendados para mim.'') apresentou
diferença significativa entre as respostas dos dois grupos, e a diferença foi favorável ao grupo AlgoritmoProposta.
Para as outras questões não foi encontrada diferença significativa entre as respostas dos
alunos dos dois grupos. A diferença existente na questão 12 mostra que os alunos do grupo AlgoritmoProposta perceberam
melhor a relação entre os conteúdos que acessaram com os links que foram recomendados para eles
do que o grupo AlgoritmoTradicional. Essa informação mostra que o uso do Decaimento na Abordagem
Baseada em Conteúdo fez com que as recomendações geradas fossem relacionadas aos conteúdos acessados recentemente,
enquanto a Abordagem Tradicional recomenda considerando todos os conteúdos acessados pelo aluno durante todo o Minicurso
igualmente.

Como pelo menos uma questão do questionário deu diferença significativa, podemos aceitar a hipótese alternativa definida
que diz que ''Há diferenças na percepção do usuário da qualidade das recomendações recebidas utilizando a abordagem
Baseada em Conteúdo tradicional e a proposta desse trabalho'' e a diferença significativa foi favorável a proposta desse trabalho.

\subsection{Questão Aberta sobre o SR}\label{subsection:questao-aberta}

Juntamente com o questionário de satisfação, foi colocada um questão aberta para alunos com a seguinte pergunta:
''Você utilizou o Sistema de Recomendação? Se sim, cite os pontos positivos e negativos desse sistema.''. Dos 85 alunos
que responderam à essa questão, 60 disseram não ter utilizado ou ter utilizado muito pouco o SR
e não citaram nenhum ponto positivo ou negativo, sendo 36 do grupo AlgoritmoTradicional
e 24 do grupo AlgoritmoProposta. Dois alunos do grupo AbordagemTradicional disseram
não ter utilizado o SR pois não perceberam a sua existência e um aluno desse mesmo grupo disse que não utilizou com tanta frequência
pois não entendeu como este funcionava. Um aluno do grupo AlgoritmoProposta disse não ter utilizado
o SR pois achou que o conteúdo do Minicurso de Algoritmos era bem completo e não viu necessidade de acessar as recomendações.

Os pontos positivos e negativos apontados pelos alunos que utilizaram o SR foram analisados utilizando a técnica de análise
e interpretação de dados qualitativos como descrito por \citeonline{creswell2014research}. O processo de análise e interpretação
é composto pelos seguintes passos: (1) Organizar e preparar os dados para análise;
(2) Ler todos os dados; (3) Codificar os dados; (4) Usar o processo de codificação para gerar descrição ou categorização
dos dados; (5) Decidir como a descrição ou categorização será representada para a narrativa qualitativa; (6) Interpretar
as descobertar ou resultados. No processo de interpretação, o autor ainda destaca que é possível citar passagens da narrativa
para transmitir os resultados da análise.

O primeiro passo, de organização e preparação dos dados, foi realizado capturando as respostas dadas pelos alunos que disseram
ter utilizado o SR e organizando em uma planilha de dados, separando as respostas pelo grupo ao qual o aluno fazia parte
(AlgoritmoTradicional ou AlgoritmoProposta). Na sequência, todos dados foram lidos e vários códigos (categorias) foram
propostos com base nas respostas dos alunos (Passos dois e três). A partir dessas categorias, o quarto passo foi ler as
respostas dos alunos e contar quantas vezes cada uma das categorias propostas apareceram nas respostas dos alunos de
cada grupo, também chamado de etiquetagem. A representação da categorização foi realizada utilizando a Tabela
\ref{ref:categorizacao-questao-aberta} (quinto passo).

\begin{table}[h]
\footnotesize
\caption{Categorização da questão aberta do questionário}
\label{ref:categorizacao-questao-aberta}
\centering
\begin{tabular}{|p{3cm}|p{10cm}|p{2cm}|}
\hline
\multicolumn{2}{|c|}{\textbf{AlgoritmoTradicional}}                                           & \textbf{Frequência} \\
\hline
\multirow{6}{*}{\textbf{Pontos positivos}} & Conteúdo das recomendações                                     & 1          \\
                          & Layout simples e intuitivo                                     & 1          \\
                          & Guiar os estudos                                               & 2          \\
                          & Funciona corretamente                                          & 1          \\
                          & Ajudou a aprofundar alguns conteúdos                           & 2          \\
                          & Recomendações adequadas ao conteúdo estudado                   & 2          \\
\hline
\textbf{Pontos negativos} & Recomendações repetidas                                        & 2          \\
\hline
\multicolumn{2}{|c|}{\textbf{AlgoritmoProposta}}                                              & \textbf{Frequência} \\
\hline
\multirow{6}{*}{\textbf{Pontos positivos}} & Encontrar materiais para estudar                               & 2          \\
                          & Recomendações adequadas ao conteúdo estudado                   & 3          \\
                          & Ajudou a engajar mais nos estudos                              & 2          \\
                          & Funciona corretamente                                          & 1          \\
                          & Recomendações são boas                                         & 2          \\
                          & Guiar os estudos                                               & 1          \\
\hline
\multirow{3}{*}{\textbf{Pontos negativos}} & Recomendações repetidas                                        & 2          \\
                          & Difícil de encontrar recomendações do meu interesse            & 1          \\
                          & Possível encontrar materiais similares com uma busca no Google & 1          \\
\hline
\end{tabular}
\end{table}

A partir da Tabela \ref{ref:categorizacao-questao-aberta} é possível realizar o último passo que é a análise e interpretação
dos resultados. Pode-se perceber pontos positivos em comum nos dois grupos, como ''Guiar os estudos'' e ''Recomendações adequadas ao conteúdo estudado''.
Com relação a esses códigos, algumas respostas do grupo AlgoritmoTradicional foram ''ajudou bastante pois foi me passando um conhecimento sequencial'',
''atende às necessidades, recomendando sempre o que convém'' e ''lhe coloca conteúdo condizente ao seu nível''.
Já dos alunos do grupo AlgoritmoProposta foram ''achei adequado ao andamento da disciplina'',
''ajuda em uma melhor e mais eficiente construção do conhecimento'' e ''sempre trouxe sugestões apropriadas''.

Também é possível perceber que o ponto negativo ''Recomendações repetidas'' aparecem nos dois grupos. Uma resposta do grupo
AlgoritmoTradicional com esse código foi ''Se repete demais em muitos conceitos'', enquanto do grupo AlgoritmoProposta
foi ''talvez eles pudessem se renovar mais a cada conteúdo''. Por essas respostas podemos afirmar que, mesmo
as respostas com esse código tendo aparecido nos dois grupos, o aluno do grupo que utilizou a abordagem Baseada em Conteúdo
tradicional aparentou estar mais decepcionado com o SR do que o aluno do grupo que utilizou a abordagem com Decaimento.
Sendo que a segunda resposta demonstrou mais uma sugestão de renovar mais o conteúdo do que uma crítica ao SR.

Dois pontos positivos que apareceram nas respostas dos alunos do grupo AlgoritmoTradicional não se referem diretamente
ao SR, que são ''Conteúdo das recomendações'' e ''Layout simples e intuitivo''. Essas respostas se referem ao conteúdo
dos links recomendados, que o aluno achou interessante por serem artigos e que ele podia baixar esse conteúdo para estudar
\textit{offline}, e à interface do SR, que era a mesma para os dois grupos.

Os principais destaques do AlgoritmoTradicional foram nas respostas de ''Guiar os estudos'',
''Ajudou a aprofundar alguns conteúdos'' e ''Recomendações adequadas ao conteúdo estudado'',
com um total de seis respostas. E o único ponto negativo citado pelo alunos desse grupo foi o ponto já mencionado
anteriormente de muitas recomendações serem repetidas.

Os principais destaques do AlgoritmoProposta foram nas respostas de ''Encontrar materiais para estudar'',
''Recomendações adequadas ao conteúdo estudado'', ''Recomendações são boas'' e ''Ajudou a engajar mais nos estudos'',
com um total de nove respostas. Por outro lado, alguns alunos desse grupo também citaram alguns pontos negativos para o
SR, que são a dificuldade de encontrar recomendações interessantes e que seria mais fácil procurar pela resposta de
suas dúvidas no Google que tentar encontrar uma recomendação do SR para isso, além do ponto negativo em comum com o
AlgoritmoTradiconal de recomendações repetidas.

Em geral, pode-se observar que o AlgoritmoProposta teve um maior número de respostas com pontos positivos (11 para
o AlgoritmoProposta e 9 para AlgoritmoTradicional). Além disso, foi notado uma diferença na forma como os alunos dos
dois grupos relataram as  recomendações repetidas que receberam, que demonstraram uma maior decepção pelos alunos
do grupo AlgoritmoTradicional. Outro ponto relevante é o fato de dois alunos do grupo AlgoritmoTradicional não lembrarem,
no momento de responder ao questionário, da existência do SR. Isso desmonstra que, mesmo as recomendações estando bem a
vista dos alunos através da interface proposta, eles provavelmente não ficaram interessados em nenhum momento pelos
Links recomendados, já que a interface para os dois grupos era a mesma e o mesmo não apareceu nas respostas do grupo
AlgoritmoProposta.

\subsection{Limitações e Ameaças a Validade do Experimento}\label{subsection:ameacas-a-validade}

Nesta Seção são apresentados as limitações do experimento realizado e as ameaças que podem ter influenciado o resultado
do experimento e poderiam invalidá-lo.

O tempo de execução do experimento (i.e., pouco menos de um mês) foi uma limitação do experimento, que não permitiu que
os alunos tivessem mais tempo para interagir com o Minicurso e com o SR. Outra limitação foi o tema do Minicurso sobre
Algoritmos e Linguagem de Programação, que é um tema fundamental e com uma ementa rigorosamente definida, enquanto um curso
sobre outro tema mais exploratório ou progressivo como Desenvolvimento \textit{Web} ou Computação Quântica poderia se aproveitar
mais da ferramenta de recomendação. Apesar da quantidade de alunos que se matriculou no Minicurso ter sido considerável,
o tamanho da amostra que efetivamente utilizou o Minicurso (145 alunos) e principalmente que efetivamente utilizou o Sistema de
Recomendação (85 alunos) é outra limitação do experimento, que poderia ter resultados mais conclusivos se mais alunos tivessem
utilizado a ferramenta de recomendação.

A qualidade dos Links de Apoio cadastrados no ambiente \adaptwebspace é uma ameaça a validade do experimento, pois podem
influenciar a percepção dos alunos sobre o SR positiva ou negativamente. Além disso, os Links de Apoio foram encontrados
por pessoas que não são professores da disciplina de Algoritmos e podem não serem totalmente adequados para alunos do Minicurso.
A representação dos Links de Apoio e dos demais itens do Minicurso através do Dicionário de Palavras criado também
podem ser uma ameaça à validade do resultado desse experimento, pois influenciaram diretamente nas recomendações geradas
para os alunos.

O questionário sobre a percepção dos alunos sobre a qualidade das recomendações recebidas ser aplicado apenas ao final
do experimento é outra ameaça à validade do experimento, pois muitos alunos podem não lembrar das recomendações recebidas
ou do SR em geral após um longo período do seu uso. Esse pode ser um dos motivos de não ter sido capturado através do
questionário o mesmo que foi observado no teste piloto, de que a proposta desse trabalho teria ajudado a diminuir o
problema da Superespecialização e que não repeteria tanto quanto na abordagem Tradicional os itens recomendados.

O interesse e engajamento dos alunos no Minicurso pode ter influenciado também o resultado do experimento, sendo que no momento
da divisão dos alunos nos dois grupos não era possível prever essa variável interveniente e pode acontecer de alunos em
um dos grupos terem um interesse e engajamento muito maior que os alunos do outro grupo.

Como último fator que pode ter influenciado o experimento e ser uma ameaça a validade do mesmo está a grande variância
existente entre os valores das variáveis dependentes do experimento, e.g., a quantidade de recomendações acessadas que variou
de 1 à 107 recomendações. Esse é um dos fatores pelo qual, apesar de a proposta desse trabalho ter uma média maior em
muitas das variáveis medidas, a maioria delas não apresentou diferença significativa.

\section{Considerações sobre o capítulo}

Neste capítulo foi descrito o ambiente no qual o Sistema de Recomendação (SR) proposto será incorporado (\adaptweb) e definido o
experimento para a avaliação da proposta desse trabalho. Nos SRs desenvolvidos no \adaptweb, tanto para a proposta
deste trabalho como para a abordagem Baseada em Conteúdo Tradicional que será utilizada como parâmetro, o perfil do
usuário é composto pelo conjunto de palavras-chave de todos os itens acessados pelo usuário, das
categorias ''Conceito'', ''Materiais Complementares'' e ''Links de Apoio''. Os itens a serem recomendados são os
Links de Apoio, visto que os itens das outras categorias são estruturados pelo professor para serem acessados em determinada
ordem e estão fortemente relacionados à conceitos específicos.

Utilizando como base as diretrizes propostas por \citeonline{pu2012evaluating} foi proposta uma interface para a
apresentação das recomendações no \adaptweb, com o objetivo de que os usuários tenham acesso direto as recomendações na
tela principal do ambiente do aluno e entendam melhor o porquê daqueles itens serem recomendados. Essa interface foi
utilizada tanto pelo grupo AlgoritmoTradicional quanto pelo AlgoritmoProposta.

O experimento realizado tem por objetivo avaliar a experiência do usuário com o SR proposto no Capítulo \ref{chapter:proposta}
(AlgoritmoProposta) em comparação à abordagem Baseada em Conteúdo tradicional (AlgoritmoTradicional). O experimento foi
realizado no ambiente \adaptwebspace através de um Minicurso de Algoritmos e Linguagem de Programação desenvolvido no
ambiente nos meses de Abril e Maio de 2018. Ao final do minicurso os alunos receberam um questionário para responder
sobre a sua experiência ao interagir com o SR, que foi adaptado do conjunto de questões definidas por
\citeonline{pu2011user}. No total, 208 alunos se matricularam no Minicurso, sendo que 85 alunos chegaram até o final e
responderam ao questionário.

Foram analisados os dados de acesso e avaliações das recomendações pelos alunos e da geração de recomendações pelos dois
algroitmos. Na análise da Cobertura e do F-measure o AlgoritmoProposta teve resultado estatisticamente melhor que
o AlgoritmoTradicional. Isso mostra que a proposta desse trabalho ajudou a diminuir o problema da Superespecialização
presente na Abordagem Baseada em Conteúdo Tradicional, aumentando a Cobertura do método sem impactar negativamente na
Precisão e na  Percepção do Usuário. Para as outras métricas, como quantidade de links acessados e precisão do
algoritmo, não existe diferença significativa entre os resultados dos dois grupos.

Ao analisar o questionário utilizando-se de técnicas estatísticas descritas na Seção \ref{subsection:tecnicas-analise-estatistica},
o resultado foi que proposta desse trabalho apresentou melhor resultado na questão 12, na qual dizia
''Eu entendi porque os itens foram recomendados para mim''. Isso mostra que os alunos puderam perceber uma melhor relação
entre os conteúdos acessados com os itens recebidos como recomedação pelos alunos que utilizaram a proposta desse trabalho
do que pelos alunos que utlizaram a abordagem Tradicional de recomendação. Sobre as outras 12 questões do questionário
não é possível realizar nenhum tipo de afirmação pois não existe diferença significativa entre os resultados
alcançados pelos dois grupos.

Na análise da questão aberta do questionário, foi aplicado uma técnica de análise e interpretação de dados qualitativos
para identificar os principais pontos positivos e negativos. Ao final do processo podemos afirmar que os dois algoritmos
de recomendação tiveram algumas respostas em comum, tanto em pontos positivos (e.g., ''Guiar o aluno'') como em pontos
negativos (e.g., ''Recomendações repetidas''). Porém, a proposta desse trabalho teve mais pontos positivos apontados,
com pontos que indicaram a importância do SR para esses alunos, como ''Ajudou a engajar mais nos estudos'' e ''Recomendações são boas''.

Como resultado final do experimento podemos aceitar as duas hipóteses alternativas, confirmando que existe diferença significativa
tanto no desempenho do algoritmo de recomendação quanto na percepção do usuário sobre as recomendações recebidas
entre o AlgoritmoTradicional e o AlgoritmoProposta. Parece essas duas hipóteses, o resultado foi favorável a proposta desse trabalho
e portanto pode-se afirmar que a aplicação do Decaimento teve um melhor resultado nos dois aspectos medidos em relação
a Abordagem Baseada em Conteúdo Tradicional.

\chapter{Considerações Finais}\label{chapter:conclusoes}

Sistemas de Recomendação (SR) são ferramentas de software que sugerem itens para os usuários de forma automatizada e personalizada,
sem a necessidade do usuário formular uma consulta para encontrar os itens do seu interesse. Esses sistemas são
explorados em Ambientes Virtuais de Aprendizagem (AVA) com o objetivo de reduzir alguns problemas existentes nesses ambientes
quando a quantidade de materiais disponíveis é grande, tais como: sobrecarga cognitiva, dificuldade de encontrar os materiais
do seu interesse e muitos materiais nunca serem utilizados.

Pesquisadores da área argumentam que os algoritmos de SRs tradicionais não são suficientes para os AVAs \cite{verbert2012context, drachsler2015panorama},
sendo necessário um nível maior de personalização a situação do usuário, como considerar dimensões do contexto. Para isso,
em \citeonline{de2017time} foi realizado um mapeamento sistemático da literatura com o objetivo de identificar como os SRs Sensíveis
ao Contexto Temporal (também chamados de SR Sensíveis ao Tempo) são utilizados. Nesse estudo, foram considerados todos os domínios
de aplicação e não apenas o domínio educacional.

Foi observado que, dos 88 artigos que utilizam esse tipo de SR, apenas quatro são aplicados no domínio educacional e esses
trabalhos carecem em avaliações em ambientes reais de uso ou que utilizem bases de dados educacionais. Analisando esses
88 artigos, também foi possível categorizar os SRs propostos nesses trabalhos pela forma que eles utilizam o tempo para
a recomendação. As sete categorias criadas como resultado do mapeamento são apresentadas na Seção \ref{section:sr-sensivel-tempo}.

O objetivo desse trabalho é a criação de perfis de usuário que levem em conta a mudança dos interesses destes usuários
ao longo do tempo. Esses perfis considerando o contexto temporal serão aplicados no algoritmo de recomendação proposto nesse
trabalho. Dentre as categorias de SR Sensíveis ao Tempo presentes na Seção \ref{section:sr-sensivel-tempo}, a proposta
desse trabalho se encaixa no \textit{Decay}.

O algoritmo proposto no Capítulo \ref{chapter:proposta} combina a (1) similaridade do perfil do usuário (representado
pelos materiais acessados pelo usuário) com os itens disponíveis para a recomendação com a (2) recência dos materiais
acessados pelo usuários, além da (3) informação se aquele item disponível para a recomendação já foi acessado ou não. A
proposta leva em conta que o ritmo de estudo dos alunos pode ser diferente, portanto a recência é considerada em relação
a sequência de itens acessados e não ao tempo absoluto (em segundos) desde o acesso. Dessa forma, para cada aluno o
decaimento acaba sendo personalizado ao seu ritmo de estudo. Também é considerado que itens já acessados podem ser
recomendados novamente, porém esses itens tem um probabilidade menor de ser recomendados do que itens ainda não acessados.

Como continuação desse trabalho está a etapa de implementação da proposta e a experimentação utilizando um ambiente
real de uso. A proposta desse trabalho será incorporada ao ambiente \adaptwebspace e será avaliado através de um minicurso de
algoritmos que é ministrado no ambiente todo semestre, no qual participam alunos da primeira fase dos cursos
do Centro de Ciências Tecnológicas (CCT) da Universidade do Estado de Santa Catarina (UDESC). O algoritmo proposto será
comparado a abordagem Baseada em Conteúdo tradicional através de um experimento utilizando um estratégia \textit{Between Subjects}.

O objetivo do experimento é verificar se existe diferença na percepção do usuário sobre a qualidade das recomendações do
algoritmo proposto em relação a abordagem Baseada em Conteúdo tradicional. A percepção do usuário será capturada utilizando
o questionário proposto por \citeonline{pu2011user} para identificar a percepção do usuário sobre a qualidade das recomendações,
presente no Anexo \ref{ane:questoes-framework}.

\section{Outros resultados}

Como outros resultados desse trabalho tem-se as seguintes publicações:

\begin{itemize}
\item BORBA, E. J.; Gasparini, I.; LICHTNOW, D. Time-Aware Recommender Systems: A Systematic Mapping. International Conference on Human-Computer Interaction (HCI), Vancouver, Part II, LNCS 10272, v. II, p. 464-479, 2017.
\item BORBA, E. J.; GASPARINI, I.; LICHTNOW, D. The Use of Time Dimension in Recommender Systems for Learning. Proceedings of the 19th International Conference on Enterprise Information Systems (ICEIS), Porto (Portugal) 2017. v. 2. p. 600-609.
\item BORBA, E. J.; GASPARINI, I.; LICHTNOW, D. Sistema de Recomendação Sensível ao Tempo em Ambientes Educacionais. IV Workshop de Teses e Dissertações em IHC (WTD-IHC), Joinville, 2017.
\end{itemize}


% ---
% Finaliza o bookmark do PDF
% ---
\bookmarksetup{startatroot}%
% ---

% ----------------------------------------------------------
% ELEMENTOS PÓS-TEXTUAIS
% ----------------------------------------------------------
\postextual

% ----------------------------------------------------------
% Referências bibliográficas
% ----------------------------------------------------------
\bibliography{references}

% ----------------------------------------------------------
% Glossário
% ----------------------------------------------------------
%
% Consulte o manual da classe abntex2 para orientações sobre o glossário.
%
%\glossary

% ----------------------------------------------------------
% Apêndices
% ----------------------------------------------------------
\begin{apendicesenv}
  % Imprime uma página indicando o início dos apêndices
  \partapendices

  \include{Partes/apeB-teste-piloto}
  \chapter{Termo de Consentimento Livre e Esclarecido}\label{ape:termo-de-consentimento}

\section{Descrição do Minicurso}

Você está sendo convidado a participar do Minicurso de Algoritmos vinculado a um projeto de mestrado. Este projeto intitula-se “Sistema de Recomendação Sensível ao Tempo em Ambientes Virtuais de Aprendizagem” e visa descobrir se o uso da informação temporal em um Sistema de Recomendação influencia na qualidade dos itens recomendados.
Durante o minicurso, seus dados de utilização serão coletados e posteriormente analisados pelos pesquisadores envolvidos no projeto.

\section{Procedimento}
Após o período de matrícula (de 09/04/2018 a 13/04/2018) no minicurso, todos os alunos matriculados terão acesso ao conteúdo do minicurso a partir do dia 16/04/2018. Ao final do minicurso, os alunos realizarão uma avaliação final e responderão a um questionário de satisfação referente ao minicurso.
Durante o minicurso, os dados de navegação/interação dos alunos com o AdaptWeb® serão coletados para análise posterior com o objetivo de descobrir aspectos positivos e negativos das ferramentas existentes no sistema.

\section{Riscos e Desconfortos}
A participação neste minicurso não apresenta riscos diretos a seus participantes. Caso você não se sinta confortável em ter suas informações coletadas; não goste do assunto abordado, da metodologia utilizada ou do material utilizado; ou ainda por quaisquer outros motivos não deseje continuar a participar do minicurso, você está livre pra desistir a qualquer momento.

\section{Benefícios da sua Participação}
Esperamos que os resultados obtidos auxiliem a identificar os benefícios práticos do uso da informação temporal em um Sistema de Recomendação através das recomendações realizadas para os alunos durante sua interação, suas características positivas e negativas. Desta forma, esperamos contribuir com uma melhor experiência do usuário em ambientes de Educação a Distância.


\section{Custos}
Sua participação no minicurso não acarretará em nenhum custo. Você também não será pago(a) para participar.

\section{Confidencialidade}
Sua identidade será preservada, pois você será referenciado por um identificador numérico, de forma que seu nome nunca será citado. As únicas pessoas que terão acesso aos dados brutos serão as pesquisadoras envolvidas no projeto: Eduardo José de Borba, profa. Dra. Isabela Gasparini e prof. Dr. Daniel Lichtnow. Os resultados, sem identificações, poderão ser veiculados em artigos técnicos e científicos.

\section{Dúvidas}
Caso haja qualquer dúvida a respeito do minicurso, sinta-se à vontade para entrar em contato.

\begin{quote}
\textbf{Eduardo José de Borba} (Aluno do Programa de Pós-Graduação em Computação Aplicada da Universidade do Estado de Santa Catarina)\\
E-mail: eduardojoseborba@gmail.com\\
\\
\textbf{Dra. Isabela Gasparini} (Orientadora)\\
E-mail: isabela.gasparini@udesc.br\\
\\
\textbf{Dr. Daniel Lichtnow} (Coorientador)\\
E-mail: dlichtnow@politecnico.ufsm.br\\
\\
\textbf{Endereço para contato:}\\
Departamento de Ciência da Computação (DCC)\\
Centro de Ciências Tecnológicas (CCT)\\
Rua Paulo Malschitzki, 200 - Campus Universitário Prof. Avelino Marcante - Bairro Zona Industrial Norte\\
Joinville - SC - Brasil\\
\end{quote}


\centerline{\textbf{Termo de Consentimento Livre e Esclarecido}}

Solicitamos a sua permissão para utilizarmos os dados coletados, bem como para divulgar os resultados em artigos técnicos e científicos. Lembramos que iremos garantir sua privacidade.
Destacamos que este estudo visa avaliar a ferramenta e não os participantes. Nós queremos saber a sua opinião!

\begin{todolist}
\item Declaro que fui informado sobre todos os procedimentos da pesquisa e, que recebi de forma clara e objetiva todas as explicações pertinentes ao projeto e, que todos os dados coletados serão sigilosos. Eu compreendo que neste estudo, as medições dos experimentos/procedimentos de tratamento serão feitas sobre minhas ações no sistema.
\item Declaro que meu responsável está ciente que estou participando deste minicurso, que dados sobre mim estão sendo coletados e que minha identidade será preservada.
\end{todolist}

Nome do responsável:

E-mail para contato:



  \chapter{Questões selecionadas do Framework Resque e traduzidas}\label{ape:questionario-de-satisfacao}

\section{QUALITY OF RECOMMENDED ITEMS}
\subsection{Accuracy}
\begin{itemize}
\item \textbf{Questão 1:} Os itens recomendados corresponderam com os meus interesses.
\end{itemize}
\subsection{Diversity}
\begin{itemize}
\item \textbf{Questão 2:} Os itens recomendados para mim são diversificados (o sistema se preocupa em trazer itens diferentes a cada recomendação).
\end{itemize}
\subsection{Context Compatibility}
\begin{itemize}
\item \textbf{Questão 3:} Os itens recomendados corresponderam aos  interesses e necessidades que eu tinha no momento.
\item \textbf{Questão 4:} As recomendações são feitas no momento adequado.
\end{itemize}
\section{INTERACTION ADEQUACY}
\begin{itemize}
\item \textbf{Questão 5:} O sistema de recomendação explica porque os links são recomendados para mim.
\end{itemize}
\section{INTERFACE ADEQUACY}
\begin{itemize}
\item \textbf{Questão 6:} A informação apresentada na interface para os itens recomendados é suficiente para mim.
\item \textbf{Questão 7:} O layout do sistema de recomendação é atrativo e adequado.
\end{itemize}
\section{PERCEIVED EASE OF USE}
\subsection{Ease of Initial Learning}
\begin{itemize}
\item \textbf{Questão 8:} Eu encontrei facilmente o local onde os itens são recomendados.
\end{itemize}
\subsection{Ease of Preference Revision}
\begin{itemize}
\item \textbf{Questão 9:} Eu percebi que o sistema de recomendação aprendia sobre minhas necessidades/preferências conforme eu avançava na disciplina.
\end{itemize}
\subsection{Ease of Decision Making}
\begin{itemize}
\item \textbf{Questão 10:} É facil encontrar um item para estudar com a ajuda do sistema de recomendação.
\end{itemize}
\section{PERCEIVED USEFULNESS}
\begin{itemize}
\item \textbf{Questão 11:} Eu me senti apoiado para encontrar itens do meu interesse com a ajuda do sistema de recomendação.
\end{itemize}
\section{CONTROL/TRANSPARENCY}
\begin{itemize}
\item \textbf{Questão 12:} Eu entendi porque os itens foram recomendados para mim.
\end{itemize}
\section{ATTITUDES}
\begin{itemize}
\item \textbf{Questão 13:} No geral, estou satisfeito com o sistema de recomendação.
\end{itemize}

  \include{Partes/apeE-dicionario}
  \chapter{Palavras-chave dos materiais no Minicurso de Algotimos e Linguagem de Programação}\label{ape:dicionario-palavras-chave}

\section{Conceitos}

\begin{longtable}{| p{.10\textwidth} | p{.45\textwidth} | p{.45\textwidth} |}
\hline
Conceito & Nome                                                     & Palavras-chave                                                                                                                \\ \hline
1        & Seja bem-vindo ao minicurso de Algoritmos e Programação! & Algoritmo                                                                                                                     \\ \hline
1.1      & Objetivo do minicurso                                    & Algoritmo                                                                                                                     \\ \hline
1.2      & Ferramentas de aprendizagem                              & Algoritmo                                                                                                                     \\ \hline
1.3      & O que você vai ver nesse curso                           & Algoritmo                                                                                                                     \\ \hline
1.4      & Como estudar                                             & Algoritmo                                                                                                                     \\ \hline
2        & Da lógica à programação                                  & Logica de programacao, Instrucao, Algoritmo, Programa de computador                                                           \\ \hline
2.1      & Por que aprender a programar?                            & Logica de programacao, Problema                                                                                               \\ \hline
2.2      & Aplicação de Algoritmos                                  & Logica de programacao, Aplicacao de algoritmo                                                                                 \\ \hline
2.3      & Algoritmos e Programação                                 & Programa de computador, Instrucao, Linguagem de programacao                                                                   \\ \hline
3        & Interpretadores e compiladores                           & Interpretador, Compilador, Linguagem de maquina                                                                               \\ \hline
3.1      & Interpretadores de algoritmos (Portugol IDE)             & Interpretador, Portugol                                                                                                       \\ \hline
3.2      & Compiladores de linguagem                                & Compilador, Linguagem C                                                                                                       \\ \hline
3.3      & Outras ferramentas de aprendizagem                       & Sintaxe, Semantica                                                                                                            \\ \hline
4        & Algoritmos                                               & Algoritmo, Instrucao, Entrada, Processamento, Saida                                                                           \\ \hline
4.1      & Criando um algoritmo                                     & Problema, Algoritmo, Entrada, Processamento, Saida                                                                            \\ \hline
4.2      & Tipos de algoritmo                                       & Tipos de algoritmo, Narrativa, Fluxograma, Pseudocodigo, Diagrama de Chapin                                                   \\ \hline
5        & Sobre narrativas                                         & Narrativa, Logica de programacao, Linguagem natural                                                                           \\ \hline
5.1      & Tipos de narrativa                                       & Narrativa, Estrutura sequencial, Estrutura condicional, Estrutura de repeticao                                                \\ \hline
5.2      & Exercitando as narrativas                                & Narrativa, Algoritmo, Estrutura condicional                                                                                   \\ \hline
6        & Sobre pseudocódigos                                      & Pseudocodigo, Linguagem natural, Linguagem de programacao                                                                     \\ \hline
6.1      & Estrutura de um pseudocódigo                             & Pseudocodigo, Comentario, Variaveis, Semantica                                                                                \\ \hline
6.2      & Predefinições para escrita de pseudocódigos              & Pseudocodigo, Sintaxe, Variaveis, Constantes, Tipos de dados, Tipo inteiro, Tipo real, Tipo caracter, Tipo texto, Tipo logico \\ \hline
7        & O que é um programa?                                     & Programa de computador, Linguagem de programacao, Linguagem de maquina                                                        \\ \hline
7.1      & Estrutura de um programa                                 & Programa de computador, Sintaxe, Semantica, Instrução, Estrutura sequencial, Estrutura condicional, Estrutura de repeticao    \\ \hline
7.2      & Linguagem C - Principais conceitos                       & Linguagem C, Compilador                                                                                                       \\ \hline
7.2.1    & main()                                                   & Linguagem C                                                                                                                   \\ \hline
7.2.2    & system()                                                 & Linguagem C                                                                                                                   \\ \hline
7.2.3    & \#include                                                & Linguagem C, Compilador                                                                                                       \\ \hline
7.3      & Constantes e comandos de atribuição                      & Constantes, Variaveis, Atribuicao, Tipo inteiro, Tipo caracter, Tipo texto                                                    \\ \hline
7.4      & Comandos de Entrada-Saída                                & Saida, Entrada, Tipo texto                                                                                                    \\ \hline
7.4.1    & printf                                                   & Linguagem C, Saida                                                                                                            \\ \hline
7.4.2    & scanf                                                    & Linguagem C, Entrada, Tipo inteiro, Tipo real, Tipo texto, Tipo caracter                                                      \\ \hline
7.5      & Strings                                                  & Linguagem C, Tipo alfanumerico, Tipo caracter, Operadores, Atribuicao                                                         \\ \hline
8        & Dados, variáveis e operadores                            & Variaveis, Operadores, Precedencia de operadores                                                                              \\ \hline
8.1      & Dados                                                    & Tipos de dados, Tipo inteiro, Tipo real, Tipo caracter, Tipo texto, Tipo logico                                               \\ \hline
8.2      & Operadores básicos                                       & Operadores, Operadores aritmeticos                                                                                            \\ \hline
8.3      & Operadores aritméticos em C                              & Operadores, Operadores aritmeticos, Linguagem C, Precedencia de operadores                                                    \\ \hline
8.4      & Operadores relacionais                                   & Operadores, Operadores relacionais, Precedencia de operadores                                                                 \\ \hline
8.5      & Operadores lógicos                                       & Operadores, Operadores logicos, Precedencia de operadores                                                                     \\ \hline
8.6      & Precedência entre todos os Operadores                    & Operadores, Precedencia de operadores                                                                                         \\ \hline
9        & Estrutura de Controle - Decisão                          & Estrutura condicional                                                                                                         \\ \hline
9.1      & Estrutura de decisão simples                             & Estrutura condicional, Decisao simples, Se                                                                                    \\ \hline
9.2      & Estrutura de decisão composta                            & Estrutura condicional, Decisao composta, Se                                                                                   \\ \hline
9.3      & Estrutura de decisão encadeada                           & Estrutura condicional, Decisao encadeada, Escolha                                                                             \\ \hline
10       & Estrutura de controle – Laços (repetição)                & Estrutura de repeticao                                                                                                        \\ \hline
10.1     & Laço while (enquanto)                                    & Estrutura de repeticao, Enquanto                                                                                              \\ \hline
10.2     & Laço do...while (repita)                                 & Estrutura de repeticao, Repita                                                                                                \\ \hline
10.3     & Laço para (for)                                          & Estrutura de repeticao, Para                                                                                                  \\ \hline
11       & Vetores e matrizes                                       & Variaveis, Matriz, Vetor                                                                                                      \\ \hline
11.1     & Vetores                                                  & Variaveis, Vetor                                                                                                              \\ \hline
11.2     & Matrizes                                                 & Variaveis, Matriz \\ \hline
\caption{Palavras-chave associadas aos Conceitos}
\label{tab:palavras-chave-conceitos}
\end{longtable}

\section{Materiais Complementares}

\begin{longtable}{| p{.20\textwidth} | p{.40\textwidth} | p{.40\textwidth} |}
\hline
Conceito relacionado & Nome                                                       & Palavras-chave                                                                                         \\ \hline
1                    & Seja Bem-Vindo                                             & Algoritmo                                                                                              \\ \hline
1                    & Créditos                                                   & Algoritmo                                                                                              \\ \hline
2                    & Podcast do Papo BJPnet                                     & Programa de computador, Linguagem de programacao, Linguagem de maquina                                 \\ \hline
2                    & Sugestão - Jogo Light-bot                                  & Logica de programacao, Problema                                                                        \\ \hline
3.3                  & Sugestão - Outros compiladores de linguagem C              & Linguagem C, Compilador                                                                                \\ \hline
3.3                  & Outros interpretadores de pseudolinguagem                  & Portugol, Interpretador                                                                                \\ \hline
4.1                  & Passos para criação de um algoritmo                        & Problema, Algoritmo, Entrada, Processamento, Saida                                                     \\ \hline
5.1                  & Narrativa Sequencial                                       & Narrativa, Estrutura sequencial                                                                        \\ \hline
5.1                  & Narrativa de Seleção                                       & Narrativa, Estrutura condicional                                                                       \\ \hline
5.1                  & Narrativa de repetição                                     & Narrativa, Estrutura de repeticao                                                                      \\ \hline
7.1                  & Pseudocódigo: Alô mundo                                    & Portugol                                                                                               \\ \hline
7.1                  & Linguagem C: Alô mundo                                     & Linguagem C                                                                                            \\ \hline
7.1                  & Comparativo das estruturas básicas de pseudocódigo e C     & Portugol, Linguagem C                                                                                  \\ \hline
7.3                  & Comando de atribuição em pseudocódigos                     & Portugol, Atribuicao                                                                                   \\ \hline
7.4                  & Comparativo Pseudocódigo x Linguagem C - Comandos de Saída & Saida, Portugol, Linguagem C                                                                           \\ \hline
7.4                  & Tabela de códigos especiais em C                           & Linguagem C, Entrada, Saida, Tipo texto                                                                \\ \hline
7.4                  & Tabela códigos de formatação em C                          & Linguagem C, Entrada, Saida, Tipo texto                                                                \\ \hline
7.4.2                & Pseudocódigo - Função de entrada                           & Portugol, Entrada                                                                                      \\ \hline
7.4.2                & Função de entrada                                          & Linguagem C, Entrada                                                                                   \\ \hline
7.4.2                & Pseudocódigo - Entrada de dados                            & Portugol, Entrada                                                                                      \\ \hline
7.4.2                & Entrada de dados                                           & Linguagem C, Entrada                                                                                   \\ \hline
7.5                  & Tabela das funções de String                               & Linguagem C, Tipo texto                                                                                \\ \hline
8.1                  & Tabela - Pseudo x C - Tipos de dados                       & Portugol, Linguagem C, Tipos de dados, Tipo inteiro, Tipo real, Tipo caracter, Tipo logico, Tipo texto \\ \hline
8.1                  & Pseudocódigo - Atribuições com variáveis                   & Portugol, Variaveis, Atribuicao, Matriz, Tipo inteiro, Tipo texto, Tipo logico, Tipo real              \\ \hline
8.1                  & Atribuições com variáveis                                  & Linguagem C, Variaveis, Atribuicao, Matriz, Tipo inteiro, Tipo texto, Tipo logico, Tipo real           \\ \hline
8.4                  & Pseudo x C - Tabela de operadores relacionais              & Linguagem C, Operadores relacionais                                                                    \\ \hline
11.1                 & Declaracao de vetores                                      & Linguagem C, Vetor, Atribuicao, Entrada                                                                \\ \hline
11.2                 & Declaração de matrizes                                     & Linguagem C, Matriz, Atribuicao \\ \hline
\caption{Palavras-chave associadas aos Materiais Complementares}
\label{tab:palavras-chave-materiais-complementares}
\end{longtable}


\section{Links de Apoio}

\begin{longtable}{| p{.10\textwidth} | p{.45\textwidth} | p{.45\textwidth} |}
\hline
\#  & Link de Apoio                                                                                                                                                                                                                & Palavras-chave associadas                                                                                                                                  \\ \hline
1   & \href{http://academicotech.blogspot.com.br/2014/02/v-behaviorurldefaultvmlo.html}{\color{blue} Lógica de Programação - Fluxograma e Portugol                                       } & Logica de programacao, Fluxograma, Portugol, Pseudocodigo, Operadores, Blocos, Tipos de dados, Variaveis,                                         \\ \hline
2   & \href{http://blog.academiadocodigo.com.br/2014/12/macas-e-laranjas-diferencas-entre-compilador-e-interpretator/}{\color{blue} Maçãs e laranjas: diferenças entre compilador e interpretator                       } & Compilador, Interpretador                                                                                                                         \\ \hline
3   & \href{http://blog.triadworks.com.br/por-que-aprender-a-programar}{\color{blue} Por que aprender a programar?                                                       } & Aplicacao de algoritmo, Programa de computador                                                                                                    \\ \hline
4   & \href{http://br.ccm.net/faq/9709-algoritmo-definicao-e-introducao}{\color{blue} Algoritmo: Definição e introdução                                                   } & Algoritmo, Linguagem de maquina, Compilador, Linguagem de programacao                                                                             \\ \hline
5   & \href{http://coral.ufsm.br/pet-si/index.php/os-beneficios-e-o-porque-de-aprender-a-programar/}{\color{blue} Os Benefícios e o Porquê de Aprender a Programar.                                   } & Aplicacao de algoritmo, Algoritmo, Instrucao, Programa de computador                                                                              \\ \hline
6   & \href{http://dcm.ffclrp.usp.br/~augusto/teaching/ici/Vetores-Matrizes.pdf}{\color{blue} Vetores e Matrizes                                                                  } & Vetor, Matriz, Contantes, Entrada, Saida                                                                                                          \\ \hline
7   & \href{http://download2.nust.na/pub4/sourceforge/v/vi/visualg30/IP\_03\_VisuALG\_Repeticao.pdf}{\color{blue} VisuALG - estrutura de repetição                                                    } & Estrutura de repetição, Enquanto, Para, Repita                                                                                                    \\ \hline
8   & \href{http://download2.nust.na/pub4/sourceforge/v/vi/visualg30/IP\_02\_VisuALG\_Basico.pdf}{\color{blue} Introdução ao VisuALG                                                               } & Portugol, Entrada, Saida, Operadores aritmeticos, Precedencia de operadores, Operadores logicos, Estrutura condicional                            \\ \hline
9   & \href{http://download2.nust.na/pub4/sourceforge/v/vi/visualg30/IP\_04\_VisuALG\_Arrays.pdf}{\color{blue} VisuALG – Arrays e Strings                                                          } & Portugol, Vetor, Tipo texto                                                                                                                       \\ \hline
10  & \href{http://eletrica.ufpr.br/~rogerio/visualg/Help/linguagem.htm}{\color{blue} A Linguagem de Programação do VisuAlg                                               } & Portugol, Pseudocodigo, Tipos de dados, Variaveis, Constantes                                                                                     \\ \hline
11  & \href{http://fabrica.ms.senac.br/2013/06/algoritmo-estrutura-de-vetores-e-matrizes/}{\color{blue} Algoritmo: Estrutura de vetores e matrizes.                                         } & Portugol, Entrada, Saida, Vetor, Matriz                                                                                                           \\ \hline
12  & \href{http://fig.if.usp.br/~esdobay/c/c.pdf}{\color{blue} Programação em C                                                                    } & Linguagem C, Variaveis, Constantes, Operadores, Vetor, Matriz                                                                                     \\ \hline
13  & \href{http://knoow.net/ciencinformtelec/informatica/linguagem-maquina/}{\color{blue} Linguagem máquina                                                                   } & Linguagem de maquina                                                                                                                              \\ \hline
14  & \href{http://linguagemc.com.br/estruturas-de-decisao-encadeadas-if-else-if-else/}{\color{blue} Estruturas de decisão encadeadas – if – else – if – else                            } & Estrutura condicional, Decisao Encadeada, Se                                                                                                      \\ \hline
15  & \href{http://linguagemc.com.br/loop-infinito-em-c/}{\color{blue} Loop infinito em C                                                                  } & Linguagem C, Estrutura de repeticao, Para, Repita                                                                                                 \\ \hline
16  & \href{http://marmsx.msxall.com/cursos/c3.html}{\color{blue} Começando a programar                                                               } & Linguagem C, Sintaxe, Semantica                                                                                                                   \\ \hline
17  & \href{http://nerdsti.com.br/?p=259}{\color{blue} Lógica de Programação - Vetores e Matrizes                                          } & Vetor, Matriz, Sintaxe                                                                                                                            \\ \hline
18  & \href{http://wiki.icmc.usp.br/images/5/57/Estruturas\_Controle\_I\_SCC0120\_v2.pdf}{\color{blue} Introdução à Ciência da Computação - Estruturas de Controle – Parte I               } & Portugol, Estrutura condicional, Decisao simples, Decisao composta, Decisao encadeada                                                             \\ \hline
19  & \href{http://www.berriel.com.br/ltpi/aula01/aula01.htm}{\color{blue} Conceito e formas de representação de algoritmos                                    } & Algoritmo, Tipos de algoritmo, Narrativa, Fluxograma, Pseudocodigo                                                                                \\ \hline
20  & \href{http://www.bosontreinamentos.com.br/logica-de-programacao/12-logica-de-programacao-desvio-condicional-aninhado-se-entao-senao-se/}{\color{blue} Lógica de Programação – Desvio Condicional Aninhado (SE…ENTÃO…SENÃO…SE)             } & Portugol, Estrutura condicional, Decisao composta, Se                                                                                             \\ \hline
21  & \href{http://www.cprogressivo.net/2013/03/O-que-e-alocacao-dinamica-de-memoria-em-Linguagem-C.html}{\color{blue} O que é e para que serve alocação dinâmica                                          } & Linguagem C, Vetor                                                                                                                                \\ \hline
22  & \href{http://www.cprogressivo.net/p/arquivos-em-c.html}{\color{blue} Arquivos em c - tutorial completo                                                   } & Linguagem C                                                                                                                                       \\ \hline
23  & \href{http://www.cristiancechinel.pro.br/my\_files/algorithms/bookhtml/node18.html}{\color{blue} Linguagem Natural                                                                   } & Linguagem natural                                                                                                                                 \\ \hline
24  & \href{http://www.cristiancechinel.pro.br/my\_files/algorithms/bookhtml/node19.html}{\color{blue} Linguagem de Máquina e Assembler                                                    } & Linguagem de maquina                                                                                                                              \\ \hline
25  & \href{http://www.cristiancechinel.pro.br/my\_files/algorithms/bookhtml/node43.html}{\color{blue} Expressões Lógicas                                                                  } & Operadores logicos                                                                                                                                \\ \hline
26  & \href{http://www.cristiancechinel.pro.br/my\_files/algorithms/bookhtml/node44.html}{\color{blue} Operadores Relacionais                                                              } & Operadores relacionais                                                                                                                            \\ \hline
27  & \href{http://www.dainf.cefetpr.br/~robson/prof/common/c/aspec.htm}{\color{blue} Aspectos básicos de linguagem C                                                     } & Linguagem C, Tipos de dados, Variaveis, Constantes, Entrada, Saida, Operadores                                                                    \\ \hline
28  & \href{http://www.dei.estt.ipt.pt/portugol/node/10}{\color{blue} Tipos de dados \textgreater{}\textgreater Constantes                                } & Constantes, Portugol                                                                                                                              \\ \hline
29  & \href{http://www.dei.estt.ipt.pt/portugol/node/14}{\color{blue} Entrada/Saída \textgreater{}\textgreater Ler                                        } & Entrada, Portugol                                                                                                                                 \\ \hline
30  & \href{http://www.dei.estt.ipt.pt/portugol/node/15}{\color{blue} Entrada/Saída \textgreater{}\textgreater Escrever                                   } & Saida, Portugol                                                                                                                                   \\ \hline
31  & \href{http://www.dei.estt.ipt.pt/portugol/node/17}{\color{blue} Operadores Aritméticos                                                              } & Operadores aritmeticos, Portugol                                                                                                                  \\ \hline
32  & \href{http://www.dei.estt.ipt.pt/portugol/node/21}{\color{blue} Operadores Lógicos                                                                  } & Operadores logicos, Portugol                                                                                                                      \\ \hline
33  & \href{http://www.dei.estt.ipt.pt/portugol/node/22}{\color{blue} Operadores Relacionais                                                              } & Operadores relacionais, Portugol                                                                                                                  \\ \hline
34  & \href{http://www.dei.estt.ipt.pt/portugol/node/24}{\color{blue} Decisão \textgreater{}\textgreater Se                                               } & Se, Estrutura condicional, Portugol                                                                                                               \\ \hline
35  & \href{http://www.dei.estt.ipt.pt/portugol/node/25}{\color{blue} Decisão \textgreater{}\textgreater Escolhe                                          } & Escolha, Estrutura condicional, Portugol                                                                                                          \\ \hline
36  & \href{http://www.dei.estt.ipt.pt/portugol/node/27}{\color{blue} Repetição \textgreater{}\textgreater Enquanto                                       } & Enquanto, Estrutura de repeticao, Portugol                                                                                                        \\ \hline
37  & \href{http://www.dei.estt.ipt.pt/portugol/node/28}{\color{blue} Repetição \textgreater{}\textgreater Para                                           } & Para, Estrutura de repeticao, Portugol                                                                                                            \\ \hline
38  & \href{http://www.dei.estt.ipt.pt/portugol/node/29}{\color{blue} Repetição \textgreater{}\textgreater Repete                                         } & Repita, Estrutura de repeticao, Portugol                                                                                                          \\ \hline
39  & \href{http://www.dei.estt.ipt.pt/portugol/node/30}{\color{blue} Repetição \textgreater{}\textgreater Faz                                            } & Repita, Estrutura de repeticao, Portugol                                                                                                          \\ \hline
40  & \href{http://www.dei.estt.ipt.pt/portugol/node/6}{\color{blue} Linguagem Algorítmica                                                               } & Sintaxe, Semantica, Portugol                                                                                                                      \\ \hline
41  & \href{http://www.dei.estt.ipt.pt/portugol/node/8}{\color{blue} Tipos de dados \textgreater{}\textgreater Básicos                                   } & Tipos de dados, Tipo inteiro, Tipo real, Tipo logico, Tipo caracter, Tipo texto                                                                   \\ \hline
42  & \href{http://www.dei.estt.ipt.pt/portugol/node/9}{\color{blue} Tipos de dados \textgreater{}\textgreater Variáveis                                 } & Variaveis, Portugol                                                                                                                               \\ \hline
43  & \href{http://www.devmedia.com.br/fluxogramas-diagrama-de-blocos-e-de-chapin-no-desenvolvimento-de-algoritmos/28550}{\color{blue} Fluxogramas, diagrama de blocos e de Chapin no desenvolvimento de algoritmos        } & Tipos de algoritmo, Fluxograma                                                                                                                    \\ \hline
44  & \href{http://www.dicasdeprogramacao.com.br/estrutura-de-selecao-multipla-escolha-caso/}{\color{blue} Estrutura de seleção multipla ESCOLHA-CASO                                          } & Estrutura condicional, Decisao encadeada, Escolha                                                                                                 \\ \hline
45  & \href{http://www.dicasdeprogramacao.com.br/o-que-e-algoritmo/}{\color{blue} O que é Algoritmo?                                                                  } & Algoritmo, Problema                                                                                                                               \\ \hline
46  & \href{http://www.din.uem.br/~teclopes/FCaula5.pdf}{\color{blue} Algoritmos – Estruturas de Controle                                                 } & Estrutura condicional, Decisao simples, Decisao composta, Decisao encadeada, Se, Escolha                                                          \\ \hline
47  & \href{http://www.gazetadopovo.com.br/vida-e-cidadania/o-uso-cotidiano-do-algoritmo-4x3n9sw4bkhoam6fzqcp27mfi}{\color{blue} O uso cotidiano do algoritmo                                                        } & Aplicacao de algoritmo                                                                                                                            \\ \hline
48  & \href{http://www.ic.unicamp.br/~sheila/mc102/02\_Entrada\%20Saida\%20e\%20Operadores.pdf}{\color{blue} Comandos de entrada e saída                                                         } & Linguagem C, Entrada, Saida, Atribuicao, Operadores                                                                                               \\ \hline
49  & \href{http://www.ic.unicamp.br/~wainer/cursos/2s2011/Cap05-EstruturasCondicionais-texto.pdf}{\color{blue} Estruturas Condicionais                                                             } & Linguagem C, Estrutura condicional, Operadores relacionais                                                                                        \\ \hline
50  & \href{http://www.inf.pucrs.br/~benso/progi/guia.htm}{\color{blue} Programação para Engenharia I                                                       } & Linguagem C, Tipos de dados, Tipo texto                                                                                                           \\ \hline
51  & \href{http://www.inf.pucrs.br/~pinho/LaproI/ComandosDeRepeticao/Repeticao.html}{\color{blue} Comandos de decisão Comandos de seleção                                             } & Linguagem C, Estrutura condicional, Estrutura de repeticao                                                                                        \\ \hline
52  & \href{http://www.inf.pucrs.br/~pinho/LaproI/IntroC/IntroC.htm}{\color{blue} Introdução à Linguagem C                                                            } & Variaveis, Tipos de dados, Tipo texto, Entrada, Saida, Tipo real, Operadores aritmeticos                                                          \\ \hline
53  & \href{http://www.inf.pucrs.br/~pinho/LaproI/Vetores/Vetores.htm}{\color{blue} Matrizes e Vetores                                                                  } & Linguagem C, Vetor, Matriz, Entrada, Saida, Atribuicao                                                                                            \\ \hline
54  & \href{http://www.inf.ufpr.br/cursos/ci067/Docs/NotasAula/notas-19\_Arrays\_Multidimensionais.html}{\color{blue} Arrays multidimensionais                                                            } & Vetor, Matriz                                                                                                                                     \\ \hline
55  & \href{http://www.inf.ufpr.br/cursos/ci067/Docs/NotasAula/notas-6\_Operadores\_Logicos.html}{\color{blue} Operadores Lógicos                                                                  } & Linguagem C, Operadores logicos                                                                                                                   \\ \hline
56  & \href{http://www.inf.ufsc.br/~bosco/ensino/ine5201/ApostilaVisuAlg20.pdf}{\color{blue} Manual do Visualg                                                                   } & Portugol, Variaveis, Tipos de dados, Sintaxe, Constantes, Atribuicao, Entrada, Saida, Operadores, Estrutura condicional, Estrutura de repeticao   \\ \hline
57  & \href{http://www.ipb.pt/~cmca/algor1.pdf}{\color{blue} Noção e Representação de Algoritmos                                                 } & Algoritmo, Tipos de algoritmo, Problema, Narrativa, Fluxograma                                                                                    \\ \hline
58  & \href{http://www.omundodaprogramacao.com/representacao-de-algoritmos/}{\color{blue} Representação de Algoritmos                                                         } & Algoritmo, Tipos de algoritmo, Narrativa, Fluxograma, Diagrama de Chapin, Pseudocodigo                                                            \\ \hline
59  & \href{http://www.rafaeltoledo.net/estruturas-de-selecao/}{\color{blue} Estrutura de seleção                                                                } & Pseudocodigo, Estrutura condicional                                                                                                               \\ \hline
60  & \href{http://www.rafaeltoledo.net/introducao-a-linguagem-c-parte-i/}{\color{blue} Introdução à linguagem C - Parte 1                                                  } & Linguagem C, Tipos de dados, Operadores relacionais, Tipo texto                                                                                   \\ \hline
61  & \href{http://www.rafaeltoledo.net/introducao-a-logica-de-programacao/}{\color{blue} Introdução à lógica de programação                                                  } & Algoritmo, Logica de programacao, Portugol, Pseudolinguagem                                                                                       \\ \hline
62  & \href{http://www.rafaeltoledo.net/lacos-de-repeticao/}{\color{blue} Laços de repetição                                                                  } & Pseudocodigo, Estrutura de repeticao                                                                                                              \\ \hline
63  & \href{http://www.rafaeltoledo.net/vetores-e-matrizes/}{\color{blue} Vetores e Matrizes                                                                  } & Pseudocodigo, Vetor, Matriz                                                                                                                       \\ \hline
64  & \href{http://www.tecmundo.com.br/programacao/2082-o-que-e-algoritmo-.htm}{\color{blue} O que é algoritmo?                                                                  } & Algoritmo, Tipos de algoritmo                                                                                                                     \\ \hline
65  & \href{http://www.tiexpert.net/programacao/algoritmo/o-que-e-um-algoritmo.php}{\color{blue} O que é um algoritmo?                                                               } & Algoritmo                                                                                                                                         \\ \hline
66  & \href{http://www.univasf.edu.br/~andreza.leite/aulas/AP/VetoresMatrizes.pdf}{\color{blue} Vetores e Matrizes                                                                  } & Vetor, Matriz                                                                                                                                     \\ \hline
67  & \href{http://www.univasf.edu.br/~jose.valentim/aula1.pdf}{\color{blue} Algoritmo e Programação                                                             } & Algoritmo, Tipos de algoritmo, Fluxograma, Narrativa, Portugol, Variaveis, Tipos de dados                                                         \\ \hline
68  & \href{http://www.univasf.edu.br/~ricardo.aramos/disciplinas/AlgProgAgr\_2011\_1/cap03AnaEmilia.pdf}{\color{blue} Estruturas de Controle                                                              } & Estrutura sequencial, Estrutura condicional, Decisao simples, Decisao composta, Decisao encadeada, Estrutura de repeticao, Para, Repita, Enquanto \\ \hline
69  & \href{https://becode.com.br/principais-linguagens-de-programacao/}{\color{blue} As 15 principais linguagens de programação do mundo!                                } & Linguagem de programacao                                                                                                                          \\ \hline
70  & \href{https://dicasdeprogramacao.com.br/o-que-sao-vetores-e-matrizes-arrays/}{\color{blue} O que são Vetores e Matrizes (arrays)                                               } & Vetor, Matriz, Portugol, Estrutura de repeticao                                                                                                   \\ \hline
71  & \href{https://dicasdeprogramacao.com.br/operadores-relacionais/}{\color{blue} Conheça os Operadores Relacionais!                                                  } & Portugol, Operadores relacionais                                                                                                                  \\ \hline
72  & \href{https://lucianopascal.wordpress.com/2010/04/02/aprendendo-a-interpretar-exercicios-de-algoritmos/}{\color{blue} Aprendendo a interpretar exercícios de algoritmos                                   } & Algoritmo, Problema, Entrada, Processamento, Saida                                                                                                \\ \hline
73  & \href{https://msdn.microsoft.com/pt-br/library/474dd6e2.aspx}{\color{blue} Operadores de atribuição C                                                          } & Linguagem C, Atribuicao                                                                                                                           \\ \hline
74  & \href{https://msdn.microsoft.com/pt-br/library/6swh93dx.aspx}{\color{blue} Operadores relacionais e de igualdade C                                             } & Linguagem C, Operadores relacionais                                                                                                               \\ \hline
75  & \href{https://msdn.microsoft.com/pt-br/library/exefbdtf.aspx}{\color{blue} Operador de expressão condicional                                                   } & Linguagem C, Operadores logicos, Operadores relacionais, Estrutura condicional                                                                    \\ \hline
76  & \href{https://msdn.microsoft.com/pt-br/library/z68fx2f1.aspx}{\color{blue} Operadores lógicos C                                                                } & Linguagem C, Operadores logicos                                                                                                                   \\ \hline
77  & \href{https://pt.wikipedia.org/wiki/Algoritmo}{\color{blue} Algoritmo                                                                           } & Algoritmo, Tipos de algoritmo, Programa de computador, Interpretador, Compilador                                                                  \\ \hline
78  & \href{https://pt.wikipedia.org/wiki/Estrutura\_de\_controle}{\color{blue} Estrutura de controle                                                               } & Estrutura sequencial, Estrutura condicional, Estrutura de repeticao                                                                               \\ \hline
79  & \href{https://pt.wikipedia.org/wiki/Estrutura\_de\_repeti\%C3\%A7\%C3\%A3o}{\color{blue} Estrutura de repetição                                                              } & Estrutura de repeticao, Enquanto, Para, Repita                                                                                                    \\ \hline
80  & \href{https://pt.wikipedia.org/wiki/Operadores\_em\_C\_e\_C\%2B\%2B}{\color{blue} Operadores em C e C++                                                               } & Linguagem C, Operadores                                                                                                                           \\ \hline
81  & \href{https://sites.google.com/site/itabits/treinamento/introducao-a-programacao-em-c/comandos-de-repeticao}{\color{blue} Comandos de Repetição (Laços ou Loops)                                              } & Linguagem C, Estrutura de repeticao, Enquanto, Para, Repita                                                                                       \\ \hline
82  & \href{https://www.dca.ufrn.br/~ivan/DCA0800/tiposDados.pdf}{\color{blue} Outros conceitos sobre logica de programação: Tipos de dados, variaveis e expressões} & Tipos de dados, Variaveis, Operacoes, Operacoes aritmeticas, Operacoes logicas                                                                    \\ \hline
83  & \href{https://www.dcc.fc.up.pt/~nam/aulas/9900/ic/slides/sliic9918/}{\color{blue} Linguagens de Programação                                                           } & Linguagem de programacao, Linguagem de maquina, Compilador                                                                                        \\ \hline
84  & \href{https://www.devmedia.com.br/estrutura-de-decisao-em-c-c/24031}{\color{blue} Estrutura de Decisão em C/C++                                                       } & Linguagem C, Estrutura condicional, Se, Escolha                                                                                                   \\ \hline
85  & \href{https://www.ime.usp.br/~hitoshi/introducao/03-Fundamentos.pdf}{\color{blue} Fundamentos                                                                         } & Linguagem C, Entrada, Saida, Atribuicao, Operadores, Variaveis, Precedencia de operadores                                                         \\ \hline
86  & \href{https://www.ime.usp.br/~pf/algoritmos/aulas/aloca.html}{\color{blue} Alocação dinâmica de memória                                                        } & Vetor, Matriz, Linguagem C                                                                                                                        \\ \hline
87  & \href{https://www.ime.usp.br/~pf/algoritmos/aulas/string.html}{\color{blue} Strings                                                                             } & Tipo texto, Entrada, Saida, Constantes                                                                                                            \\ \hline
88  & \href{https://www.inf.pucrs.br/~pinho/LaproI/Vetores/Vetores.htm}{\color{blue} Programação C/C++ - Matrizes e Vetores                                              } & Linguagem C, Vetor, Matriz                                                                                                                        \\ \hline
89  & \href{https://www.learnconline.com/2010/03/if-else-statement-c-programming-language.html}{\color{blue} The if-else Statement in c programming language                                     } & Se, Linguagem C                                                                                                                                   \\ \hline
90  & \href{https://www.learnconline.com/2010/03/while-loop-statement-in-c.html}{\color{blue} While Statement in c programming language                                           } & Enquanto, Linguagem C                                                                                                                             \\ \hline
91  & \href{https://www.programiz.com/c-programming/c-multi-dimensional-arrays}{\color{blue} C Programming Multidimensional Arrays                                               } & Linguagem C, Vetor, Matriz                                                                                                                        \\ \hline
92  & \href{https://www.tutorialspoint.com/cprogramming/c\_arrays.htm}{\color{blue} C- arrays                                                                           } & Linguagem C, Vetor                                                                                                                                \\ \hline
93  & \href{https://www.tutorialspoint.com/cprogramming/c\_multi\_dimensional\_arrays.htm}{\color{blue} Multi-dimensional Arrays in C                                                       } & Linguagem C, Matriz                                                                                                                               \\ \hline
94  & \href{https://www.tutorialspoint.com/cprogramming/c\_strings.htm}{\color{blue} C - Strings                                                                         } & Linguagem C, Tipo texto                                                                                                                           \\ \hline
95  & \href{https://www.youtube.com/watch?v=7oA8SBAOOAo}{\color{blue} Programar em C - Revisão Vetores/Matrizes                                           } & Vetor, Matriz, Problema                                                                                                                           \\ \hline
96  & \href{https://www.youtube.com/watch?v=7ph98Ih\_ckc}{\color{blue} Lógica de programação - Aula 03 - Legibilidade do código                            } & Algoritmo, Comentario, Blocos                                                                                                                     \\ \hline
97  & \href{https://www.youtube.com/watch?v=Ds1n6aHchRU}{\color{blue} Lógica de programação - Aula 01 - Introdução                                        } & Logica de programacao, Aplicacao de algoritmo                                                                                                     \\ \hline
98  & \href{https://www.youtube.com/watch?v=dZq7l9Oj-\_c}{\color{blue} Portugol - VisuALG - Aula 01 (Princípios Básicos)                                   } & Tipos de dados, Tipo inteiro, Tipo real, Tipo logico, Tipo caracter, Entrada, Saida, Atribuicao, Operadores aritmeticos                           \\ \hline
99  & \href{https://www.youtube.com/watch?v=ED7QtgXDShY}{\color{blue} Libraries                                                                           } & Linguagem C                                                                                                                                       \\ \hline
100 & \href{https://www.youtube.com/watch?v=g0iIVeeQo1M}{\color{blue} Lógica de programação - Aula 05 - Expressões, operadores e comandos                 } & Entrada, Saida, Operadores, Operadores relacionais, Operadores lógicos, Operadores aritmeticos, Atribuicao                                        \\ \hline
101 & \href{https://www.youtube.com/watch?v=JLlTo3SwxJE}{\color{blue} Lógica de programação - Aula 02 - Tipos de algoritmo                                } & Tipos de algoritmo, Fluxograma, Pseudocodigo, Narrativa                                                                                           \\ \hline
102 & \href{https://www.youtube.com/watch?v=l26oaHV7D40}{\color{blue} Programming Basics: Statements \& Functions: Crash Course Computer Science          } & Semantica, Sintaxe, Blocos                                                                                                                        \\ \hline
103 & \href{https://www.youtube.com/watch?v=Lelg\_sOYSm0}{\color{blue} Portugol Studio - Teste de Mesa                                                     } & Algoritmo, Processamento, Logica de programacao                                                                                                   \\ \hline
104 & \href{https://www.youtube.com/watch?v=mHW1Hsqlp6A}{\color{blue} Por que todos deveriam aprender a programar?                                        } & Aplicacao de algoritmo                                                                                                                            \\ \hline
105 & \href{https://www.youtube.com/watch?v=UuTmEcy5rV0}{\color{blue} Vetores e Matrizes                                                                  } & Vetor, Matriz, Fluxograma, Pseudocodigo                                                                                                           \\ \hline
106 & \href{https://www.youtube.com/watch?v=vgu8x\_Ivjd0}{\color{blue} Lógica de Programação - Estruturas de Repetição (Enquanto, Para, FacaEnquanto)      } & Logica de programacao, Estrutura de repeticao, Para, Enquanto, Repita                                                                             \\ \hline
107 & \href{https://www.youtube.com/watch?v=vp4jgXA\_BB0}{\color{blue} Lógica de programação - Aula 04 - Variáveis e constantes                            } & Variaveis, Constantes, Tipos de dados, Tipo inteiro, Tipo real, Tipo texto, Tipo logico, Vetor, Matriz                                            \\ \hline
108 & \href{https://www.zemoleza.com.br/trabalho-academico/exatas/informatica/as-principais-bibliotecas-em-linguagem-c/}{\color{blue} As principais bibliotecas em linguagem C                                            } & Linguagem C \\ \hline
\caption{Palavras-chave associadas aos Links de Apoio}
\label{tab:palavras-chave-links-de-apoio}
\end{longtable}

  \include{Partes/apeG-intervencoes}
  \chapter{Análise Estatística das Questões do Questionário}\label{ape:analise-estatistica-questionario}

\section{Questão 1: Os itens recomendados corresponderam com os meus interesses.}

\begin{figure}[htb]
  \caption{\label{fig:questao1-boxplot}Boxplot da questão 1}
  \begin{center}
      \includegraphics[scale=0.4]{./Figuras/questao1-boxplot.png}
  \end{center}
  \legend{Fonte: O autor.}
\end{figure}

\begin{multicols}{2}

\noindent\textbf{Tradicional}\\
Min = 1.000\\
1\textsuperscript{o} Quad = 4.000\\
Mediana = 4.000\\
Média = 4.032\\
3\textsuperscript{o} Quad = 5.000\\
Max = 5.000\\
\columnbreak

\noindent\textbf{Proposta}\\
Min = 3.000\\
1\textsuperscript{o} Quad = 3.750\\
Mediana = 4.000\\
Média = 4.042\\
3\textsuperscript{o} Quad = 5.000\\
Max = 5.000
\end{multicols}

Wilcoxon rank sum test with continuity correction

\noindent
data:  $data\_1\_tradicional$ and $data\_1\_proposta$\\
W = 404, p-value = 0.5635\\
alternative hypothesis: true location shift is not equal to 0

\noindent
\textbf{Resultado: Aceita a hipótese nula - Sem diferença significativa}

\newpage
\section{Questão 2: Os itens recomendados para mim são diversificados (o sistema se preocupa em trazer itens diferentes a cada recomendação).}

\begin{figure}[htb]
  \caption{\label{fig:questao2-boxplot}Boxplot da questão 2}
  \begin{center}
      \includegraphics[scale=0.4]{./Figuras/questao2-boxplot.png}
  \end{center}
  \legend{Fonte: O autor.}
\end{figure}

\begin{multicols}{2}

\noindent\textbf{Tradicional}\\
Min = 1.000\\
1\textsuperscript{o} Quad = 4.000\\
Mediana = 4.000\\
Média = 3.933\\
3\textsuperscript{o} Quad = 4.000\\
Max = 5.000\\
\columnbreak

\noindent\textbf{Proposta}\\
Min = 2\\
1\textsuperscript{o} Quad = 3\\
Mediana = 4\\
Média = 4\\
3\textsuperscript{o} Quad = 5\\
Max = 5
\end{multicols}

Wilcoxon rank sum test with continuity correction

\noindent
data:  $data\_2\_tradicional$ and $data\_2\_proposta$\\
W = 376.5, p-value = 0.8178\\
alternative hypothesis: true location shift is not equal to 0

\noindent
\textbf{Resultado: Aceita a hipótese nula - Sem diferença significativa}

\newpage
\section{Questão 3: Os itens recomendados corresponderam aos  interesses e necessidades que eu tinha no momento.}

\begin{figure}[htb]
  \caption{\label{fig:questao3-boxplot}Boxplot da questão 3}
  \begin{center}
      \includegraphics[scale=0.4]{./Figuras/questao3-boxplot.png}
  \end{center}
  \legend{Fonte: O autor.}
\end{figure}

\begin{multicols}{2}

\noindent\textbf{Tradicional}\\
Min = 1.000\\
1\textsuperscript{o} Quad = 3.000\\
Mediana = 4.000\\
Média = 3.806\\
3\textsuperscript{o} Quad = 4.500\\
Max = 5.000\\
\columnbreak

\noindent\textbf{Proposta}\\
Min = 2\\
1\textsuperscript{o} Quad = 4\\
Mediana = 4\\
Média = 4\\
3\textsuperscript{o} Quad = 5\\
Max = 5
\end{multicols}

Wilcoxon rank sum test with continuity correction

\noindent
data:  $data\_3\_tradicional$ and $data\_3\_proposta$\\
W = 364, p-value = 0.5119\\
alternative hypothesis: true location shift is not equal to 0

\noindent
\textbf{Resultado: Aceita a hipótese nula - Sem diferença significativa}

\newpage
\section{Questão 4: As recomendações são feitas no momento adequado.}

\begin{figure}[htb]
  \caption{\label{fig:questao4-boxplot}Boxplot da questão 4}
  \begin{center}
      \includegraphics[scale=0.4]{./Figuras/questao4-boxplot.png}
  \end{center}
  \legend{Fonte: O autor.}
\end{figure}

\begin{multicols}{2}

\noindent\textbf{Tradicional}\\
Min = 1.000\\
1\textsuperscript{o} Quad = 4.000\\
Mediana = 4.000\\
Média = 3.931\\
3\textsuperscript{o} Quad = 4.000\\
Max = 5.000\\
\columnbreak

\noindent\textbf{Proposta}\\
Min = 2\\
1\textsuperscript{o} Quad = 3\\
Mediana = 4\\
Média = 4\\
3\textsuperscript{o} Quad = 5\\
Max = 5
\end{multicols}

Wilcoxon rank sum test with continuity correction

\noindent
data:  $data\_4\_tradicional$ and $data\_4\_proposta$\\
W = 378, p-value = 0.8198\\
alternative hypothesis: true location shift is not equal to 0

\noindent
\textbf{Resultado: Aceita a hipótese nula - Sem diferença significativa}

\newpage
\section{Questão 5: O sistema de recomendação explica porque os links são recomendados para mim.}

\begin{figure}[htb]
  \caption{\label{fig:questao5-boxplot}Boxplot da questão 5}
  \begin{center}
      \includegraphics[scale=0.4]{./Figuras/questao5-boxplot.png}
  \end{center}
  \legend{Fonte: O autor.}
\end{figure}

\begin{multicols}{2}

\noindent\textbf{Tradicional}\\
Min = 1.000\\
1\textsuperscript{o} Quad = 3.500\\
Mediana = 4.000\\
Média = 3.889\\
3\textsuperscript{o} Quad = 5.000\\
Max = 5.000\\
\columnbreak

\noindent\textbf{Proposta}\\
Min = 2.00\\
1\textsuperscript{o} Quad = 3.00\\
Mediana = 4.00\\
Média = 3.72\\
3\textsuperscript{o} Quad = 4.00\\
Max = 5.00
\end{multicols}

Wilcoxon rank sum test with continuity correction

\noindent
data:  $data\_5\_tradicional$ and $data\_5\_proposta$\\
W = 396, p-value = 0.2665\\
alternative hypothesis: true location shift is not equal to 0

\noindent
\textbf{Resultado: Aceita a hipótese nula - Sem diferença significativa}

\newpage
\section{Questão 6: A informação apresentada na interface para os itens recomendados é suficiente para mim.}

\begin{figure}[htb]
  \caption{\label{fig:questao6-boxplot}Boxplot da questão 6}
  \begin{center}
      \includegraphics[scale=0.4]{./Figuras/questao6-boxplot.png}
  \end{center}
  \legend{Fonte: O autor.}
\end{figure}

\begin{multicols}{2}

\noindent\textbf{Tradicional}\\
Min = 1.000\\
1\textsuperscript{o} Quad = 3.000\\
Mediana = 4.000\\
Média = 3.655\\
3\textsuperscript{o} Quad = 5.000\\
Max = 5.000\\
\columnbreak

\noindent\textbf{Proposta}\\
Min = 3.000\\
1\textsuperscript{o} Quad = 4.000\\
Mediana = 4.000\\
Média = 4.192\\
3\textsuperscript{o} Quad = 5.000\\
Max = 5.000
\end{multicols}

Wilcoxon rank sum test with continuity correction

\noindent
data:  $data\_6\_tradicional$ and $data\_6\_proposta$\\
W = 301.5, p-value = 0.1799\\
alternative hypothesis: true location shift is not equal to 0

\noindent
\textbf{Resultado: Aceita a hipótese nula - Sem diferença significativa}

\newpage
\section{Questão 7: O layout do sistema de recomendação é atrativo e adequado.}

\begin{figure}[htb]
  \caption{\label{fig:questao7-boxplot}Boxplot da questão 7}
  \begin{center}
      \includegraphics[scale=0.4]{./Figuras/questao7-boxplot.png}
  \end{center}
  \legend{Fonte: O autor.}
\end{figure}

\begin{multicols}{2}

\noindent\textbf{Tradicional}\\
Min = 1.000\\
1\textsuperscript{o} Quad = 3.000\\
Mediana = 4.000\\
Média = 3.676\\
3\textsuperscript{o} Quad = 4.750\\
Max = 5.000\\
\columnbreak

\noindent\textbf{Proposta}\\
Min = 1.0\\
1\textsuperscript{o} Quad = 3.0\\
Mediana = 4.0\\
Média = 3.8\\
3\textsuperscript{o} Quad = 5.0\\
Max = 5.0
\end{multicols}

Wilcoxon rank sum test with continuity correction

\noindent
data:  $data\_7\_tradicional$ and $data\_7\_proposta$\\
W = 413, p-value = 0.8536\\
alternative hypothesis: true location shift is not equal to 0

\noindent
\textbf{Resultado: Aceita a hipótese nula - Sem diferença significativa}

\newpage
\section{Questão 8: Eu encontrei facilmente o local onde os itens são recomendados.}

\begin{figure}[htb]
  \caption{\label{fig:questao8-boxplot}Boxplot da questão 8}
  \begin{center}
      \includegraphics[scale=0.4]{./Figuras/questao8-boxplot.png}
  \end{center}
  \legend{Fonte: O autor.}
\end{figure}

\begin{multicols}{2}

\noindent\textbf{Tradicional}\\
Min = 2.0\\
1\textsuperscript{o} Quad = 3.0\\
Mediana = 4.0\\
Média = 3.8\\
3\textsuperscript{o} Quad = 5.0\\
Max = 5.0\\
\columnbreak

\noindent\textbf{Proposta}\\
Min = 2.00\\
1\textsuperscript{o} Quad = 3.00\\
Mediana = 4.00\\
Média = 3.75\\
3\textsuperscript{o} Quad = 4.25\\
Max = 5.00
\end{multicols}

Wilcoxon rank sum test with continuity correction

\noindent
data:  $data\_8\_tradicional$ and $data\_8\_proposta$\\
W = 381, p-value = 0.7093\\
alternative hypothesis: true location shift is not equal to 0

\noindent
\textbf{Resultado: Aceita a hipótese nula - Sem diferença significativa}

\newpage
\section{Questão 9: Eu percebi que o sistema de recomendação aprendia sobre minhas necessidades/preferências conforme eu avançava na disciplina.}

\begin{figure}[htb]
  \caption{\label{fig:questao9-boxplot}Boxplot da questão 9}
  \begin{center}
      \includegraphics[scale=0.4]{./Figuras/questao9-boxplot.png}
  \end{center}
  \legend{Fonte: O autor.}
\end{figure}

\begin{multicols}{2}

\noindent\textbf{Tradicional}\\
Min = 1.000\\
1\textsuperscript{o} Quad = 3.000\\
Mediana = 4.000\\
Média = 3.643\\
3\textsuperscript{o} Quad = 4.250\\
Max = 5.000\\
\columnbreak

\noindent\textbf{Proposta}\\
Min = 3.000\\
1\textsuperscript{o} Quad = 3.000\\
Mediana = 4.000\\
Média = 3.875\\
3\textsuperscript{o} Quad = 5.000\\
Max = 5.000
\end{multicols}

Wilcoxon rank sum test with continuity correction

\noindent
data:  $data\_9\_tradicional$ and $data\_9\_proposta$\\
W = 313.5, p-value = 0.6722\\
alternative hypothesis: true location shift is not equal to 0

\noindent
\textbf{Resultado: Aceita a hipótese nula - Sem diferença significativa}

\newpage
\section{Questão 10: É facil encontrar um item para estudar com a ajuda do sistema de recomendação.}

\begin{figure}[htb]
  \caption{\label{fig:questao10-boxplot}Boxplot da questão 10}
  \begin{center}
      \includegraphics[scale=0.4]{./Figuras/questao10-boxplot.png}
  \end{center}
  \legend{Fonte: O autor.}
\end{figure}

\begin{multicols}{2}

\noindent\textbf{Tradicional}\\
Min = 1.00\\
1\textsuperscript{o} Quad = 3.00\\
Mediana = 4.00\\
Média = 3.69\\
3\textsuperscript{o} Quad = 4.00\\
Max = 5.00\\
\columnbreak

\noindent\textbf{Proposta}\\
Min = 2\\
1\textsuperscript{o} Quad = 4\\
Mediana = 4\\
Média = 4\\
3\textsuperscript{o} Quad = 5\\
Max = 5
\end{multicols}

Wilcoxon rank sum test with continuity correction

\noindent
data:  $data\_10\_tradicional$ and $data\_10\_proposta$\\
W = 296, p-value = 0.1452\\
alternative hypothesis: true location shift is not equal to 0

\noindent
\textbf{Resultado: Aceita a hipótese nula - Sem diferença significativa}

\newpage
\section{Questão 11: Eu me senti apoiado para encontrar itens do meu interesse com a ajuda do sistema de recomendação.}

\begin{figure}[htb]
  \caption{\label{fig:questao11-boxplot}Boxplot da questão 11}
  \begin{center}
      \includegraphics[scale=0.4]{./Figuras/questao11-boxplot.png}
  \end{center}
  \legend{Fonte: O autor.}
\end{figure}

\begin{multicols}{2}

\noindent\textbf{Tradicional}\\
Min = 1.000\\
1\textsuperscript{o} Quad = 3.000\\
Mediana = 4.000\\
Média = 3.607\\
3\textsuperscript{o} Quad = 4.000\\
Max = 5.000\\
\columnbreak

\noindent\textbf{Proposta}\\
Min = 2\\
1\textsuperscript{o} Quad = 3\\
Mediana = 4\\
Média = 4\\
3\textsuperscript{o} Quad = 5\\
Max = 5
\end{multicols}

Wilcoxon rank sum test with continuity correction

\noindent
data:  $data\_11\_tradicional$ and $data\_11\_proposta$\\
W = 286, p-value = 0.2309\\
alternative hypothesis: true location shift is not equal to 0

\noindent
\textbf{Resultado: Aceita a hipótese nula - Sem diferença significativa}

\newpage
\section{Questão 12: Eu entendi porque os itens foram recomendados para mim.}

\begin{figure}[htb]
  \caption{\label{fig:questao12-boxplot}Boxplot da questão 12}
  \begin{center}
      \includegraphics[scale=0.4]{./Figuras/questao12-boxplot.png}
  \end{center}
  \legend{Fonte: O autor.}
\end{figure}

\begin{multicols}{2}

\noindent\textbf{Tradicional}\\
Min = 1.0\\
1\textsuperscript{o} Quad = 3.0\\
Mediana = 4.0\\
Média = 3.4\\
3\textsuperscript{o} Quad = 4.0\\
Max = 5.0\\
\columnbreak

\noindent\textbf{Proposta}\\
Min = 2.00\\
1\textsuperscript{o} Quad = 4.00\\
Mediana = 4.00\\
Média = 4.12\\
3\textsuperscript{o} Quad = 5.00\\
Max = 5.00
\end{multicols}

Wilcoxon rank sum test with continuity correction

\noindent
data:  $data\_12\_tradicional$ and $data\_12\_proposta$\\
W = 240, p-value = 0.01513\\
alternative hypothesis: true location shift is not equal to 0

\noindent
\textbf{Resultado: Aceita a hipótese alternativa - Com diferença significativa}

\newpage
\section{Questão 13: No geral, estou satisfeito com o sistema de recomendação.}

\begin{figure}[htb]
  \caption{\label{fig:questao13-boxplot}Boxplot da questão 13}
  \begin{center}
      \includegraphics[scale=0.4]{./Figuras/questao13-boxplot.png}
  \end{center}
  \legend{Fonte: O autor.}
\end{figure}

\begin{multicols}{2}

\noindent\textbf{Tradicional}\\
Min = 1\\
1\textsuperscript{o} Quad = 4\\
Mediana = 4\\
Média = 4\\
3\textsuperscript{o} Quad = 5\\
Max = 5\\
\columnbreak

\noindent\textbf{Proposta}\\
Min = 2.000\\
1\textsuperscript{o} Quad = 4.000\\
Mediana = 5.000\\
Média = 4.308\\
3\textsuperscript{o} Quad = 5.000\\
Max = 5.000
\end{multicols}

Wilcoxon rank sum test with continuity correction

\noindent
data:  $data\_13\_tradicional$ and $data\_13\_proposta$\\
W = 356.5, p-value = 0.2388\\
alternative hypothesis: true location shift is not equal to 0

  \chapter{Análise Estatística do Desempenho do Sistema de Recomendação}\label{ape:analise-estatistica-do-uso}

\section{Links Acessados por Aluno}

\begin{figure}[htb]
  \caption{\label{fig:uso-sr-boxplot}Boxplot dos links acessados por aluno}
  \begin{center}
      \includegraphics[scale=0.4]{./Figuras/uso-sr-boxplot.png}
  \end{center}
  \legend{Fonte: O autor.}
\end{figure}

\noindent
Total de alunos comparados: 85

\begin{multicols}{2}

\noindent\textbf{Tradicional}\\
Min =  1.000\\
1\textsuperscript{o} Quad =  2.000\\
Mediana =  4.000\\
Média =  4.935\\
3\textsuperscript{o} Quad = 6.500\\
Max = 18.000\\

\columnbreak

\noindent\textbf{Proposta}\\
 Min =   1.00\\
 1\textsuperscript{o} Quad =   2.00\\
 Mediana =   5.00\\
 Média =  10.15\\
 3\textsuperscript{o} Quad =  9.00\\
 Max = 107.00
\end{multicols}

Shapiro-Wilk normality test

\noindent
data:  data[["quantidade"]]\\
W = 0.42461, p-value < 2.2e-16

\noindent
\textbf{Resultado: Aceita a hipótese alternativa - Distribuição não normal}

Wilcoxon rank sum test with continuity correction

\noindent
data:  data[["quantidade"]] by data[["algoritmo\_recomendacao"]]\\
W = 820, p-value = 0.4957\\
alternative hypothesis: true location shift is not equal to 0

\noindent
\textbf{Resultado: Aceita a hipótese nula - Sem diferença significativa}

\newpage
\section{Links Avaliados Positivamente por Aluno}

\begin{figure}[htb]
  \caption{\label{fig:avaliados-positivamente-boxplot}Boxplot dos links avaliados positivamente por aluno}
  \begin{center}
      \includegraphics[scale=0.4]{./Figuras/avaliados-positivamente-boxplot.png}
  \end{center}
  \legend{Fonte: O autor.}
\end{figure}

\noindent
Total de alunos comparados: 52

\begin{multicols}{2}

\noindent\textbf{Tradicional}\\
Min =  1.0\\
1\textsuperscript{o} Quad =  1.0\\
Mediana =  4.0\\
Média =  4.7\\
3\textsuperscript{o} Quad = 6.0\\
Max = 16.0\\
\columnbreak

\noindent\textbf{Proposta}\\
Min =   1.00\\
1\textsuperscript{o} Quad =   1.00\\
Mediana =   2.50\\
Média =  10.64\\
3\textsuperscript{o} Quad =  9.75\\
Max = 107.00
\end{multicols}

  Shapiro-Wilk normality test

\noindent
data:  data[["quantidade"]]\\
W = 0.37836, p-value = 1.654e-13

\noindent
\textbf{Resultado: Aceita a hipótese alternativa - Distribuição não normal}

Wilcoxon rank sum test with continuity correction

\noindent
data:  data[["quantidade"]] by data[["algoritmo\_recomendacao"]]\\
W = 318, p-value = 0.8275\\
alternative hypothesis: true location shift is not equal to 0

\noindent
\textbf{Resultado: Aceita a hipótese nula - Sem diferença significativa}

\newpage
\section{Links avaliados negativamente por aluno que acessou pelo menos uma recomendação}

\begin{figure}[htb]
  \caption{\label{fig:uso-sr-boxplot}Boxplot dos links acessados por aluno}
  \begin{center}
      \includegraphics[scale=0.4]{./Figuras/uso-sr-boxplot.png}
  \end{center}
  \legend{Fonte: O autor.}
\end{figure}

\noindent
Total de alunos comparados: 7

\begin{multicols}{2}

\noindent\textbf{Tradicional}\\
Min = 1\\
1\textsuperscript{o} Quad = 1\\
Mediana = 1\\
Média = 1\\
3\textsuperscript{o} Quad =1\\
Max = 1\\

\columnbreak

\noindent\textbf{Proposta}\\
Min = 1.0\\
1\textsuperscript{o} Quad = 1.5\\
Mediana = 2.0\\
Média = 2.0\\
3\textsuperscript{o} Quad =2.5\\
Max = 3.0
\end{multicols}

  Shapiro-Wilk normality test

\noindent
data:  data[["quantidade"]]\\
W = 0.45297, p-value = 4.136e-06

\noindent
\textbf{Resultado: Aceita a hipótese alternativa - Distribuição não normal}

Wilcoxon rank sum test with continuity correction

\noindent
data:  data[["quantidade"]] by data[["algoritmo\_recomendacao"]]\\
W = 2.5, p-value = 0.2059\\
alternative hypothesis: true location shift is not equal to 0

\noindent
\textbf{Resultado: Aceita a hipótese nula - Sem diferença significativa}

\newpage
\section{Precisão dos algoritmos de recomendação por aluno}

\begin{figure}[htb]
  \caption{\label{fig:precisao-boxplot}Boxplot da precisão dos algoritmos de recomendação por aluno}
  \begin{center}
      \includegraphics[scale=0.4]{./Figuras/precisao-boxplot.png}
  \end{center}
  \legend{Fonte: O autor.}
\end{figure}

\noindent
Total de alunos comparados: 85

\begin{multicols}{2}

\noindent\textbf{Tradicional}\\
Min = 0.04348\\
1\textsuperscript{o} Quad = 0.20000\\
Mediana = 0.42857\\
Média = 0.49679\\
3\textsuperscript{o} Quad =0.81364\\
Max = 1.00000\\

\columnbreak

\noindent\textbf{Proposta}\\
Min = 0.05556\\
1\textsuperscript{o} Quad = 0.18333\\
Mediana = 0.31818\\
Média = 0.39859\\
3\textsuperscript{o} Quad =0.52778\\
Max = 1.00000
\end{multicols}

  Shapiro-Wilk normality test

\noindent
data:  data[["precisao"]]\\
W = 0.89495, p-value = 4.234e-06

\noindent
\textbf{Resultado: Aceita a hipótese alternativa - Distribuição não normal}

Wilcoxon rank sum test with continuity correction

\noindent
data:  data[["precisao"]] by data[["algoritmo\_recomendacao"]]\\
W = 1025, p-value = 0.2603\\
alternative hypothesis: true location shift is not equal to 0

\noindent
\textbf{Resultado: Aceita a hipótese nula - Sem diferença significativa}

\newpage
\section{Cobertura dos algoritmos de recomendação por aluno}

\begin{figure}[htb]
  \caption{\label{fig:coverage-boxplot}Boxplot da cobertura dos algoritmos de recomendação por aluno}
  \begin{center}
      \includegraphics[scale=0.4]{./Figuras/coverage-boxplot.png}
  \end{center}
  \legend{Fonte: O autor.}
\end{figure}

\noindent
Total de alunos comparados: 85

\begin{multicols}{2}

\noindent\textbf{Tradicional}\\
Min = 0.04630\\
1\textsuperscript{o} Quad = 0.06481\\
Mediana = 0.09259\\
Média = 0.10064\\
3\textsuperscript{o} Quad =0.12963\\
Max = 0.21296\\

\columnbreak

\noindent\textbf{Proposta}\\
Min = 0.04630\\
1\textsuperscript{o} Quad = 0.08796\\
Mediana = 0.14815\\
Média = 0.17450\\
3\textsuperscript{o} Quad =0.16667\\
Max = 0.99074
\end{multicols}

    Shapiro-Wilk normality test

\noindent
data:  data[["coverage"]]\\
W = 0.55934, p-value = 1.421e-14

\noindent
\textbf{Resultado: Aceita a hipótese alternativa - Distribuição não normal}

  Wilcoxon rank sum test with continuity correction

\noindent
data:  data[["coverage"]] by data[["algoritmo\_recomendacao"]]\\
W = 525.5, p-value = 0.001031\\
alternative hypothesis: true location shift is not equal to 0

\noindent
\textbf{Resultado: Aceita a hipótese alternativa - Com diferença significativa}

\newpage
\section{F-measure do algoritmo de recomendação por aluno}

\begin{figure}[htb]
  \caption{\label{fig:media-harmonica-boxplot}Boxplot do F-measure dos dois algoritmos}
  \begin{center}
      \includegraphics[scale=0.4]{./Figuras/media-harmonica-boxplot.png}
  \end{center}
  \legend{Fonte: O autor.}
\end{figure}

\noindent
Total de alunos comparados: 85

\begin{multicols}{2}

\noindent\textbf{Tradicional}\\
Min = 0.07221\\
1\textsuperscript{o} Quad = 0.09545\\
Mediana = 0.12952\\
Média = 0.13852\\
3\textsuperscript{o} Quad =0.16618\\
Max = 0.30717\\

\columnbreak

\noindent\textbf{Proposta}\\
Min = 0.07519\\
1\textsuperscript{o} Quad = 0.10913\\
Mediana = 0.16393\\
Média = 0.21719\\
3\textsuperscript{o} Quad =0.24919\\
Max = 0.99535
\end{multicols}

Shapiro-Wilk normality test

\noindent
data:  data[["media\_harmonica"]]\\
W = 0.63292, p-value = 2.931e-13

\noindent
\textbf{Resultado: Aceita a hipótese alternativa - Distribuição não normal}

Wilcoxon rank sum test with continuity correction

\noindent
data:  data[["media\_harmonica"]] by data[["algoritmo\_recomendacao"]]\\
W = 634.5, p-value = 0.02082\\
alternative hypothesis: true location shift is not equal to 0

\noindent
\textbf{Resultado: Aceita a hipótese alternativa - Com diferença significativa}

\end{apendicesenv}

% ----------------------------------------------------------
% Anexos
% ----------------------------------------------------------
\begin{anexosenv}
  % Imprime uma página indicando o início dos anexos
  \partanexos

  \chapter{60 Questões do Framework de Avaliação de Sistemas de Recomendação ResQue}\label{ane:questoes-framework}

\section{Quality of Recommended Items}

\subsection{Accuracy}

\begin{itemize}
\item The items recommended to me matched my interests.*
\item The recommender gave me good suggestions.
\item I am not interested in the items recommended to me (reverse scale).
\end{itemize}

\subsection{Relative Accuracy}

\begin{itemize}
\item The recommendation I received better fits my interests than what I may receive from a friend.
\item A recommendation from my friends better suits my interests than the recommendation from this system (reverse scale).
\end{itemize}

\subsection{Familiarity}

\begin{itemize}
\item Some of the recommended items are familiar to me.
\item I am not familiar with the items that were recommended to me (reverse scale).
\end{itemize}

\subsection{Attractiveness}

\begin{itemize}
\item The items recommended to me are attractive.
\end{itemize}

\subsection{Enjoyability}

\begin{itemize}
\item I enjoyed the items recommended to me.
\end{itemize}

\subsection{Novelty}

\begin{itemize}
\item The items recommended to me are novel and interesting.*
\item The recommender system is educational.
\item The recommender system helps me discover new products.
\item I could not find new items through the recommender (reverse scale).
\end{itemize}

\subsection{Diversity}

\begin{itemize}
\item The items recommended to me are diverse.*
\item The items recommended to me are similar to each other (reverse scale).*
\end{itemize}

\subsection{Context Compatibility}

\begin{itemize}
\item I was only provided with general recommendations.
\item The items recommended to me took my personal context requirements into consideration.
\item The recommendations are timely.
\end{itemize}

\section{Interaction Adequacy}

\begin{itemize}
\item The recommender provides an adequate way for me to express my preferences.
\item The recommender provides an adequate way for me to revise my preferences.
\item The recommender explains why the products are recommended to me.*
\end{itemize}

\section{Interface Adequacy}

\begin{itemize}
\item The recommender’s interface provides sufficient information.
\item The information provided for the recommended items is sufficient for me.
\item The labels of the recommender interface are clear and adequate.
\item The layout of the recommender interface is attractive and adequate.*
\end{itemize}

\section{Perceived Ease of Use}

\subsection{Ease of Initial Learning}

\begin{itemize}
\item I became familiar with the recommender system very quickly.
\item I easily found the recommended items.
\item Looking for a recommended item required too much effort (reverse scale).
\end{itemize}

\subsection{Ease of Preference Elicitation}

\begin{itemize}
\item I found it easy to tell the system about my preferences.
\item It is easy to learn to tell the system what I like.
\item It required too much effort to tell the system what I like (reversed scale).
\end{itemize}

\subsection{Ease of Preference Revision}

\begin{itemize}
\item I found it easy to make the system recommend different things to me.
\item It is easy to train the system to update my preferences.
\item I found it easy to alter the outcome of the recommended items due to my preference changes.
\item It is easy for me to inform the system if I dislike/like the recommended item.
\item It is easy for me to get a new set of recommendations.
\end{itemize}

\subsection{Ease of Decision Making}

\begin{itemize}
\item Using the recommender to find what I like is easy.
\item I was able to take advantage of the recommender very quickly.
\item I quickly became productive with the recommender.
\item Finding an item to buy with the help of the recommender is easy.*
\item Finding an item to buy, even with the help of the recommender, consumes too much time.
\end{itemize}

\section{Perceived Usefulness}

\begin{itemize}
\item The recommended items effectively helped me find the ideal product.*
\item The recommended items influence my selection of products.
\item I feel supported to find what I like with the help of the recommender.*
\item I feel supported in selecting the items to buy with the help of the recommender.
\end{itemize}

\section{Control/Transparency}

\begin{itemize}
\item I feel in control of telling the recommender what I want.
\item I don’t feel in control of telling the system what I want.
\item I don’t feel in control of specifying and changing my preferences (reverse scale).
\item I understood why the items were recommended to me.
\item The system helps me understand why the items were recommended to me.
\item The system seems to control my decision process rather than me (reverse scale).
\end{itemize}

\section{Attitudes}

\begin{itemize}
\item Overall, I am satisfied with the recommender.*
\item I am convinced of the products recommended to me.*
\item I am confident I will like the items recommended to me. *
\item The recommender made me more confident about my selection/decision.
\item The recommended items made me confused about my choice (reverse scale).
\item The recommender can be trusted.
\end{itemize}

\section{Behavioral Intentions}

\subsection{Intention to Use the System}

\begin{itemize}
\item If a recommender such as this exists, I will use it to find products to buy.
\end{itemize}

\subsection{Continuance and Frequency}

\begin{itemize}
\item I will use this recommender again.*
\item I will use this type of recommender frequently.
\item I prefer to use this type of recommender in the future.
\end{itemize}

\subsection{Recommendation to Friends}

\begin{itemize}
\item I will tell my friends about this recommender.*
\end{itemize}

\subsection{Purchase Intention}

\begin{itemize}
\item I would buy the items recommended, given the opportunity.*
\end{itemize}

  \include{Partes/aneB-desafios}
\end{anexosenv}

\end{document}